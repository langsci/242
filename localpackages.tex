% add all extra packages you need to load to this file  
%\usepackage{tabularx} 
\usepackage{forest}
\forestset{
  nice nodes/.style={
    for tree={
      inner sep=0.75pt, s sep=10pt, 
      fit=band,
    },
  },
  default preamble=nice nodes,
}

\forestset{
fairly nice empty nodes/.style={
            delay={where content={}{shape=coordinate,for parent={
                  for children={anchor=north}}}{}}
, angled/.style={content/.expanded={$<$\forestov{content}$>$}}
}}



\useforestlibrary{linguistics}
\forestapplylibrarydefaults{linguistics}
% \usepackage{tipa}


%\usepackage{tikz}
%\usepackage{tikz-qtree}

%\usepackage[normalem]{ulem}
  %\renewcommand{\ULthickness}{1.1pt} %thick line, remove for a thin line
%%%%%%%%%%%%%%%%%%%%%%%%%%%%%%%%%%%%%%%%%%%%%%%%%%%%
%%%                                              %%%
%%%           Examples                           %%%
%%%                                              %%%
%%%%%%%%%%%%%%%%%%%%%%%%%%%%%%%%%%%%%%%%%%%%%%%%%%%% 
%% to add additional information to the right of examples, uncomment the following line
%\usepackage{jambox}
%% if you want the source line of examples to be in italics, uncomment the following line
% \renewcommand{\exfont}{\itshape}
\usepackage{./langsci/styles/langsci-optional}
%\usepackage{./langsci/styles/langsci-gb4e}  %disable this if \linguex is activated and vice versa
\usepackage{./langsci/styles/langsci-lgr}
\usepackage{./langsci/styles/langsci-glyphs}

%\usepackage[english]{babel}
\usepackage{tikz}
\usetikzlibrary{matrix,calc}
\newcommand\tikzmark[1]{\tikz[remember picture, baseline=(#1.base)] \node[anchor=base,inner sep=0pt, outer sep=0pt] (#1) {#1};}

\tikzset{
    ncbar angle/.initial=90,
    ncbar/.style={
        to path=(\tikztostart)
        -- ($(\tikztostart)!#1!\pgfkeysvalueof{/tikz/ncbar angle}:(\tikztotarget)$)
        -- ($(\tikztotarget)!($(\tikztostart)!#1!\pgfkeysvalueof{/tikz/ncbar angle}:(\tikztotarget)$)!\pgfkeysvalueof{/tikz/ncbar angle}:(\tikztostart)$)
        -- (\tikztotarget)
    },
    ncbar/.default=0.5cm,
}


\newcommand{\arrow}[2]{\begin{tikzpicture}[remember picture,overlay]
\draw[->,shorten >=3pt,shorten <=3pt] (#1.base) to [ncbar=\arrowht] (#2.base);
\end{tikzpicture}
\setlength{\arrowht}{0ex}
}

\usepackage{multicol}
\usepackage{xcolor,soul}
\sethlcolor{lightgray}

\usepackage{linguex}
 
% \usepackage[hang,flushmargin]{footmisc}
% \setlength\footnotemargin{10pt}

 
