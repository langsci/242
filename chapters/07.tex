\chapter{Beyond Slavic: Sorting out a Latvian paradigm}\label{chapter:latvian}


\section{Introduction}
We expect the proposed hierarchy in \Next (repeated from the previous chapter) to hold outside \ili{Slavic}, too, irrespective of whether  indefinite demonstratives are morphologically contained in the bigger categories of this sequence, like it is in the case of \ili{Russian} \textit{\v{c}to}, or not.

\ex.\label{milito} 
Dem\textsubscript{def}\,$>$\,Comp\,$>$\,Rel\,$>$\,Wh\,$>$\,Dem\textsubscript{indef}

\noindent In Chapter \ref{chapter:nanosyntax} we discussed the reason why morphological \isi{containment} is a possible but not a necessary effect of the presence of a particular category in an fseq.  Namely, morphological containment is either a result of \isi{spell-out} driven movement or the formation of the left branch (the ``pre-'' distribution in morphosyntax). Both these operations that are both ranked after \textsc{stay} in the spell-out procedure.\footnote{The term ``spell-out driven movement'' is understood here as a cover term for all three kinds of movement subsumed in the  \textsc{move} leg of the spell-out scheme: spec-to-spec movement, snowballing, and \isi{subextraction}.
} %end of fn 
This means that the layers of the sequence of heads lexicalized by \textsc{stay} will not visibly (i.e. morphologically) contain the smaller categories of the same sequence of projections in syntax.
\par
Incorporating Dem\textsubscript{indef} into the bottom of the fseq that covers \isi{syncretism}s with the declarative \isi{complementizer} makes a correct prediction about a curious \isi{paradigm} found in \ili{Latvian} (\ili{Baltic}). In Latvian, the nominative case marker \textit{-s} is part of the morphological structure of Dem, Wh, and Rel, but it is absent from the morphological structure of Comp.  This is shown in \tabref{Lat:problem}. While  \ili{Latvian} does not have definite articles, it marks definiteness on adjectives to the effect that the contrast between definite and indefinite noun phrases is fully meaningful, as shown in \ref{conc1}\,(see for instance \citealt{Budina1966,Nau1998,Praulins2012}, among others).\pagebreak

\begin{table}
\caption{Latvian}
\label{Lat:problem}
\begin{tabular}[h]{ l l l l l l }
\lsptoprule
\textsc{dem} 	& \textsc{comp} 	& \textsc{rel}  	& \textsc{wh}\\	
\midrule
ta-s & \hl{ ka } & \hl{ ka }-s & \hl{ ka }-s\\
\lspbottomrule
\end{tabular}
\end{table}


\ex. \ili{Latvian} (\citealt[84]{Lyons1999})\label{conc1}
\ag. liel-s kok-s\\
big-\textsc{nom} tree-\textsc{nom}\\
\strut `a big tree'
\bg. liel-ai-s kok-s\\
big-\textsc{def-nom} tree-\textsc{nom}\\
\strut `the big tree'\label{Lat:koks}

Despite this fact, the arrangement of the \ili{Latvian} \isi{paradigm} in the way shown  in \tabref{Lat:problem}  creates a problem since the case suffix \textit{-s} is present in three non-adjacent cells. While this is not an instance of the \isi{*ABA} violation since the \textit{-s}  represents the same (non-syncretic) nominative marker in all the cells, it is unexpected for the case marker to be absent on a category (Comp) that is sandwiched in the paradigm by the categories this case marker is a part of (Dem and Rel). 
\par
Let us discuss how the sequence in \ref{milito} and the representation of polymorphemic categories as  singleton projection lines in syntax help us describe the \ili{Latvian} paradigm in a more insightful way.

\section{Latvian demonstratives}\largerpage

While \ili{Latvian} does not have articles, it morphologically distinguishes between definite and indefinite adjectives, often described as long and short forms. Just like \ili{Latvian} nouns, they are inflected for case (see for instance \citealt[57--58]{Mathiassen1997}). The definite marker can be identified as suffix \textit{-ai} or \textit{-aj}, which is placed between the adjectival \isi{root} and the case suffixes, as illustrated in \tabref{Lat:adj-decl} on the example of the masculine declension of the adjective \textit{labs} `good' (examples from \citealt[293--294]{Eckert-etal-1994}). \is{demonstrative}

\begin{table}
\caption{Declension of the \ili{Latvian} \textit{labs} `good'}
\label{Lat:adj-decl} 
\begin{tabular}[h!]{ l l l l l l }
\lsptoprule	
\multicolumn{5}{  c  }{\hskip 0.35cm \textsc{singular} \hskip 2.35cm \textsc{plural}}\\
 				&  \textsc{indef.}	& \textsc{def.} 	 	&\textsc{indef.}	& \textsc{def.}\\\midrule
\textsc{nom} 	&	lab-s		& 	lab-ai-s				&	lab-i		& 	lab-ie\\
\textsc{acc} 	& 	lab-u		& 	lab-o					&	lab-us	& 	lab-os\\
\textsc{gen} 	&	lab-a		& 	lab-\={a}				&	lab-u		& 	lab-o\\
\textsc{dat} 	&	lab-am	& 	lab-aj-am 				&	lab-iem	& 	lab-aj-iem \\
\textsc{loc} 	&	lab-\={a} & lab-aj-\={a} & lab-os	& 	lab-aj-os\\
\lspbottomrule	
\end{tabular}
\end{table}

\par \ili{Latvian} morphologically distinguishes between two forms of the \isi{demonstrative}: the proximal \textit{\v{s}is} and the medial/distal \textit{tas} (e.g. \citealt{Budina1966}; \citealt[111]{Lyons1999}). The definite function of the long form of the adjective is further manifested by the fact that an occurrence of the medial/distal demonstrative \textit{tas} together with an adjective, requires the adjective to come in the definite form. This is illustrated in \ref{dio}.


\ex.\label{dio} \ili{Latvian} (\citealt[318]{Fennell-Gelsen-1980})
\ag. Kur ir tas vec-ai-s kok-s?\\
where is \textsc{dem} old-\textsc{def}-\textsc{nom} tree-\textsc{nom}\\
\strut `Where is that old tree?'\label{Lat:koks}
\bg. Ko tu lasi taj\={a}s jaun-aj-\={a}s gr\={a}mat-\={a}s?\\
what  you read those new-\textsc{def}-\textsc{loc} book-\textsc{loc}\\
\strut `What are you reading in those new books?'

\noindent
In a similar way to what we have observed on the examples of \ili{Polish} and \ili{Russian}, \ili{Latvian} \isi{demonstrative} pronouns \textit{tas} and \textit{\v{s}is} can be decomposed into spatial deictic stems and case suffixes: \textit{ta-s} and \textit{\v{s}i-s} in the nominative. This is so since they are inflected just like possessive pronouns, as shown in \tabref{Lat:cases}.\footnote{Let us take note of the fact that Tables \ref{Lat:adj-decl} and \ref{Lat:cases} list only a subset of exponents while \ili{Latvian} distinguishes three masculine and three feminine declensions. The list provided here, however, is sufficient to identify case marking on the demonstratives.
} %end of fn
The demonstratives share the same declension class with \textit{kas}, a syncretic form for Wh/Rel. \textit{Kas}, however, appears only in the singular and the locative adverb \textit{kur} `where' is used in the locative, as shown in \tabref{kas:cases}. 

\begin{table}
\caption{Masculine declension of the \ili{Latvian} \isi{demonstrative}s: distal/medial \textit{tas} and proximal \textit{\v{s}is}}
\label{Lat:cases} 
\begin{tabular}[th]{ l l l l l l }
\lsptoprule	
 			&  \textsc{sg}	& \textsc{pl} & & \textsc{sg} & \textsc{pl}\\\midrule	
\textsc{nom} 	&	ta-s					& t-ie 	& & \v{s}i-s	& \v{s}i-e\\
\textsc{acc} 	& 	t-o					& t-os 	& & \v{s}-o 	& \v{s}-os\\
\textsc{gen} 	&	t-\={a}	& t-o 	& & \v{s}-\={a}, \v{s}-\={i} & \v{s}-o\\
\textsc{dat} 	&	t-am					& t-iem 	& & \v{s}-im	& \v{s}-iem\\
\textsc{loc} 	&	ta-j\={a} & ta-is 	& & \v{s}a-j\={a} & \v{s}a-jos\\
\lspbottomrule	
\end{tabular}
\end{table}

\begin{table}
\caption{Singular declension of the \ili{Latvian} syncretic Wh/Rel \textit{kas}}
\label{kas:cases} 
\begin{tabular}[t]{ l l l l l l }
\lsptoprule	
\textsc{nom} 	&	ka-s\\
\textsc{acc} 	& 	k-o\\
\textsc{gen} 	&	k-\={a}\\
\textsc{dat} 	&	k-am\\\midrule
\textsc{loc}	&	k-ur `where'\\
\lspbottomrule
\end{tabular}
\end{table}
 
\par Let us consider the \ili{Latvian} declarative \isi{complementizer} \textit{ka}.


\ex. \ili{Latvian} declarative complementizer \textit{ka}\label{ex:ka} (\citealt[229]{Holvoet2016})
\ag.[]\hspace{-22pt}Es zinu  ka tu atbrauksi paciemoties \\
\hspace{-22pt}I know.\textsc{1sg}   \textsc{comp} you come.\textsc{fut.2sg} visit.\textsc{inf}\\
\hspace{-22pt}\strut `I know you will come on a visit.'

\noindent Unlike the demonstratives \textit{tas}, \textit{\v{s}is} and the syncretic Wh/Rel  \mbox{\textit{kas}}, the \isi{complementizer} \textit{ka} is uninflected for case. 
This situation contrasts with complementizers such as the \ili{Russian} \textit{\v{c}to} or the \ili{Serbo-Croatian} \textit{\v{s}to}, which include a neuter nominative case suffix \textit{-o}, and also the \ili{Polish} complementizer \textit{\.ze}.\footnote{Assuming with \cite{BaunazLander2018} that \textit{\.ze} should be analyzed as a bi-morphemic \textit{\.z-e}, where the usual neuter nominative case suffix \textit{-o} surfaces as /e/ after a soft consonant \textit{\.z}- /ʒ/ (see Footnote \ref{FN:ze} in  Chapter \ref{chapter:resolving}).
} %end of a new fn
The fact that the \ili{Latvian} declarative complementizer \textit{ka} lacks the invariant case suffix leads to an interesting observation: while the \ili{Latvian} noun phrase such as e.g. `that old tree' in \ref{Lat:koks} includes a definite marker in its structure, this marker must be distinct from the Def category of the ``Dem\textsubscript{def}\,$>$\,Comp\,$>$\,Rel\,$>$\,Wh\,$>$\,Dem\textsubscript{indef}'' sequence. This follows from the fact that equating the adjectival definite marker with the Def category in our sequence results in the arrangement of the \isi{paradigm} as in \tabref{Lat:problem}. In \tabref{Lat:problem}, on the one hand Comp is a category intermediate in terms of complexity and on the other hand it is the only category which does not comprise the case marker.

This puzzle becomes less absorbing if the \ili{Latvian} \isi{demonstrative}, which itself does not comprise the definite marker, instead corresponds to the Dem\textsubscript{indef} at the bottom of our fseq, yielding the order as in \tabref{Lat:solution}. When compared to the arrangement in \tabref{Lat:problem}, the one in \tabref{Lat:solution} keeps the syncretic span of the stems of Comp$=$Rel$=$Wh and groups the case-inflected categories into a different span including Rel, Wh, and Dem.

\begin{table}
\caption{Reordered paradigm in Latvian}
\label{Lat:solution}
\begin{tabular}[h]{ l l l l l l }
\lsptoprule
\textsc{comp} 	& \textsc{rel}  	& \textsc{wh} & \textsc{dem} \\
\midrule	
\hl{ ka } & \hl{ ka }-s & \hl{ ka }-s & ta-s\\
\lspbottomrule
\end{tabular}
\end{table}

    
While the arrangement of the \isi{paradigm} as in \tabref{Lat:solution} by itself does not provide an answer to the question why the \ili{Latvian} declarative \isi{complementizer} \textit{ka} does not take any (invariant) case suffix the way other languages we have so far looked at do, it at least allows us to identify the pattern in the noise.
\par
What has helped us resolve the morphological \isi{containment} problem of indefinite demonstratives in \ili{Slavic} is the idea that an underlying syntactic representation of morphologically complex categories in the sequence ``Dem\textsubscript{def}\,$>$\,Comp\,$>$ Rel\,$>$\,Wh\,$>$\,Dem\textsubscript{indef}'' has a shape of singleton projection line.  
Such a simplex sequence becomes partitioned into geometrically more complex trees only as a result of spell-out driven operations.
Let us now move on to consider how this sequence is lexicalized and extended by the case \isi{feature}(s) in \ili{Latvian}, bearing in mind that -- just like in \ili{Russian} and \ili{Polish} but unlike in \ili{Germanic} -- it reaches only up to the CompP layer in \ili{Latvian} and does not include the top Def layer.

\section{Refining the pronominal base}

The comparison of \textit{tas} and \textit{kas} with other interrogative pronouns suggests that the stems for the merger of the case suffix are morphologically complex, too. Namely, while \textit{kas} is a syncretic form for `what' and `who', the forms of other interrogative pronouns in \ili{Latvian} comprise the initial \textit{k-} and a different ending, as listed in \tabref{Lat:wh-pronouns}. \is{wh-pronoun}

\begin{table}
\caption{\ili{Latvian} interrogative pronouns}
\label{Lat:wh-pronouns}
\begin{tabular}[t]{ l l l l l l }
\lsptoprule	
kas & `what', `who'\\
kur & `where'\\
k\={a} & `how'\\
k\={a}p\={e}c & `why'\\
\lspbottomrule
\end{tabular}
\end{table}

If \textit{k-} is a wh-prefix added to different stems in the formation of interrogative pronouns, then the \ili{Latvian} pattern adheres to what we find throughout \ili{Indo-European}, including the English pattern involving \textit{wh-at, wh-o, wh-ich, wh-en, wh-ere}.\footnote{To a large extent, this pattern is also present in \ili{Slavic} but it can be sometimes blurred by phonological factors. In \ili{Polish} for instance, the personal interrogative pronoun \textit{kto} `who' includes the wh-prefix \textit{k-}, which is present in \textit{k-iedy} `when' and \textit{k-\k{e}dy }`through where' but, as stated in \cite{Wiland-PSiCL}, it is also present in forms such as \textit{g-dzie} `where' or \textit{g-dy} `when', where /g/ is a voiced allomorph \is{allomorphy} of /k/ appearing before a voiced /d/ in the onset of the stem. Also, the form of the \ili{Polish} \textit{do-k-\k{a}d} `where to', as in \Next, includes the interrogative prefix \textit{k-}, which is merged directly with the locative stem, and the external prefix denoting path \textit{do-} `to'.

\noindent\parbox{\linguexfootnotewidth}{\ex. \ili{Polish}
\ag.[]\hspace{-22pt}Dok\k{a}d idziecie?\\
\hspace{-22pt}where.to go.\textsc{2pl}\\
\hspace{-22pt}\strut `Where are you going to?'


}} %end of fn
\par
This leads us to a tri-mor\-phe\-mic analysis of the \ili{Latvian} \textit{t-a-s} and \textit{k-a-s} in a similar way to the \ili{Russian} \textit{\v{c}-t-o} `what', with -- in the case of \textit{k-a-s} -- more than one syncretic \isi{morpheme} in its structure. Apart from the syncretic prefix \textit{k-} covering Wh, Rel, and Comp, also the nominal stem \textit{-a}, which is the base for the merger of \textit{t-} and \textit{k-} in \textit{t-a-s}/\textit{k-a-s}, must be syntactically complex since \textit{kas} is syncretic for `what' and `who'. In this respect the \ili{Latvian} \textit{kas} stands out from a well-attested pattern where the stems for the wh-prefix in morphological forms of kind and person queries are non-syncretic (including the English \textit{wh-at}, \textit{wh-o} or the Italian \textit{ch-e} `what', \textit{ch-i} `who'). 
\par
We can fairly straightforwardly account for the complexity of the \ili{Latvian} stem \textit{-a} by identifying it as an internally complex NP, the (pro)nominal base component in our fseq. The  fseq, repeated in \ref{repe} for convenience, projects only up to the Comp layer in \ili{Latvian} and it excludes Def, the top-most ingredient whose presence results in the formation of definite \isi{demonstrative}s, which \ili{Latvian} lacks.

\ex.\label{repe} 
\begin{forest}nice empty nodes, for tree={l sep=0.65em,l=0,calign angle=63}
 [CompP [Comp]
 [RelP [Rel]
 [WhP
 [Wh ] [\hspace{10pt}Dem\textsubscript{indef} 
 [Dem] [NP]]
 ]]
 ]
 ]
\end{forest}


\noindent The complexity of Dem\textsubscript{indef} can in principle apply not only to the Dem component but also to its (pro)nominal NP component. That is, the decomposition of the Dem in \ref{repe} into independent features that encode spatial deictic contrast, discussed in \ref{LH:Dem} in Chapter \ref{chapter:resolving}, renders the representation of the Dem\textsubscript{indef} as in \ref{HL-rep}, with deictic features projected on top of the (pro)nominal NP base. 

\ex.\label{HL-rep} 
\begin{forest}nice empty nodes, for tree={l sep=0.65em,l=0,calign angle=63}
 [DistP
 [Deix$_3$ ] [MedP 
 [Deix$_2$] [ProxP
 [Deix$_1$] [NP] ]]]]
 \end{forest} 

\noindent
There exists independent evidence that what we have so far been referring to as the (pro)nominal NP base in the structure of Dem\textsubscript{indef} has it own complex structure, too. Namely, the decomposition of the NP base into a sequence of nominal \isi{feature}s N\textsubscript{n} as in \ref{refined-NP} captures the different sizes of stems present in \isi{wh-pronoun}s denoting Thing (`what'),  Person (`who'), and Place (`where').

\ex.\label{refined-NP} Refined NP base\\[1ex]
\begin{forest}nice empty nodes, for tree={l sep=0.65em,l=0,calign angle=63}
 [PlaceP, s sep=15pt [N$_{3}$]
 [PersonP, s sep=10pt [N$_{2}$]
 [ThingP [N$_{1}$]]]]
\end{forest}

\noindent An argument in favor of a partial hierarchy in \ref{refined-NP} can be found in \cite{BaunazLanderTUM}, who organize the list of closed class light nouns. The full list includes interrogative words denoting the concepts listed in \ref{lightnouns},  which are organized into a sequence based on their \isi{syncretism}s and morphological \isi{containment}.\footnote{See \citeauthor{Cysouw2004} (\citeyear{Cysouw2004,Cysouw2005}) for the topology of wh-pronouns including Thing, Person, and Person wh-queries. See also \cite{Vangsnes2013}, who on the basis of syncretic alignment argues that the Person wh-pronuns are syntactically more complex than Thing wh-pronouns in \ili{Germanic}. 
} %end of fn


\ex.\label{lightnouns} 
\a.Thing (`what')
\b. Person (`who')
\c. Place (`where')
\d. Manner (`how')
\e. Amount (`how much/many') 
\f. Time (`when')
 
\noindent 
Let us briefly go through the evidence provided in \cite{BaunazLanderTUM} in support of syntactic inclusion of Thing inside Person and Person inside Place  before we move on to represent the \ili{Latvian} \textit{kas} as a form which comprises subsets of \ref{refined-NP} in its syntactic structure. 
\par
The argument in favor of the inclusion of Thing inside Person wh-queries comes from morphological containment found in \ili{Amuecha} (\ili{Arawakan}) and in \ili{Muna} (\ili{Austronesian}), as shown in:

\ex. \ili{Amuecha} (\citealt[573]{Wise1986} as cited in \citealt{Cysouw2004})\\[0.25ex]
\begin{tabular}[t]{ l l l l l l }
es & \textsc{thing}\\
es-e\v{s}a  & \textsc{person}\\
\end{tabular}

\ex. \ili{Muna} (\citealt[\S8.6.2]{VanDenBerg1989} as cited in \citealt{BaunazLanderTUM})\\[0.25ex]
\begin{tabular}[t]{ l l l l l l }
hae & \textsc{thing}\\
la-hae  & \textsc{person}\\
\end{tabular}

In turn, the argument in favor of the inclusion of Person inside Place inside wh-queries comes from morphological \isi{containment} found for instance in \ili{Sanum\'a} (\ili{Yanomaman}) and \ili{Pipil} (\ili{Uto-Aztecan}):

\ex. \ili{Sanum\'a} (\citealt[67, 70]{Borgman1990})\\[0.25ex]
\begin{tabular}[t]{ l l l l l l }
witi & \textsc{person}\\
witi ha  & \textsc{place}\\
\end{tabular}

\ex. \ili{Pipil} (\citealt[114]{Campbell1985})\\[0.25ex]
\begin{tabular}[t]{ l l l l l l }
ka: & \textsc{person}\\
ka:n  & \textsc{place}\\
\end{tabular}

\noindent
Apart from morphological containment, an argument for the `Place\,$>$\,Person\,$>$ Thing' sequence comes from syncretic alignment. \cite{BaunazLanderTUM} note that there are cross-lin\-guis\-tic\-ally attested syncretisms between the Person query and the Place query, as for instance in \ili{Awa Pit} (\ili{Barbacoan}).



\ex. \ili{Awa Pit} (\citealt[225]{Curnow2006})\label{AwaPit}\\[0.25ex]
\begin{tabular}[t]{ l l l l l l }
shi & \textsc{thing}\\
m\sout{i}n & \textsc{person}\\
m\sout{i}n=  & \textsc{place}\\
\end{tabular}


\noindent At the same time \isi{syncretism} involving the Thing query and the Place query to the exclusion of the Person query has not been attested.\footnote{The Person$=$Place syncretism can also be found in \ili{Modern \ili{Greek}} if we qualify the dative \textit{p\'u} `to whom' as a Person wh-query in sentences as \ref{person:pu}:

\noindent\parbox{\linguexfootnotewidth}{\ex. \ili{Modern \ili{Greek}} \textit{p\'u} `where'/`to whom' (\citealt[ex. 12]{Roussou2016})
\ag. P\'u pas?\\
where go.\textsc{2sg}\\
\strut `Where are you going?'
\bg. P\'u to edhoses?\\
where it gave.\textsc{2sg}\\
\strut `Who did you give it to?'\label{person:pu}

}} %end of fn
Given the \isi{*ABA} generalization, the structure of Person comes out as intermediate in terms of syntactic complexity between Place and Thing.
\par
To summarize, while syncretism indicates that the three forms constitute a paradigm with the Person-cell intermediate in terms of complexity, as in \tabref{stonoga} morphological \isi{containment} facts indicate that Place is more complex than both Person and Thing, as indicated in the fseq \ref{refined-NP}.

\begin{table} 
\caption{Syncretic alignment of wh-pronouns}
\label{stonoga}
\begin{tabular}[h]{ l l l l l l }
\lsptoprule
			& \textsc{place} & \textsc{person}  	& \textsc{thing}\\	
\midrule
English   		& where 		& who 			& what\\
\ili{Latvian}	& kur			& kas\cellcolor[gray]{0.9} & kas\cellcolor[gray]{0.9}\\
\ili{Awa Pit}	& m\sout{i}n=\cellcolor[gray]{0.9}	& m\sout{i}n\cellcolor[gray]{0.9}& shi\\
unattested	& \cellcolor[gray]{0.9}			&	&\cellcolor[gray]{0.9}\\
\lspbottomrule
\end{tabular}
\end{table}

\noindent The essential difference between the \ili{Latvian} \textit{kas} and forms for `what' and `who' in languages like \ili{Amuecha} or \ili{Muna} is two-fold: the \textit{k-} marker in \textit{kas} is a prefix and the \textit{-a} is a syncretic stem. Given the refined nominal base in \ref{refined-NP}, we are able to describe the lexical entry for the \ili{Latvian} pronominal stem \textit{-a} as comprising the two bottom layers of \ref{refined-NP}, as specified in \ref{Lat:lex-a}, to the exclusion of a separate \textit{k-} prefix, as specified in:

\ex. Lexical entry for the \ili{Latvian} pronominal stem \textit{-a}\label{Lat:lex-a}\\[0.5ex]
[\textsubscript{PersonP} N$_{2}$ [\textsubscript{ThingP} N$_{1}$ ]] $\Leftrightarrow$ \textit{a}

Such an entry not only allows us to straightforwardly derive \textit{k-a-s}, the syncretic form for `what', `who', and Rel, but also to explain the contrast in the morphological structure between the medial/distal demonstrative pronoun \textit{t-a-s} and the proximal demonstrative pronoun \textit{si-s}, which has a mono-morphemic stem. 
\par
Let us discuss the structure and \isi{spell-out} of the proximal \textit{\v{s}i-s} first, since the medial/distal \textit{t-a-s} and the Wh/Rel \textit{k-a-s} include bigger structures that build up on the structure of \textit{\v{s}is}.

\section{Proximal \textit{\v{s}is} and medial \textit{tas}}

Assuming the decomposition of \isi{demonstrative}s in \cite{Lander-Haegeman2016} in \ref{HL-rep} and the refinement of the pronominal stem in \ref{refined-NP}, the syntactic representation of proximal demonstrative pronouns minimally includes the pronominal Thing-forming \isi{feature} N$_{1}$ and the Prox-forming feature Deix$_{1}$, as in \ref{18}. On the strength of the \isi{Superset Principle}, the Thing layer of such a representation is realized as \textit{-a} as the subset spell-out of the lexical entry in \ref{Lat:lex-a}. 

\ex.\label{18} 
\begin{forest}nice empty nodes, for tree={l sep=0.75em,l=0,calign angle=63}
 [ProxP [Deix$_{1}$]
 [ThingP [N$_{1}$]]{\draw (.east) node[right]{$\Rightarrow$ \textit{a}}; } ]
\end{forest}


\noindent In order to lexicalize the Prox layer of this structure there needs to exist another lexical entry  in the \ili{Latvian} lexicon: the one which includes the Deix$_{1}$ \isi{feature}. While the lexical entry for \textit{-a} in \ref{Lat:lex-a} lacks Deix$_{1}$, the lexical entry for the proximal stem \textit{\v{s}i}- defined as in \ref{Lat:prox} includes it.

\ex. Lexical entry for the \ili{Latvian} (uninflected) proximal \isi{demonstrative}\\ 
pronoun \textit{\v{s}i-}\label{si}\label{Lat:prox}\\[0.5ex]
[\textsubscript{ProxP} Deix$_{1}$ [\textsubscript{ThingP} N$_{1}$ ]] $\Leftrightarrow$ \textit{\v{s}i}

The insertion of \textit{\v{s}i-} in the ProxP node in the syntactic representation results in the over-riding \is{Cyclic Over-ride} of \textit{-a}, as shown in \ref{si-}.\largerpage

\ex.\label{si-} Spell-out of the \ili{Latvian} proximal \isi{demonstrative} stem \textit{\v{s}i-}\\[0.5ex]
\begin{forest}nice empty nodes, for tree={l sep=0.75em,l=0,calign angle=63}
 [ProxP [Deix$_{1}$]
 [ThingP [N$_{1}$]]{\draw (.east) node[right]{$\Rightarrow$ \textit{a}}; }
 ]{\draw (.east) node[right]{$\Rightarrow$ \textit{\v{s}i}}; }
\end{forest}

In this way, \textit{\v{s}i-} comes out as a portmanteau stem that realizes the pronominal base and the proximal deictic \isi{feature}. 
\par
The pronominal base, however, is visibly retained in other forms in \ili{Latvian}. 
Whereas the proximal feature is realized in the stem of \textit{\v{s}i-s}, the medial feature is realized in the prefix \textit{t-} in the demonstrative \textit{t-a-s}, not in the stem \textit{-a}. The lexicalization of the medial feature as part of the stem would result in an ABA pattern, \is{*ABA} as it requires the realization of MedP and ThingP as syncretic \textit{-a} to the exclusion of ProxP, which is intermediate in terms of complexity, as \textit{\v{s}i}-, as outlined in \ref{death}.

\ex.\label{death} Unattested \isi{spell-out} (\isi{*ABA} violation)\\[0.5ex]
\begin{forest}nice empty nodes, for tree={l sep=0.65em,l=0,calign angle=63}
 [*MedP [Deix$_{2}$]
 [ProxP [Deix$_{1}$]
 [ThingP [N$_{1}$]]{\draw (.east) node[right]{$\Rightarrow$ \textit{a}}; }
 ]{\draw (.east) node[right]{$\Rightarrow$ \textit{\v{s}i}}; }
 ]{\draw (.east) node[right]{$\Rightarrow$ \textit{a}}; }
\end{forest}

The preservation of the \textit{-a} stem in \textit{t-a-s} indicates that there is no lexical item in the \ili{Latvian} lexicon which realizes both the pronominal base ThingP and the Med-forming feature Deix$_{2}$. In turn, the lack of morphological containment of \textit{\v{s}i-} in the structure of \textit{t-a-s} indicates that the \textit{t-a-} sequence is not derived by \textsc{move} but by a last resort \textsc{subderive}, which results in the merger of an XP with the mainline derivation, the procedure resulting in the formation of the complex left branch discussed in \sectref{sec:Starke2018}.
\par
The decomposition of indefinite demonstratives into independent Deix\textsubscript{n} features projected on top of a pronominal structure in \ref{HL-rep} comes out as a necessary result in identifying the base feature for spawning the subderivation. We are only able to capture the distinction between the proximal stem \textit{\v{s}i-} and the medial \mbox{\textit{t-a-}} if it is precisely the proximal \isi{feature} Deix$_{1}$ of the split category Dem\textsubscript{indef} which is provided as the base feature for the formation of the left branch. Its merger with the next feature in line, the Med-forming feature Deix$_{2}$, as shown in \Next, forms an XP constituent that is subsequently merged with the pronominal ThingP of the mainline derivation.

\ex. Spell-out of the \ili{Latvian} medial demonstrative stem \textit{t-a-}\label{Lat:ta}\\[-0.5ex]
\begin{forest}nice empty nodes, for tree={l sep=0.65em,l=0,calign angle=63}
 [~, s sep=20pt
 [MedP [Deix$_{2}$][Deix$_{1}$]]{\draw (.east) node[right]{$\Rightarrow$ \textit{t}}; } 
 [ThingP [N$_{1}$]]{\draw (.east) node[right]{$\Rightarrow$ \textit{a}}; }
 ]
\end{forest}

The left branch of such a tree can be spelled out as \textit{t-} if its lexical specification includes a constituent specified as in \Next, where Deix$_{2}$ and Deix$_{1}$ are sisters.\footnote{In line with \citeauthor{Starke2018}'s \citeyearpar{Starke2018} insight that prefixes but not suffixes have a binary foot in their syntactic representations, a consequence of \textsc{subderive}.}\largerpage[-2]%end of fn

\ex. Lexical entry for the \ili{Latvian} medial prefix \textit{t-}\label{Lat:t}\\[0.5ex]
[\textsubscript{MedP} Deix$_{2}$ Deix$_{1}$ ] $\Leftrightarrow$ \textit{t}

The decomposition of \isi{demonstrative}s in the way seen in \ref{HL-rep} is here necessary since the Prox-forming \isi{feature} Deix$_{1}$ spells out together with ThingP as a single portmanteau \isi{morpheme} \textit{si-} only when there is no higher Med-forming Deix$_{2}$ added to the derivation. The addition of Deix$_{2}$ requires \isi{backtracking} and the formation of the left branch, which becomes merged with the pronominal stem \textit{-a}, the subset of the proximal \textit{si-}. 
\par
Importantly, for the present derivation of \textit{t-a-} to work, the subderivation of the complex left branch in \ref{Lat:ta} must be able to enforce \isi{backtracking}. As pointed out by a reviewer, this is different than in \cite{Starke2018}, where \textsc{subderive} does not involve backtracking. When we compare \textit{\v{s}i-} in \ref{si-} with \textit{t-a-} in \ref{Lat:ta}, for the present analysis to work, the derivation must backtrack down \is{backtracking} to ThingP and start the subderivation of the left branch from that level. If the subderivation started from ProxP, i.e. the stage in \ref{si-}, we would expect an unattested form like \textit{t-\v{s}i-s} to be generated.
\par
The suffixal case marking on \textit{\v{s}i-s} and \textit{t-a-s} follows straightforwardly if the case fseq projects on top of the categories forming the ``Dem\textsubscript{def}\,$>$\,Comp\,$>$\,Rel\,$>$\,Wh\,$>$ Dem\textsubscript{indef}'' sequence rather than directly on the pronominal base, the subset of Dem\textsubscript{indef}. Thus, assuming \ref{postmortem} to be a stand-in entry for the \ili{Latvian} nominative singular marker \textit{-s}, 

\ex.\label{postmortem} 
[\textsubscript{K$_{1}$P} K$_{1}$ ] $\Leftrightarrow$ \textit{s}

the merger of the nominative feature K$_{1}$ on top of the proximal \textit{si-} and the medial \textit{t-a-} becomes spelled out in both instances following complement movement, as illustrated in (\pref{so:Lat:sis}--\pref{so:Lat:tas}).\footnote{The \textit{-s} marker is the nominative exponent of the 1st declension class in the \ili{Latvian} conjugation system, which includes demonstrative pronouns (see \citealt{Mathiassen1997} and \citealt{Nau2011}).
}\pagebreak\largerpage[-2]%end of fn


\ex. Spell-out of the \ili{Latvian} \textit{\v{s}i-s}\label{so:Lat:sis}\\[1ex]
\begin{forest}nice empty nodes, for tree={l sep=0.65em,l=0,calign angle=63}
 [K$_{1}$P [K$_{1}$][ProxP [Deix$_{1}$]
 [ThingP [N$_{1}$]]
 ]{\draw (.east) node[right]{$\Rightarrow$ \textit{\v{s}i}};}
 ]
\end{forest}
\hskip 0.5cm$\leadsto$
\begin{forest}nice empty nodes, for tree={l sep=0.65em,l=0,calign angle=63}
 [~, s sep=25pt [ProxP, name=tgt, s sep=8pt [Deix$_{1}$]
 [ThingP [N$_{1}$]]
 ]{\draw (.east) node[right]{$\Rightarrow$ \textit{\v{s}i}};}
 [K$_{1}$P [K$_{1}$][\ldots, name=t]]{\draw (.east) node[right]{$\Rightarrow$ \textit{s}};}
 ]
\draw[dashed,->,>=stealth,overlay] (t) [in=-165,out=-120,looseness=2]  to (tgt);
\end{forest}


\ex. Spell-out of the \ili{Latvian} \textit{t-a-s}\label{so:Lat:tas} 

\resizebox{\linewidth}{!}{%
\begin{forest}nice empty nodes, for tree={l sep=0.65em,l=0,calign angle=63}
 [K$_{1}$P, s sep=-5pt [K$_{1}$][MedP, s sep=20pt
 [MedP, s sep=4pt [Deix$_{2}$][Deix$_{1}$]]{\draw (.east) node[right]{$\Rightarrow$ \textit{t}}; } 
 [ThingP [N$_{1}$]]{\draw (.east) node[right]{$\Rightarrow$ \textit{a}}; }
 ]]
\end{forest}
$\leadsto$
\begin{forest}nice empty nodes, for tree={l sep=0.65em,l=0,calign angle=63}
 [~, s sep=35pt [MedP, name=tgt, s sep=15pt
 [MedP, s sep=4pt [Deix$_{2}$][Deix$_{1}$]]{\draw (.east) node[right]{$\Rightarrow$ \textit{t}}; } 
 [ThingP [N$_{1}$]]{\draw (.east) node[right]{$\Rightarrow$ \textit{a}}; }
 ]
 [K$_{1}$P, s sep=1pt [K$_{1}$][\ldots, name=t]]{\draw (.east) node[right]{$\Rightarrow$ \textit{s}}; }
 ]
 \draw[dashed,->,>=stealth,overlay] (t) [in=-165,out=-130,looseness=3]  to (tgt);
\end{forest}
}\vspace*{2\baselineskip}

\noindent Let us observe that if case fseq projects on top of the ``Dem\textsubscript{def} $>$\,Comp\,$>$\,Rel\,$>$ Wh\,$>$\,Dem\textsubscript{indef}'' sequence, case suffixation in \textit{\v{s}i-s} and in the complex \textit{t-a-s} is possible only if the left branch constituent \textit{t-} in the second is a complex head. By ``complex head'' I understand the node that  provides its label for the merger with its sister. For \textit{t-a-s}, MedP \textit{t-} must be a head (rather than a non-projecting specifier) on the ThingP stem \textit{-a}. This result is in agreement with \citeauthor{Starke2004}'s \citeyearpar{Starke2004} reanalysis of specifiers as complex heads. If, against this idea, the prefix \textit{t-} in \textit{t-a-s} is a non-projecting specifier and what projects is the pronominal ThingP \textit{-a}, the case fseq will have to apply to the latter. Such an alternative is illustrated in \ref{burn}.\largerpage[3]

\ex.\label{burn} Unattested sequence K$_{1}$P\,$>$\,ThingP\,$>$\,MedP derived by non-projecting left branches\label{unatt-fseq}\\[0.75ex]
\begin{forest}nice empty nodes, for tree={l sep=0.65em,l=0,calign angle=63}
 [K$_{1}$P, s sep=-5pt [K$_{1}$][ThingP, s sep=25pt
 [MedP [Deix$_{2}$][Deix$_{1}$]]{\draw (.east) node[right]{$\Rightarrow$ \textit{t}}; } 
 [ThingP [N$_{1}$]]{\draw (.east) node[right]{$\Rightarrow$ \textit{a}}; }
 ]]
\end{forest}

 
\noindent The scenario with non-projecting left branches in \ref{unatt-fseq} would create a contradictory situation: we would have one sequence ``K$_{1}$P\,$>$\,ProxP\,$>$\,ThingP'' for the proximal \textit{\v{s}is} and another sequence ``K$_{1}$P\,$>$\,ThingP\,$>$\,MedP'' for the medial \textit{tas}. With ThingP listed as smaller than ProxP in the first and as bigger than MedP in the second, we would incorrectly expect to have a sequence ``ProxP\,$>$\,ThingP\,$>$\,MedP'', suggesting that proximal demonstratives structurally contain medial demonstratives. The evidence for \ref{HL-rep} discussed in \cite{Lander-Haegeman2016} shows the opposite to be true.
We avoid this contradiction if we follow \ref{so:Lat:tas}, where left branches formed by \textsc{subderive} are complex heads.\footnote{If we return to the discussion of \isi{spell-out} driven extraction in the domain of \ili{Czech} and \ili{Polish} \isi{semelfactive} \textit{-n-ou} stems in Chapter \ref{chapter:explaining}, we can observe the difference between the projecting vs. non-projecting status of specifier-like XPs. In semelfactives like the \ili{Czech} \textit{kop-n-ou-t} `give a kick', following the roll-up derivation, the constituent \textit{kop-n-} ends up as non-projecting specifier of the verbalizing theme \is{thematic suffix} vowel \textit{-ou}. Thus, distinction between projecting and non-projecting  specifier-like XPs appears to be running along the following description: internally merged XPs form non-projecting specifiers whereas externally merged XPs are complex heads. See also \cite{CCW2017}, who reach the same conclusion about projecting vs. non-projecting specifiers in the domain of \ili{Czech} comparative morphology.
} %end of fn    

\section{Deriving the three readings of \textit{kas}}\label{section:kas}

\textit{Kas} is a declinable syncretic form for \isi{wh-pronoun}s denoting Thing (`what') and Person (`who') as well as the relative pronoun, \is{relativizer} as shown below for nominative \textit{kas}, accusative \textit{ko}, and genitive 
\textit{k\={a}}.\footnote{Both Wh and Rel \textit{kas} are inflected for all the cases in the \ili{Latvian} \isi{paradigm}, as shown in \tabref{kas:cases} above, but the use of the genitive form of Rel is rare. However, it is nevertheless possible in contexts such as in the following:

\noindent\parbox{\linguexfootnotewidth}{\ex. \ili{Latvian} (Nicole Nau, p.c.)
\ag.[]\hspace{-22pt}suns, no k\={a} man bail\\
\hspace{-22pt}dog.\textsc{nom} of \textsc{rel.gen} me.\textsc{dat} afraid.\textsc{1sg.pres}\\
\hspace{-22pt}\strut `the dog of which I am afraid'

}}%end of fn

\ex. \ili{Latvian} \textit{kas} as pronominal `what' (\citealt{Praulins2012})
\ag. Kas vainas?\\
what.\textsc{nom} fault.\textsc{acc}\\
\strut `What's the matter?'
\bg. Ko j\={u}s dar\={a}t?\\
what.\textsc{acc} you.\textsc{2pl} do.\textsc{2pl.pres}\\
\strut `What are you doing?'

\ex. \ili{Latvian} \textit{kas} as pronominal `who'
\ag. Kas nozaga manu maku?\\
who.\textsc{nom} stole my wallet.\textsc{acc}\\
\strut `Who stole my wallet?'
\bg. K\={a} cepure ir t\={a}?\label{conc2}\\
who.\textsc{gen} hat.\textsc{nom} be.\textsc{3sg.pres} that.\textsc{gen}\\
\strut `Whose hat is that?'

\ex.  \ili{Latvian} \textit{kas} Rel (Tatjana Navicka,\ia{Navicka, Tatjana@Navicka, Tatjana} p.c.; \citealt{Nau2009})
\ag. cilv\={e}ks kas tur s\={e}\v{z}\label{conc3}\\
man.\textsc{nom} \textsc{rel.nom} there sit.\textsc{3sg.pres}\\
\strut `the man who is sitting there'
\bg. Vai ir k\={a}ds liels sapnis, ko grib\={e}tos \={i}stenot?\\
\textsc{prt} be.\textsc{3sg.pres} any great dream \textsc{rel.acc} want.\textsc{2pl} realize.\textsc{inf}\\
\strut `Do you have any great dream you want to realize?'


\noindent If both Wh and Rel are based on the indefinite medial \isi{demonstrative}, we can straightforwardly derive the Wh$=$Rel \isi{syncretism} of \textit{kas} by extending the structure of \textit{t-a-} in \ref{Lat:ta} by adding the higher \isi{feature}s Wh and Rel as shown in \ref{sub:k} below.  More specifically, features Wh and Rel must belong to the lexical entry for \textit{k-} (as in \pref{old:k}), which is bigger than the entry for \textit{t-} (in \ref{Lat:t} above). 

\ex. Lexical entry for the \ili{Latvian} prefix \textit{k-} (1st approximation)\label{old:k}\\[0.5ex]
[ Rel [ Wh [\textsubscript{MedP} Deix$_{2}$ Deix$_{1}$ ]]] $\Leftrightarrow$ \textit{k}

This can be inferred from the fact that, given the `Rel\,$>$\,Wh\,$>$\,Dem\textsubscript{indef}' sequence, \textit{k-} over-rides \textit{t-} to the exclusion of the stem \textit{-a}. \is{Cyclic Over-ride}

\ex. Spell-out of the \ili{Latvian} Wh/Rel \textit{k-a-} \label{sub:k}\\[0.5ex]
\begin{forest}nice empty nodes, for tree={l sep=0.65em,l=0,calign angle=63}
[RelP, s sep=25pt [RelP [Rel][WhP [Wh]
[MedP [Deix$_{2}$][Deix$_{1}$]
]{\draw (.east) node[right]{$\Rightarrow$ \textit{t}}; }
]]{\draw (.east) node[right]{$\Rightarrow$ \textit{k}}; }
[ThingP [N$_{1}$]]{\draw (.east) node[right]{$\Rightarrow$ \textit{a}}; }
]
 \end{forest}

With the lexical entry in \ref{old:k}, the Rel$=$Wh syncretism of \textit{kas} results from the subset \isi{spell-out} of \textit{k-} as Rel or its Wh subset (while the stem \textit{-a} is invariant in both categories). 
\par
The subsequent merger and spell-out of the case fseq on top of \textit{k-a-} takes place exactly as in \textit{tas} in \ref{so:Lat:tas}, as shown below with the suffix \textit{-s} spelling out the nominative \isi{feature} K$_{1}$ following complement movement.

\ex. Spell-out of the \ili{Latvian} Wh/Rel \textit{k-a-s}\label{so:kas}\\[-0.5ex]
{\small \begin{forest}nice empty nodes, for tree={l sep=0.65em,l=0,calign angle=63}
 [~, s sep=45pt [RelP, name=tgt [RelP [Rel][WhP [Wh]
 [MedP [Deix$_{2}$][Deix$_{1}$]]{\draw (.east) node[right]{$\Rightarrow$ \textit{t}}; }
 ]]{\draw (.east) node[right]{$\Rightarrow$ \textit{k}}; } 
 [ThingP [N$_{1}$]]{\draw (.east) node[right]{$\Rightarrow$ \textit{a}}; }]
 [K$_{1}$P, s sep=1pt [K$_{1}$][\ldots, name=t]]{\draw (.east) node[right]{$\Rightarrow$ \textit{s}}; }]
 \draw[dashed,->,>=stealth] (t) [in=-125,out=-110,looseness=1.5]  to (tgt);
\end{forest}}

\noindent This leaves us with the pronominal `who' reading of \textit{kas} to explain. 
That is, we now need to structurally differentiate not between the categories from the \makebox[\linewidth][s]{``Comp\,$>$\,Rel\,$>$\,Wh\,$>$\,Dem\textsubscript{indef}'' sequence but between two wh-pronouns:  \textit{kas}}\\
 `what' and \textit{kas} `who'. Descriptively speaking, we need to represent the structural difference between the two vertical cells in the following two-dimensional \isi{paradigm}:

\begin{table}
\caption{Latvian}
\begin{tabular}[h]{ l l l l l l }
\lsptoprule
\textsc{comp} & \textsc{rel} 	& \textsc{wh}  	& \textsc{dem}\textsubscript{indef}\\	
\midrule
 ka		 & kas\cellcolor[gray]{0.9}			& kas	\textsubscript{what}\cellcolor[gray]{0.9} 	& tas\\
		 & 							& kas	\textsubscript{who}\cellcolor[gray]{0.9}	& 	\\
\lspbottomrule
\end{tabular}
\end{table}

\noindent With the refined pronominal stem in \ref{refined-NP}, we can represent the difference between both \isi{wh-pronoun}s as the size difference of the \textit{-a} stem, as in \ref{enough} (modulo case).\pagebreak


\ex.\label{enough} Spell-out of the \ili{Latvian} \textit{k-a-} `who'\\[0.75ex]
\begin{forest}nice empty nodes, for tree={l sep=0.65em,l=0,calign angle=63}
 [WhP, s sep=20pt 
 [WhP [Wh][MedP [Deix$_{2}$][Deix$_{1}$]]]{\draw (.east) node[right]{$\Rightarrow$ \textit{k}}; } 
 [PersonP [N$_{2}$][ThingP [N$_{1}$]]]{\draw (.east) node[right]{$\Rightarrow$ \textit{a}}; }
 ]
\end{forest}

The difference between the stem in the pronominal \textit{kas} `what' and \textit{kas} `who' reduces to the presence of the Person-forming \isi{feature} N$_{2}$ in the latter. Given the lexical entry in \ref{Lat:lex-a}, the stem comes out in both wh-pronouns as \textit{-a}.
\par If we extend this logic to the English \textit{who}, we can analyze it as a bi-morphemic \makebox[\linewidth][s]{\textit{wh-o} with \textit{-o} lexicalizing the PersonP superstructure and  \textit{-at} lexicalizing its}\\
 ThingP subset. One difference between the \ili{Latvian} \textit{-a} stem in \textit{kas} `what' and the English \textit{-at} stem in \textit{wh-at} is that the latter also contains the deictic medial (and perhaps also distal) features, as specified in \ref{lex:at} in Chapter \ref{chapter:resolving}. 


\section{Place \textit{-ur} as a pronominal superstructure in  \textit{kur}}


Let us move on to \textit{kur} `where', which unlike other case forms of \textit{kas} does not comprise the \textit{-a} stem, as shown in \tabref{tab:decl} (both the demonstrative \textit{tas} and \textit{kas} belong to the 1st declension class).
 
\begin{table}
\caption{Singular declension of \textit{tas} and \textit{kas}}
\label{tab:decl}
 \begin{tabular}[b]{lllll} % add l for every additional column or remove as necessary
  \lsptoprule
  \textsc{nom}  	& tas					& kas\\
  \textsc{acc} 	& to					& ko\\
  \textsc{gen}	& t\={a}	& k\={a}\\
  \textsc{dat}	& tam				&	kam\\
  \textsc{loc}	& taj\={a} / tai / tan\={i} & \\
  \lspbottomrule
 \end{tabular}
\end{table}

\noindent Whereas in the accusative \textit{ko} we can explain the deletion of the exponent of the  \textit{-a} stem in front of the vocalic case suffix by vowel \is{vowel truncation} truncation, \textit{kur} simply does not have a locative case suffix and hence there is no ground to describe it as a locative form of \textit{kas}.  
\par 
That \textit{kur} is a locative pronoun `where' rather than the prefix-stem complex \textit{k-a-} with an added locative case suffix is inferred from the fact that \textit{kur} is preserved in a caseless form \textit{kaut kur} `somewhere'. Moreover, the forms of \textit{kur} `where' and the locative \isi{demonstrative} \textit{tur} `there' indicate that \textit{k-} and \textit{t-} are distinct morphemes, which both can merge with the locative stem, the bound \isi{morpheme} \textit{-ur} (see e.g. \citealt{Praulins2012}).
\par
The latter fact points toward the analysis of \textit{kur} as comprising the \textit{k-} prefix and the stem \textit{-ur} denoting Place, the superset of the (pro)nominal features in \ref{refined-NP}. The lexical entry is defined as follows:

\ex. Lexical entry for the \ili{Latvian} stem \textit{-ur}\label{Lat:lex-ur}\\[0.5ex]
[\textsubscript{PlaceP} N$_{3}$ [\textsubscript{PersonP} N$_{2}$ [\textsubscript{ThingP} N$_{1}$ ]]] $\Leftrightarrow$ \textit{ur}

The description of \textit{-ur} as Place in both \textit{t-ur} `there' and \textit{k-ur} `where' is in agreement with \citeauthor{Katz-Postal1964}'s \citeyearpar{Katz-Postal1964} description of the English \textit{here}, \textit{there}, and \textit{where} as involving an underlying PP structure as in:

\ex. 
\textit{here}	$=$	at this place\\
\textit{there} $=$ at that place\\
\textit{where} $=$ at what place

\noindent Likewise, it is in agreement with \citeauthor{Kayne2007}'s \citeyearpar{Kayne2007} description of \textit{there} and \textit{where} as containing a silent noun Place, as in \ref{Kayne}.\footnote{By and large, \citeauthor{Kayne2007}'s \citeyearpar{Kayne2007} abstract Place corresponds to a silent noun proposed in \cite{Katz-Postal1964} to be present in \textit{where}, which they analyze to be a pro-form of \textit{at which place}. There is a short history of applying \citeauthor{Kayne2007}'s \citeyearpar{Kayne2007} analysis to the description of locative expressions as involving a pronominal Place in other languages (see \citealt{Pantcheva2008} for \ili{Persian}, \citealt{Leu2015} for \ili{Germanic}, \citealt{CahaPantcheva-Shona} for \ili{Shona}, \citealt{BR-T2008} for \ili{Hebrew} and \ili{Greek}, \citealt{Wiland-PSiCL} for \ili{Russian} and \ili{Polish}).
}%end of fn

\ex. \label{Kayne}
\textit{there} $=$ [ at [ that [ Place ]]]\\
\textit{where} $=$ [ at [ what [ Place ]]]
 
\noindent In what is essentially a refinement of the descriptions above, \cite{GVW-Olinco} proposes that the English \textit{there} be described as in \Next, which explains the distribution of \textit{there} with manner of motion and directed motion \isi{verb}s.

\ex. [ Dir [ Loc [ Dem [ Place ]]]]\label{dembeforeplace}

Such a refinement stems from a body of work on spatial expressions which shows that directions are more complex than locations (see \citealt{Koopman2000,Kracht2002,Zwarts2005,Cinque2010,DenDikken2010,Svenonius2010,Pantcheva2011}). In such an analysis, the syn-sem structure of a VP with a directional \isi{preposition} (e.g. \textit{to that place}) contains the structure of the locative preposition (e.g. \textit{in that place}), as outlined in the following:

\ex.\label{dir-loc} [ V [ Dir [ Loc [ Dem Place ]]]]

\noindent More specifically, Vanden Wyngaerd argues that manner of motion verbs like \textit{walk}, \textit{dance}, \textit{run} will merge with \textit{there} which is ambiguous between direction and location, as in \ref{shedanced}.\footnote{The descriptions in (\pref{dembeforeplace}--\pref{dir-loc}) include Dem, which \cite{GVW-Olinco} does not list as a separate category in the structure of \textit{there}. Dem, however, must remain a category distinct from both Dir/Loc and Place to allow for the deictic contrast between the English proximal \textit{here} and the medial/distal \textit{there}. Moreover,  the fact that the PP \textit{in that place} as in \ref{itp} below can be described as either \textit{in there}  in \ref{it} or \textit{there} in \ref{solothere} but not as a periphrastic *\textit{that there} points to an analysis of \textit{there} as realizing demonstrative \textit{that} as its ingredient.

\noindent\parbox{\linguexfootnotewidth}{\ex. 
\a. She danced \textit{in that place}.\label{itp}
\b. She danced \textit{in there}.\label{it}
\c. She danced \textit{there}.\label{solothere}
\d. *She danced (\textit{in}) \textit{that there}.

}}%end of fn


\ex. She danced \textit{there} ($=$ to that place/in that place).\label{shedanced}

In turn, directed motion \isi{verb}s like \textit{go} or \textit{come} will merge with only a locative \textit{there}, as in:

\ex.\label{loc-there}She went \textit{there} ($=$ *to that place/in that place).\label{shewent}

\noindent In \citeauthor{GVW-Olinco}'s \citeyearpar{GVW-Olinco} analysis, this contrast reflects the fact that manner of motion verbs are process verbs, a class of verbs which do not include the Dir layer in their own lexical entries. This means that in a VP headed by a manner of motion verb, the Dir layer is part of a different lexical item than the \isi{verb}. Consequently, such verbs can select either a directional PP (when the Dir layer is selected) as indicated in \ref{VPdir} or its locative subset (when the Dir layer is absent) as indicated in \ref{VPloc}.\largerpage[-1]

\ex. [ V\textsubscript{process} [ Dir \ [ Loc \ [ Dem Place ]]]]\label{VPdir}

\begin{picture}(0,1)
\put(45,1){\it dance}
\begin{rotate}{90}
\put(8,-60){
$\left\{\begin{array}
{cl}
\\ \\ \\
\end{array}\right.$
}
\end{rotate}
\end{picture}
 %%%%%%%%%%% to:
\begin{picture}(0,1)
\put(98,1){\it to}
\begin{rotate}{90}
\put(7,-105){
$\left\{\begin{array}
{cl}
\\ \\ \\ \\
\end{array}\right.$
}
\end{rotate}
\end{picture}
%%%%%%%%%%% that place:
\begin{picture}(0,1)
\put(140,1){\it that place}
\begin{rotate}{90}
\put(7,-163){
$\left\{\begin{array}
{cl}
\\ \\ \\ \\ \\ 
\end{array}\right.$
}
\end{rotate}
\end{picture}
%%%%%%%%%%% there #1:
\begin{picture}(0,1)
\put(118,-18){\it there}
\begin{rotate}{90}
\put(-11,-133){
$\left\{\begin{array}
{cl}
\\ \\ \\ \\ \\ \\ \\ \\ \\
\end{array}\right.$
}
\end{rotate}
\end{picture}

\vskip 0.8cm

\ex. [ V\textsubscript{process} \  [ Loc [ Dem Place ]]]]\label{VPloc}

\begin{picture}(0,1)
\put(45,1){\it dance}
\begin{rotate}{90}
\put(8,-60){
$\left\{\begin{array}
{cl}
\\ \\ \\
\end{array}\right.$
}
\end{rotate}
\end{picture}
 %%%%%%%%%%% in:
\begin{picture}(0,1)
\put(85,1){\it in}
\begin{rotate}{90}
\put(8,-93){
$\left\{\begin{array}
{cl}
\\ \\ 
\end{array}\right.$
}
\end{rotate}
\end{picture}
%%%%%%%%%%% that place:
\begin{picture}(0,1)
\put(115,1){\it that place}
\begin{rotate}{90}
\put(7,-138){
$\left\{\begin{array}
{cl}
\\ \\ \\ \\ \\ 
\end{array}\right.$
}
\end{rotate}
\end{picture}
%%%%%%%%%%% there #2:
\begin{picture}(0,1)
\put(105,-18){\it there}
\begin{rotate}{90}
\put(-11,-121){
$\left\{\begin{array}
{cl}
\\ \\ \\ \\ \\ \\ \\
\end{array}\right.$
}
\end{rotate}
\end{picture}

\vskip 0.9cm
\noindent Thus, the directional `to that place' reading of \textit{there} in \ref{shedanced} follows from the lexicalization of the directional superstructure, whereas the `in that place' reading of \textit{there} follows from the lexicalization of its syncretic locative subset.
In contrast, the Dir layer is always lexicalized as part of a directed motion \isi{verb} leaving only the Loc layer to be lexicalized by the PP. Hence, the \textit{there} in \ref{shewent} spells out only the locative subset of the directional superstructure, as indicated in \ref{killers}.

 
\ex.\label{killers} [ V\textsubscript{process} \ [ Dir [ Loc \ [ Dem Place ]]]]

\begin{picture}(0,1)
\put(65,1){\it go}
\begin{rotate}{90}
\put(7,-72){
$\left\{\begin{array}
{cl}
\\ \\ \\ \\ \\
\end{array}\right.$
}
\end{rotate}
\end{picture}
 %%%%%%%%%%% in:
\begin{picture}(0,1)
\put(110,1){\it in}
\begin{rotate}{90}
\put(7,-117){
$\left\{\begin{array}
{cl}
\\ \\ 
\end{array}\right.$
}
\end{rotate}
\end{picture}
%%%%%%%%%%% that place:
\begin{picture}(0,1)
\put(140,1){\it that place}
\begin{rotate}{90}
\put(7,-163){
$\left\{\begin{array}
{cl}
\\ \\ \\ \\ \\ 
\end{array}\right.$
}
\end{rotate}
\end{picture}
%%%%%%%%%%% there with go :
\begin{picture}(0,1)
\put(131,-20){\it there}
\begin{rotate}{90}
\put(-12,-146){
$\left\{\begin{array}
{cl}
\\ \\ \\ \\ \\ \\ \\ 
\end{array}\right.$
}
\end{rotate}
\end{picture}

\vskip 0.9cm
 
\noindent
We can add to these observations the fact that the locative but not the directional \textit{there} can be preceded by \textit{in} with both manner of motion and directed motion verbs, as in \ref{bornagain}.

\ex.\label{bornagain}
\a. She danced \textit{in there} ($=$ *to that place/in that place).
\b. She went \textit{in there} ($=$ *to that place/in that place).


\noindent 
This indicates that in such cases \textit{there} corresponds only to \textit{that place}, the complement of the locative PP, which is predicted by the analysis of the locative \isi{preposition} \textit{in} as a subset of the directional \textit{to}.\footnote{As pointed out by a reviewer, \textit{here/there} in expressions such as \textit{dance in here/there} is analysed in \cite{Svenonius2010} as a PP modifier that is crossed by a PP with a silent pronominal Ground, as shown in the the following:

\noindent\parbox{\linguexfootnotewidth}{\ex.
\setlength{\arrowht}{3ex}
\newcommand*\cgdepthstrut{{\vrule height 0pt depth \arrowht width 0pt}}
\renewcommand\eachwordone{\cgdepthstrut\rmfamily}
% \renewcommand\glt{\vskip -\topsep}
% \let\trans=\glt
\newcommand\arrowex{\setlength{\arrowht}{1ex}\ex}
[\textsubscript{PP} [\textsubscript{PlaceP} \tikzmark{in} \textit{pro} ][\textsubscript{PP} \{ here/there \}  \tikzmark{t}\textsubscript{PlaceP}  ]]
 \arrow{t}{in}

} \vspace*{.5\baselineskip}

\noindent This contrasts with the representation of \textit{there} here as a complement to the preposition. While it is certainly interesting to see to what extent the analysis of the \ili{Latvian} demonstratives can be informed by Svenonius's analysis, I will continue to work with a simpler representation. As long as \textit{there} is not a sister to the \isi{preposition}al Dir or Loc, however, expressions such as \textit{in there} can in principle still be analyzed as structures involving a silent pronominal Ground, as in: [\,in \textit{pro} there\,].
} %end of fn
Using the notational convention in \cite{GVW-Olinco}, the above can be summarized as in \tabref{table:guido}.

\begin{table}
\caption{Readings of \textit{there}}
\label{table:guido}
\begin{tabular}[t]{ c c c c c }
\lsptoprule	
 \multicolumn{1}{c|}{\textsc{process}}  & \multicolumn{1}{c|}{\textsc{dir}} &\multicolumn{1}{c|}{\textsc{loc}}  & \multicolumn{1}{c|}{\textsc{that}}	& \multicolumn{1}{c}{\textsc{place}}\\\midrule
 \multicolumn{1}{c|}{\textit{dance}}& \multicolumn{2}{c|}{\textit{to}}	&\multicolumn{2}{c}{\textit{there}}\\\hline
 \multicolumn{1}{c|}{\textit{dance}}& \multicolumn{4}{c}{\textit{there}}\\\hline
 \multicolumn{1}{c|}{\textit{dance}}& \multicolumn{1}{c|}{$\times$}	&\multicolumn{1}{c|}{\textit{in}}	&\multicolumn{2}{c}{\textit{there}}\\\hline
\multicolumn{1}{c|}{\textit{dance}}& \multicolumn{1}{c|}{$\times$}	&		  & \textit{there} & \multicolumn{1}{c}{} \\\hline
 \multicolumn{2}{c|}{\textit{go}}	&\multicolumn{1}{c|}{\textit{in}}	&\multicolumn{2}{c}{\textit{there}}\\\hline
 \multicolumn{2}{c|}{\textit{go}}	&		  & \textit{there} & \multicolumn{1}{c}{} \\
 \lspbottomrule
\end{tabular}
\end{table}


\par Note that since the English \textit{there} can appear as a complement to \isi{preposition}s \textit{to} and \textit{in}, we must be able to define the minimal syntactic structure \textit{there} can lexicalize without relying on Dir and Loc layers. If this logic is carried over to the \ili{Latvian} \textit{tur} we can describe it as comprising the medial prefix \textit{t-} and the pronominal base Place \textit{-ur} as the minimal subset of features it lexicalizes, as shown in \ref{so:kur} below.\footnote{``Minimal'' in the sense that if we take any feature out of the equation from what spells out as \textit{tur} in \ref{so:kur}, we are going to end up with other forms. The pronominal base that is a notch smaller than Place in \ref{so:kur} gives us the stem \textit{t-a-} of the medial \isi{demonstrative} \textit{tas} in \ref{so:Lat:tas}. In turn, stripping the pair of deictic features in \ref{so:kur} down to the sole Deix$_{1}$ allows us to construe nothing more than the stem \textit{\v{s}i}- of the proximal demonstrative \textit{\v{s}is} in \ref{so:Lat:sis}.
 }%end of fn

\ex. Minimal \isi{spell-out}s of the \ili{Latvian} \textit{tur} `there' and \textit{kur} `where'\label{so:kur}\\[0.5ex]
\begin{forest}nice empty nodes, for tree={l sep=0.65em,l=0,calign angle=63}
[WhP, s sep=30pt [WhP [Wh]
[MedP [Deix$_{2}$][Deix$_{1}$]
]{\draw (.east) node[right]{$\Rightarrow$ \textit{t}}; }
]{\draw (.east) node[right]{$\Rightarrow$ \textit{k}}; }
[PlaceP [N$_{3}$][PersonP [N$_{2}$][ThingP [N$_{1}$]]
]{\draw (.east) node[right]{$\Rightarrow$ \textit{a}}; }
]{\draw (.east) node[right]{$\Rightarrow$ \textit{ur}}; }
]
 \end{forest}

\noindent 
In such a representation, the difference between the stems \textit{tas}, \textit{kas}\textsubscript{what} and the locative \textit{tur}, \textit{kur} is in the size of the (pro)nominal base, Thing vs. Place, in line with the \isi{containment} hierarchy in \ref{refined-NP}, rather than in the locative case suffix.   In turn, the contrast between the forms for the locative `there' and `where', which is realized by a prefix, is by no means specific to \ili{Latvian} or English as essentially the same pattern holds for example in \ili{Czech}, where these forms are, respectively, \textit{t-am} and \textit{k-am}.\footnote{See also \cite{Greenberg2000} and the references cited there for a lists of \ili{Indo-European} forms comprising the \textit{-r} stem, a likely source of present day \ili{Latvian} locative stem \textit{-ur}, in adverbs and certain verbal compounds. In particular, \citet[147]{Greenberg2000} also cites \citet[1087]{Pokorny1959}, who reconstructs forms parallel to the \ili{Indo-European} locative \textit{-r} based on the \isi{demonstrative} \textit{t-} as *\textit{tor} or *\textit{t\={e}r} as `there' including the \ili{Latvian} \textit{tur}. 
} %end of fn
\par
To wrap up the discussion of the locative \textit{kur}, this form is best described as belonging to the vertical (inter-categorial) set of the Wh forms in the two-dimensional \isi{paradigm} in \ref{kury} rather than to the case declension paradigm of \textit{kas}\textsubscript{what} given in \tabref{tab:decl}.

\begin{table}
\caption{Locative \textit{kur} in a two-dimensional paradigm}
\label{kury}
\begin{tabular}[t]{ l l l l l l }
\lsptoprule
\textsc{comp} & \textsc{rel} 	& \textsc{wh}  	& \textsc{dem}\textsubscript{indef}\\	
\midrule
 ka		 & kas\cellcolor[gray]{0.9}			& kas	\textsubscript{what}\cellcolor[gray]{0.9} 	& tas\\
		 & 							& kas	\textsubscript{who}\cellcolor[gray]{0.9}	& 	\\
		 &							& kur\textsubscript{where}				&	\\
\lspbottomrule
\end{tabular}
\end{table}

\noindent
Before we move on to the \ili{Latvian} caseless \isi{complementizer} \textit{ka}, let us juxtapose English \textit{there}, \textit{where} against \ili{Latvian} \textit{tur}, \textit{kur}.
\par An essential difference between these categories is that the English \textit{there} includes the \textit{th-}prefix, which is syncretic not only with the Rel and Comp but also with the Def-marker, which \ili{Latvian} lacks. In the previous chapter, we reduced the differences between syncretic alignment of Wh, Rel, and Comp with definite and indefinite and demonstratives to the ``Dem\textsubscript{def}\,$>$\,Comp\,$>$\,Rel\,$>$\,Wh\,$>$\,Dem\textsubscript{indef}'' containment sequence, which is closed by Def, the top-most category in the \isi{fseq}. This allowed us to describe the structure realized by the English \textit{wh-} as a subset of the structure realized by \textit{th-} (see \ref{Eng:LB} in \sectref{sec:hdd}). This result is seamlessly  retained for \textit{th-ere} and \textit{wh-ere} if the entries for \textit{th-} and \textit{wh-} are refined by a decomposed spatial deixis and the entry for \textit{-ere} is defined as Place, as specified in \ref{Reek}.  

\ex.\label{Reek} Lexical entries in English (3rd and final approximations for \textit{th} and \textit{wh}, which supersede the ones in \ref{th+wh:2nd} in \sectref{sec:hdd})
\a. [ Def [ Comp [ Rel [ Wh [ Deix$_{2}$ Deix$_{1}$ ]]]]] $\Leftrightarrow$ \textit{th}
\b. [ Wh [ Deix$_{2}$ Deix$_{1}$ ]]  $\Leftrightarrow$ \textit{wh}
\c. [\textsubscript{PlaceP} N$_{3}$ [\textsubscript{PersonP} N$_{2}$ [\textsubscript{ThingP} N$_{1}$ ]]] $\Leftrightarrow$ \textit{ere}

\noindent These items realize a syntactic representation in which the spell-out of (at least) the Med-forming \isi{feature} Deix$_{2}$ is unachievable by \textsc{stay} or \textsc{move} and its lexicalization takes place in the left branch, as shown in \ref{roose}.


\ex.\label{roose} Minimal spell-outs of \textit{there} and \textit{where}\label{so:there+where}\\[0.5ex]
{\small \begin{forest}nice empty nodes, for tree={l sep=0.65em,l=0,calign angle=63}
[Dem\textsubscript{def}, s sep=20pt [Dem\textsubscript{def}[Def][CompP [Comp][RelP [Rel][WhP  [Wh]
[MedP, s sep=6pt [Deix$_{2}$][Deix$_{1}$]
]
]{\draw (.east) node[right]{$\Rightarrow$ \textit{wh}}; }]]
]{\draw (.east) node[right]{$\Rightarrow$ \textit{th}}; }
[PlaceP [N$_{3}$][PersonP [N$_{2}$]
[ThingP [N$_{1}$]]
]
]{\draw (.east) node[right]{$\Rightarrow$ \textit{ere}}; }
]
\end{forest}}

\noindent As indicated in \ref{roose}, subsequent mergers of Wh, Rel, Comp, and Def on top of WhP will extend the subderivation (the left branch) in a familiar way.\footnote{Let us note that the \isi{spell-out} of Place as \textit{-ere} in \ref{so:there+where} does not appear to trivially over-ride \is{Cyclic Over-ride} the lexical entry for \textit{-at}. This follows from the fact that only the second includes the overt marking of the deictic contrast, as in \textit{th-is} vs. \textit{th-at}, which indicates that the lexical entry for \textit{-at} includes Dem\textsubscript{indef} rather than a bare pronominal base Thing, as specified in \ref{lex:at} in \sectref{sec:hdd}. We do not find overt evidence for the deictic contrast between \textit{th-ere} vs. \textit{h-ere} to be lexicalized in \textit{-ere}, unless the proximal \textit{here} /hir/ is analyzed as an allomorph \is{allomorphy} of a bound morpheme \textit{-ere} /er/.  
}%end of fn on -ere
\par
 With the lexical entries covering \textit{tas}, \textit{kas}, and \textit{kur}, we are in a position to discuss the \ili{Latvian} complementizer \textit{ka}. 


\section{Caseless complementizer \textit{ka}}

On the one hand, we have seen in \sectref{section:kas} that suffixal case marking on demonstratives \textit{\v{s}is} and \textit{tas} as well as \textit{kas} `what'/`who'/Rel follows straightforwardly if case is projected on top of the categories of the ``Dem\textsubscript{def}\,$>$\,Comp\,$>$\,Rel\,$>$\,Wh\,$>$ Dem\textsubscript{indef}'' sequence rather than directly on top of the categories of the (pro)no\-mi\-nal base PersonP\,$>$\,ThingP.
On the other hand, setting up the \isi{paradigm} like in \tabref{Lat:solution} allows us to assemble the categories with the case suffix into an adjacent span of cells. This leads to the observation that the projection of the case is delimited by the Rel layer. 
\par
There is independent evidence that the case \isi{fseq} is ordered on top of the Rel\,$>$\,Wh\,$>$\,Dem\textsubscript{indef} sequence in Lavian as part of a more general pattern. 
If we recall the representation of the \ili{Polish} bi-morphemic Dem\textsubscript{indef} \textit{t-o}, Rel$=$Wh \textit{c-o}, and Comp \textit{\.z-e} in (\pref{sp:K1}--\pref{sp:Pol2}) in Chapter \ref{chapter:resolving}, whose prefixless structure indicates that the ``Comp\,$>$\,Rel\,$>$\,\ldots \,'' sequence is all lexicalized by the most basic spell-out option \textsc{stay}, we observe that case is projected on top of all its categories. This is the only possible location of the case markers to come out as suffixes. 
We can, thus, conclude that case is projected on top of the categories that comprise the ``Comp\,$>$\,Rel\,$>$\,\ldots \,'' sequence irrespective of the geometry of the tree, whose segregation into multiple subtrees is solely a matter of the \isi{spell-out} mechanism. 
\par
An exception to the first part of this statement is the \ili{Latvian} Comp \textit{ka} once we break it down into a complex \textit{k-a}. Such an analysis comes naturally as it keeps the lexical entry for the stem \textit{-a} in \ref{Lat:lex-a} intact and it only requires us to update the entry for \textit{k}- with the Comp \isi{feature} on top, as in the following.

\ex. Lexical entry for the \ili{Latvian} prefix \textit{k-} (2nd and final approximation, replaces \pref{old:k})\label{Lat:lex:k}\\[0.5ex]
[ Comp [ Rel [ Wh [\textsubscript{MedP} Deix$_{2}$ Deix$_{1}$ ]]]] $\Leftrightarrow$ \textit{k}

With a complex \textit{k-a}, we arrive at a picture where the subset structures of the cross-categorial sequence comprising Dem\textsubscript{indef} (\textit{tas} in \pref{so:Lat:tas}), Wh and Rel (\textit{kas} in \pref{so:kas}) are all extended by the case features while the Comp superset structure in \ref{ka:so} is not.\largerpage

\ex.\label{ka:so}\ili{Latvian} \isi{complementizer} \textit{ka}\\[0.5ex]
\begin{forest}nice empty nodes, for tree={l sep=0.65em,l=0,calign angle=63}
 [CompP, s sep=5pt
 [CompP [Comp]
 [RelP [Rel]
 [WhP [Wh]
 [MedP [Deix$_{2}$][Deix$_{1}$]
 ]{\draw (.east) node[right]{$\Rightarrow$ \textit{t}}; }]
 ]]{\draw (.east) node[right]{$\Rightarrow$ \textit{k}}; }
 [ThingP [N$_{1}$]]{\draw (.east) node[right]{$\Rightarrow$ \textit{a}}; }]
\end{forest}

\vskip 0.2cm

\noindent 
Technically speaking, the \ili{Latvian} RelP delimits the projection of the case \isi{fseq} but this result leads to a new more arduous question: why?
\par
A possible answer can be informed by the contrasts with the \ili{Polish} invariant Rel \textit{co} and the \ili{Russian} invariant Rel/Comp \textit{\v{c}to}, whose suffix \textit{-o} is the exponent of the neuter nominative (see \tabref{to:decl} in Chapter \ref{chapter:resolving}). If the status of the \ili{Slavic} neuter \textit{-o} suffix teaches us about default case morphology (in the sense that it need not show concord), then the lack of neuter gender in \ili{Latvian} results in a caseless invariant \textit{ka}. In this way Comp \textit{ka} contrasts with Dem \textit{tas} and Wh/Rel \textit{kas}  with respect to case concord with masculine and feminine nouns, as shown for instance in \ref{conc1} or \ref{conc2}.

\section{Multi-dimensional morphological paradigms as homeomorphic singleton projection lines in syntax}\label{sec:multidimensional}
 \rohead{Multi-dimensional paradigms}


One final remark about the ordering of the case fseq with respect to the categories of the ``Dem\textsubscript{def}\,$>$\,Comp\,$>$\,Rel\,$>$\,Wh\,$>$\,Dem\textsubscript{indef}'' sequence is in place at this point.\is{paradigm}
\par
On the one hand we have seen an argument from syncretic alignment and morphological \isi{containment} for a strict ordering between the categories as seen in the tree in \ref{updated-fseq} in Chapter \ref{chapter:resolving}. On the other hand, in principle every category in this sequence can project case on its top: Dem\textsubscript{def} in \ili{German}; Dem\textsubscript{indef}, Wh, Rel, and Comp in \ili{Polish} and \ili{Russian}. Though, the invariant categories like the Polish Rel \textit{co} only project a default neuter nominative. In \ili{Latvian}, Dem\textsubscript{Indef}, Wh, and Rel all project the case \isi{fseq} on their top except for the Comp. In this respect, the combination of case marking and the categories of the \mbox{``Comp\,$>$\,Rel\,$>$\,\ldots \,''} sequence results in the formation of two-dimensional paradigms, as shown on the example of \ili{Polish} and \ili{Latvian} declensions in \tabref{2D:Pol} and in \tabref{2D:Lat}.

\begin{table}
\caption{Neuter case declension of the categories syncretic with the declarative complementizer in \ili{Polish}}
\label{2D:Pol}
\begin{tabular}[h]{ l l l l l l }
 \lsptoprule
		& 			& \textsc{comp} & \textsc{rel} 	& \textsc{wh}  	& \textsc{dem}\textsubscript{indef}\\
\midrule	
 	& \textsc{stem}	& \.z-		 & c-\cellcolor[gray]{0.95}		& c-\cellcolor[gray]{0.95}	& t-\\
		& \textsc{nom}	&  \.z-e		 & c-o\cellcolor[gray]{0.85}	& c-o\cellcolor[gray]{0.85}	& t-o\\
 		& \textsc{acc}	&			 & 			& c-o\cellcolor[gray]{0.85}		& t-o\\
		& \textsc{gen}	&			 & 			& cz-ego		& t-ego\\	
		& \textsc{dat}	&			 & 			& cz-emu		& t-emu\\
		& \textsc{loc}	&			 & 			& cz-ym\cellcolor[gray]{0.75}	& t-ym\\
		& \textsc{inst}	&			 &			& cz-ym\cellcolor[gray]{0.75}	& t-ym\\
\lspbottomrule
\end{tabular}
\end{table}

\begin{table}
\caption{Masculine case declension of the categories syncretic with the declarative complementizer in \ili{Latvian}}
\label{2D:Lat}
\begin{tabular}[h]{ l l l l l l }
 \lsptoprule
		& 			& \textsc{comp} & \textsc{rel} 	& \textsc{wh}  	& \textsc{dem}\textsubscript{indef}\\
\midrule	
	& \textsc{stem}	& ka\cellcolor[gray]{0.95}		& ka-\cellcolor[gray]{0.95}	& ka-\cellcolor[gray]{0.95}	& ta-\\
		& \textsc{nom}	& 			 & ka-s\cellcolor[gray]{0.90}	& ka-s\cellcolor[gray]{0.90}	& ta-s\\
 		& \textsc{acc}	&			 & k-o\cellcolor[gray]{0.80}	& k-o\cellcolor[gray]{0.80}		& t-o\\
		& \textsc{gen}	&	& k-\={a}\cellcolor[gray]{0.75}& k-\={a}\cellcolor[gray]{0.75} 
																	& t-\={a}\\	
		& \textsc{dat}	&			 & k-am\cellcolor[gray]{0.7}	& k-am\cellcolor[gray]{0.7}	& t-am\\
		& \textsc{loc}	&			 & 						& 		& ta-j\={a}\\
\lspbottomrule
\end{tabular}
\end{table}

\noindent
This begs the following question: how are the horizontal Dem, Wh, Rel, and Comp \isi{feature}s ordered with respect to case-forming vertical K\textsubscript{n}  features so that their mergers create two-dimensional \isi{paradigm}s? In the approach to a syntactic representation of multi-morphemic forms advanced here both horizontal and vertical cells must result form a monotonically growing singleton projection line in syntax. This result can be achieved if the features forming the same \isi{fseq} are ordered both with respect to each other, as in \ref{bothsequences}, and with respect to the features in the other fseq, as in \ref{combine}.

\ex.\label{bothsequences} 
\a.\label{ver} \ldots \ $>$\,K$_{3}$P\,$>$\,K$_{2}$P\,$>$\,K$_{1}$P \hfill (case fseq) 
\b.\label{hor} Comp\,$>$\,Rel\,$>$\,Wh\,$>$\,Dem\textsubscript{indef}  \hfill (\isi{complementizer} fseq)

\ex.\label{combine} \ldots \ $>$\,K$_{3}$P\,$>$\,K$_{2}$P\,$>$\,K$_{1}$P\,$>$\,Comp\,$>$\,Rel\,$>$\,Wh\,$>$\,Dem\textsubscript{indef}

 The familiar sequences in \ref{ver} and \ref{hor} form the vertical and the horizontal paradigm; their combination in \ref{combine} incorporates both \isi{paradigm}s into one complex morphological system.
 \par
 All we need to do to derive case-marked forms of Dem, Wh, Rel, and Comp (if applicable) is to accommodate the basic premise that the fseqs in \ref{ver} and \ref{hor} can appear as subsets.\footnote{Different classes of ordered features (fseqs) that form a singleton projection line are informally referred to as ``\isi{fseq} zones'' in \citet{LTN,NU}. \is{fseq zone}
} %end of fn
For instance, the Dem\textsubscript{indef} subset of \ref{hor} can be directly extended by K$_{1}$, K$_{2}$, etc. when \isi{feature}s forming Wh, Rel, Comp are not selected as in the formation of case-inflected demonstratives. However, when these features are selected, they must be strictly ordered with respect to the other features within  the same fseq (on top of Dem\textsubscript{indef}) and with respect to the case fseq (below K$_{1}$).
\par
Let us point out that the fact that \textit{co} in \tabref{2D:Pol} forms a syncretic triplet targeting adjacent horizontal and vertical cells is expected in two-dimensional paradigms (see \citealt{TaraldsenNELS} and \citealt{CahaPantcheva2012}). The paradigms covered in \cite{TaraldsenNELS} and \cite{CahaPantcheva2012} include morphologically simplex forms while the ones discussed here include multi-morphemic forms. More specifically, \cite{TaraldsenNELS} discusses abstract exponents organized into feature sets and \cite{CahaPantcheva2012} discuss \isi{syncretism}s between mono-morphemic dative, allative, and locative markers. However, if the hypothesis advanced here that paradigms can be described as a singleton \isi{fseq} is on the right track, then there is no reason to differentiate between two-dimensional \isi{paradigm}s on the basis of the number of \isi{morpheme}s they involve since multimorphemic forms are solely a result of the segregation of a single  projection line in the syntactic representation into multiple subtrees at \isi{spell-out}. 
While such a system allows for the accommodation of case features with different stems, we are not able to rule out (partial or complete) caselessness of certain forms in the paradigm (e.g. the \ili{Polish} invariant Rel \textit{co}), an explanation for which must come from elsewhere, as suggested for the \ili{Latvian} caseless \textit{ka} above.
\par
The representation of two-dimensional paradigms as a
sequence of syntactic heads, a de facto one-dimensional space, leads to the conjecture that any \textit{n}-dimensional \isi{paradigm} can be represented as a homeomorphic \isi{fseq}. This conjecture can be illustrated for a three-dimensional paradigm that includes the three \ili{Latvian} \isi{wh-pronoun}s, the syncretic \textit{kas} `what'/`who' and \textit{kur} `where', that form a backward coordinate (the aisle) in the paradigm in \ref{3D}. 

\ex.\label{3D}

\begin{center}\vspace{-15pt}
\begin{tikzpicture}[every node/.style={anchor=base west,fill=white,minimum width=0.1cm,minimum height=2mm}]
\matrix (mA) [draw,matrix of math nodes,matrix anchor=north east]
{
& \textsc{{\color{white}comp}} & \textsc{{\color{white}rel}} &[2mm] \textsc{\textbf{where}} & \textsc{{\color{white}dem}} \\\hline
\textsc{{\color{white}stem}}	& {\color{white}0} 	& {\color{white}0} 	& \textup{kur} 	& {\color{white}0}\\\hline
\textsc{nom} 	&  			& \textup{kas} &  & {\color{white}0}\\\hline
\textsc{acc} 	& 			& \textup{ko} 	&   & {\color{white}0}\\\hline
\textsc{gen} 	&  			& \textup{k} &  & {\color{white}0} \\\hline
\textsc{dat} 	&  			& \textup{kam} &  & {\color{white}0} \\\hline
\textsc{loc}	&  			&  &  & {\color{white}0} \\
};

\matrix (mB) [draw,matrix of math nodes,matrix anchor=north east] at ($(mA.south west)+(3.25,3.10)$)
{
& \textsc{{\color{white}comp}} & \textsc{{\color{white}rel}} & [2mm]\textsc{\textbf{who}} & \textsc{{\color{white}dem0}} \\\hline
			&  	&  	& \textup{\hl{ka-}} 	& {\color{white}0}\\\hline
\textsc{nom} 	&  			& \textup{kas} & \textup{kas} & {\color{white}0}\\\hline
\textsc{acc} 	& 			& \textup{ko} 	& \textup{ko}  & {\color{white}0}\\\hline
\textsc{gen} 	&  			& \textup{k}\textup{\={a}} & \textup{k}\textup{\={a}} & {\color{white}0} \\\hline
\textsc{dat} 	&  			& \textup{kam} & \textup{kam} & {\color{white}0} \\\hline
\textsc{loc}	&  			&  &  & {\color{white}00} \\
};

\matrix (mC) [draw,matrix of math nodes,matrix anchor=north east] at ($(mB.south west)+(3.15,3.25)$)
{
& \textsc{comp} & \textsc{rel} 		& \textsc{\textbf{what}} 				& \textsc{dem} \\\hline
\textsc{stem}	& \textup{\hl{ka}} 	& \textup{\hl{ka-}} 	& \textup{\hl{ka-}} 	& \textup{ta-}\\\hline
\textsc{nom} 	&  				& \textup{kas} 		& \textup{kas} 		& \textup{tas}\\\hline
\textsc{acc} 	& 				& \textup{ko} 		& \textup{ko}  		& \textup{to}\\\hline
\textsc{gen} 	&  	& \textup{k}\textup{\={a}} & \textup{k}\textup{\={a}} & \textup{t}\textup{\={a}} \\\hline
\textsc{dat} 	&  				& \textup{kam} 		& \textup{kam} & \textup{tam} \\\hline
\textsc{loc}	&  				&  			&  & \textup{taj}\textup{\={a}} \\
};

\draw[dashed](mA.north east)--(mC.north east);
\draw[dashed](mA.north west)--(mC.north west);
\draw[dashed](mA.south east)--(mC.south east);
\end{tikzpicture}
\end{center}

\noindent Only one of these wh-pronouns, \textit{kas} `what', is a cell in the cross-categorial paradigm (the horizontal coordinate) and both `what' and `who' are inflected for case (the vertical coordinate). The values of the vertical coordinate in \ref{3D} are described by the case fseq in \ref{ver}, the values of the horizontal coordinate by \ref{hor}, and the values of the backward aisle by a decomposition of the (pro)nominal base in \ref{refined-NP}, the subset of the \isi{wh-pronoun}s, repeated below.

\ex. Place\,$>$\,Person\,$>$\,Thing

The ordering of the refined (pro)nominal base with respect to the other fseqs gives us the updated singleton sequence, as in the following:

\ex. \ldots \ $>$\,K$_{3}$P\,$>$\,K$_{2}$P\,$>$\,K$_{1}$P\,$>$\,Comp\,$>$\,Rel\,$>$\,Wh\,$>$\,Dem\textsubscript{indef}\,$>$\,Place\,$>$\\
Person $>$\,Thing

\noindent
If the \isi{*ABA} generalization follows from the \isi{Superset Principle} that applies to an ordered \isi{fseq}, then we correctly expect syncretism to be restricted to adjacent cells in \textit{n}-dimensional \isi{paradigm}s, a result described independently for two-dimensional paradigms earlier in \cite{CahaPantcheva2012} and \cite{GVW2018}. In \ref{3D} we observe the syncretic span restricted to adjacent cells of the horizontal and the backward coordinates that includes the `what'-cell at their juncture.
\par With the decomposition of Dem\textsubscript{indef} into `Dist\,$>$\,Med\,$>$\,Prox', we are able to further refine the singleton sequence of projections as in:

\ex. \ldots \ $>$\,K$_{3}$P\,$>$\,K$_{2}$P\,$>$\,K$_{1}$P\,$>$\,Comp\,$>$\,Rel\,$>$\,Wh\,$>$\,Dist\,$>$\,Med\,$>$\,Prox\,$>$\\ 
Place\,$>$\,Person\,$>$\,Thing

With this refinement in place, the distinction between the \ili{Latvian} proximal \textit{\v{s}is} and the medial/distal \textit{tas} belongs to the third coordinate in the paradigm (the forward aisle), as in \ref{3D:dem}.
\par
 The representation of the Prox \textit{\v{s}is} as a cell forming the third coordinate reflects the fact that both Prox \textit{\v{s}is} and Med/Dist \textit{tas} are case inflected but only the latter is a cell in the cross-categorial paradigm with Comp, Rel, and Wh. 
Such an ordering also captures the observation we can make on the basis of the data discussed so far, namely that proximal \isi{demonstrative}s by and large do not belong to the ``Dem\textsubscript{def}\,$>$\,Comp\,$>$\,Rel\,$>$\,Wh\,$>$\,Dem\textsubscript{indef}'' sequence, the statement which appears to hold both for languages with ``high'' Dem\textsubscript{def}  (e.g. English \textit{that - what} or  Spanish \textit{aqu\'el - qu\'e}) and the ``low'' Dem\textsubscript{indef} (e.g. \ili{Russian} \textit{to - \v{c}to} or \ili{Polish} \textit{to - co}). Though, more typological work is required before this can be turned into a generalization.\pagebreak

\ex.\label{3D:dem}

\begin{center}\vspace{-20pt}
\begin{tikzpicture}[every node/.style={anchor=base west,fill=white,minimum width=0.1cm,minimum height=2mm}]
\matrix (mA) [draw,matrix of math nodes,matrix anchor=north east]
{
\textsc{{\color{white}ooo}} & \textsc{{\color{white}ooo}} & \textsc{{\color{white}ooo}} &[6mm] \textsc{where} & \textsc{{\color{white}0}} \\\hline
{\color{white}ooo}	& {\color{white}ooo} 	& {\color{white}0} 	& \textup{kur} 	& {\color{white}0}\\\hline
 	&  			&  &  & {\color{white}0}\\\hline
 	& 			&  	&   & {\color{white}0}\\\hline
	&  			&  &  & {\color{white}0} \\\hline
 	&  			&  &  & {\color{white}0} \\\hline
	&  			&  &  & {\color{white}0} \\
};

\matrix (mB) [draw,matrix of math nodes,matrix anchor=north east] at ($(mA.south west)+(3.4,3.0)$)
{
& \textsc{{\color{white}comp}} & \textsc{{\color{white}rel}} &[4mm] \textsc{who} & \textsc{{\color{white}0}} \\\hline
{\color{white}ooo}	&  	&  	& \textup{\hl{ka-}} 	& {\color{white}0}\\\hline
\textsc{nom} 	&  			& \textup{kas} 	& \textup{kas} 		& {\color{white}0}\\\hline
\textsc{acc} 	& 			& \textup{ko} 	& \textup{ko}  		& {\color{white}0}\\\hline
\textsc{gen} 	&  			& \textup{k}\textup{\={a}} 	& \textup{k}\textup{\={a}} & {\color{white}0} \\\hline
\textsc{dat} 	&  			& \textup{kam} & \textup{kam} 		& {\color{white}0} \\\hline
\textsc{loc}	&  			&  &  & {\color{white}00} \\
};

\matrix (mC) [draw,matrix of math nodes,matrix anchor=north east] at ($(mB.south west)+(3.25,3.2)$)
{
& \textsc{comp} & \textsc{rel} 			&[4mm] \textsc{what} 	& \textsc{\textbf{med}} \\\hline
	& \textup{\hl{ka}} 	& \textup{\hl{ka-}} 	& \textup{\hl{ka-}} 	& \textup{ta-}\\\hline
\textsc{n} 	&  					& \textup{kas} 		& \textup{kas} 		& \textup{tas}\\\hline
\textsc{a} 	& 					& \textup{ko} 		& \textup{ko}  		& \textup{to}\\\hline
\textsc{g} 	&  & \textup{k}\textup{\={a}} & \textup{k}\textup{\={a}} & \textup{t}\textup{\={a}}\\\hline
\textsc{d} 	&  					& \textup{kam} 		& \textup{kam} 		& \textup{tam} \\\hline
\textsc{i}	&  			&  		&  				& \textup{taj}\textup{\={a}} \\
};

\matrix (dem) [draw,matrix of math nodes,matrix anchor=north east] at ($(mC.south west)+(3,2.85)$)
{
& {\color{white}ooooo} & {\color{white}oooo} & {\color{white}oooo} 				& \textsc{\textbf{prox}} \\\hline
\textsc{stem}	& {\color{white}oooo} & {\color{white}oooo} & {\color{white}oooo} 	& \textup{\v{s}i-}\\\hline
\textsc{nom} 	&  			&   		&   								& \textup{\v{s}is}\\\hline
\textsc{acc} 	& 			&  		&   								& \textup{\v{s}o}\\\hline
\textsc{gen} 	&  			& 		&  					& \textup{\v{s}}\textup{\={a}}\\\hline
\textsc{dat} 	&  			&  		&  								& \textup{\v{s}im} \\\hline
\textsc{loc}	&  			&  		&  					& \textup{\v{s}aj}\textup{\={a}} \\
};


\draw[dashed](mA.north east)--(dem.north east);
\draw[dashed](mA.north west)--(dem.north west);
\draw[dashed](mA.south east)--(dem.south east);
\end{tikzpicture}
\end{center}


 
\section{Summary}

The inclusion of the indefinite \isi{demonstrative} pronoun as the bottom category in an \isi{fseq} which covers \isi{syncretism}s with the declarative \isi{complementizer} allowed us to explain morphological \isi{containment} and syncretic alignment in such a \isi{paradigm} in \ili{Slavic}. The same holds true for \ili{Latvian}, too, which enabled us to describe the paradigm with the Comp \textit{ka}, the only suffixless item in  the fseq in \Next,  as a caseless category in a sequence where case marking is delimited by Rel. 

\ex. Comp\,$>$\,Rel\,$>$\,Wh\,$>$\,Dem\textsubscript{indef}

\noindent
Such a result follows naturally from the representation of these morphologically complex categories as a singleton sequence of syntactic projections, whose segregation into more complex subtrees is exclusively an effect of the \isi{spell-out} procedure, not of the complexity of an underlying syntactic representation.
One consequence of that approach is a possibility to describe multi-dimensional \isi{paradigm}s as a single homeomorphic sequence of syntactic projections, a conjecture shown to hold for a three-dimensional \isi{paradigm} in \ili{Latvian}.



