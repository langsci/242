\chapter{Introduction}


The aim of this book is two-fold. The first goal is to explain a curious instance of analytic vs. fusional realization of grammatical categories that we find in a  \is{iterative alternation} semelfactive-iterative alternation in \ili{Czech} and \ili{Polish} verbs. Namely, a  \isi{semelfactive}  \isi{verb} stem as in the \ili{Czech} \textit{kop-n-ou-t} `give a kick' alternates with an \isi{iterative} verb stem as in \textit{kop-a-t} `kick repeatedly', which is a regular alternation between these two categories in both languages. The iterative \textit{-aj} stem is morphologically less complex than the semelfactive stem formed with the \textit{-n-ou} sequence, which is paradoxical given an analysis of iteratives as categories whose syn-sem representation is more complex than semelfactives. 
\par
The second goal is empirically unrelated to the verb stem alternation and, instead, focuses on categories related to the declarative complementizer, \is{complementizer} such as \isi{demonstrative}, interrogative,\is{wh-pronoun} and \is{relativizer} relative pronouns.   Namely, the aim in this domain is to sort out those patterns in morphological paradigms with the complementizer which are in certain ways unexpected.  The problems in such \isi{paradigm}s include an unexpected morphological containment (in \ili{Russian}), a degree of morphological complexity (in \ili{Latvian}), and a so-called  ABA pattern of syncretic alignment (in \ili{Basa\'a}), which we do not expect to find if syncretism is restricted to adjacent cells in a paradigm (cf. \citealt{Bobaljik2012}).\is{syncretism} \is{*ABA}
\par
The reason why morphological alternations inside the \ili{Czech} and \ili{Polish} verbs and morphological \isi{containment} in the domain of \ili{Russian} and some other complementizers are addressed in one book is that, I argue, both kinds of problems boil down to the way syntactic (hierarchical) representations become lexicalized (realized as linear representations).  More specifically, the approach to lexicalization taken up in this work is informed by research on syntactic representations in the last quarter of a century, which shows that syntactic structures are maximally fine-grained, the result that is sometimes described as ``one grammatical feature per one syntactic head''. This result has led to a situation where syntactic representations are in principle submorphemic, in the sense that a lexical item, as for instance represented by $\alpha$ in \Next, corresponds to more than one syntactic head in a phrase marker, a strand of research that has become known as Nanosyntax (\citealt{Starke2009}, among others).

\ex.\label{1} 
\begin{forest}nice empty nodes, for tree={l sep=0.7em,l=0,calign angle=63}
 [F$_{3}$P [F$_{3}$][F$_{2}$P [F$_{2}$][F$_{1}$P [F$_{1}$]]]] {\draw (.east) node[right]{$\Rightarrow$ $\alpha$}; }
\end{forest} 


\noindent 
A scenario whereby a set of \isi{terminal node}s in syntax can be realized by a single lexical item has led both to the change in the way we should think about syntax and lexicon and to the change in the methodology of explaining morpho-syntactic problems.
The relation between syntax and lexical items (words and morphemes) comes out as a relation between a fine-grained mental representation of grammatical \isi{feature}s (illustrated in \ref{1} as an ordered sequence of F\textsubscript{n}) and their linguistic exponents ($\alpha$ in \pref{1}). This architecture immediately excludes the existence of any kind of a pre-syntactic lexicon, not even the one which stores abstract morphemes, as these are created only in the process of realizing grammatical features (cf. \citealt[1]{Starke2009}). 
\par
This set-up requires a spell-out formula which applies to phrasal rather than to \isi{terminal node}s, a procedure recently detailed in \cite{Starke2018}. This work investigates the limits of such a procedure in resolving the selected empirical problems in the domain of \ili{Slavic} verbs and declarative complementizers. The overarching goal of the book is, thus, modest in the sense that it argues that we can get a better understanding of these empirical problems if we consider them from the perspective of the way the spell-out mechanism applies to the sequences of syntactic heads that make up the investigated grammatical categories. \is{spell-out algorithm} \is{fseq} 
One novelty that this book brings to the table, however, is the addition of \isi{subextraction} to the list of spell-out driven operation. The list of operations that has been argued in the literature to facilitate spell-out already includes successive cyclic movement and complement movement so extending this list by the third type of phrasal movement comes out as a legitimate step to consider. 
\par
The logical organization of the book is as follows. First, in Chapter \ref{chapter:nanosyntax}, I provide an overview of the \isi{spell-out} mechanism in \isi{Nanosyntax} with a particular attention to the operations that allow us to predict if realizing a syntactic subtree as a \isi{morpheme} is going to come out as a suffix or a ``pre-'' element, that is a prefix, a \isi{preposition}, a \isi{particle}, etc. In Chapter \ref{chapter:explaining}, I move on to discussing the alternation between \isi{semelfactive} and \isi{iterative} \isi{verb}s in \ili{Czech} and \ili{Polish}, which appears to result in the \isi{reduction} in the number of \isi{morpheme}s. I explore the possibility to derive such a \isi{reduction} with extending the list of \isi{spell-out} driven operations with \isi{subextraction} and I point out limitations of such an analysis and discuss a possible alternative. \is{subextraction} Subextraction as a \isi{spell-out} driven movement, however, is considered  only in the domain of \ili{Slavic} verbs \is{verb} and is not further explored in the domain of the declarative \isi{complementizer} and related grammatical categories in \ili{Russian} (in Chapter \ref{chapter:resolving}), in what is logically the second part of the book. The discussion of this domain is followed by a comparative look at the similar problem with these categories in \ili{Latvian} (\ili{Baltic}) in Chapter \ref{chapter:latvian} and in \ili{Basa\'a} (\ili{Bantu}) in Chapter \ref{chapter:basaa}. The book ends with a summary and a list of loose ends that can be hopefully worked out in the future work.




















