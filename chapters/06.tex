\chapter{Resolving a morphological containment problem}\label{chapter:resolving}

\section{Introduction}

Let us move on to a different kind of problem that, I will argue, can be resolved with the application of the spell-out procedure to a singleton projection line of syntactic heads. Namely, the problem discussed in this chapter involves a situation in which the organization of a \isi{paradigm} based on syncretic alignment does not seem to make the right prediction about morphological \isi{containment}. 
\par
A domain where such a situation can be observed is a cross-categorial paradigm comprising the declarative \isi{complementizer} (Comp for short), the \isi{demonstrative} pronoun (Dem), the \isi{relativizer} (Rel), and the \isi{wh-pronoun} `what' (Wh). Syncretisms between these categories have led \citeauthor{BaunazLander2017} (\citeyear{BaunazLander2017,BaunazLander2018}) to advance a thesis that they form a complexity scale as in the following: \is{syncretism}

\ex.\label{BL:fseq} 
Dem\,$>$\,Comp\,$>$\,Rel\,$>$\,Wh

This inclusion sequence is based on the presumption that syncretism anchors structural containment since it holds only between adjacent layers of a syntactic structure, i.e. the \isi{*ABA} generalization. Syncretisms between these four categories that are consistent upon the sequence in \ref{BL:fseq} are well illustrated by languages such as \ili{English}, \ili{Italian}, or \ili{Romanian}, as shown in \tabref{bonzo}. 

\begin{table}
\caption{Syncretic alignment}
\label{bonzo}
\begin{tabular}[h]{ l l l l l l }
\lsptoprule
& \textsc{dem} 	& \textsc{comp} 	& \textsc{rel}  	& \textsc{wh}\\	
\midrule
\ili{English} & that\cellcolor[gray]{0.9} & that\cellcolor[gray]{0.9} & that\cellcolor[gray]{0.9} 	& what\\
\ili{Italian} & quello 	& che\cellcolor[gray]{0.9} 	& che\cellcolor[gray]{0.9} 	& che\cellcolor[gray]{0.9}\\
\ili{Romanian}	& acel	& c\v{a}	& ce\cellcolor[gray]{0.9}		& ce\cellcolor[gray]{0.9}\\
\lspbottomrule
\end{tabular}
\end{table}

However, when we consider the set of related forms in \ili{Russian}, as seen in \tabref{table2}, we observe that the morphological form of the \isi{demonstrative} pronoun \textit{to} is contained in \textit{\v{c}to} (henceforth indicated as \textit{\v{c}-to} where it is relevant), the form of  the declarative \isi{complementizer}, the relative pronoun, and the \isi{wh-pronoun}.

\begin{table}
\caption{Morphological containment of Dem}
\label{table2}
\begin{tabular}[t]{ l l l l l l }
\lsptoprule
& \textsc{dem} 	& \textsc{comp} 	& \textsc{rel}  	& \textsc{wh}\\	
\midrule
\ili{Russian} & to & \v{c}-to\cellcolor[gray]{0.9} & \v{c}-to\cellcolor[gray]{0.9} & \v{c}-to\cellcolor[gray]{0.9}\\
\ili{Serbo-Croatian} & to & \v{s}-to,\cellcolor[gray]{0.9} & \v{s}-to\cellcolor[gray]{0.9} & \v{s}-to\cellcolor[gray]{0.9}\\
			& 	& da						&					& \\
\lspbottomrule
\end{tabular}
\end{table}

\noindent Such a morphological \isi{containment} is opposite to what we expect if the demonstrative syntactically contains the remaining three categories. 
\par
An immediate observation that can be made about such forms as in \tabref{bonzo}, which follow the sequence in \ref{BL:fseq}, and the \ili{Russian} forms is that the first include demonstratives that are marked for definiteness while the second include a definiteless demonstrative. 
I will argue that there is a non-trivial way of accommodating \isi{demonstrative}s without definiteness marking, like the \ili{Russian} \textit{to}, into the same \isi{containment} sequence that describes containment between the demonstrative with definiteness marking, the Comp, the Rel, and the Wh. Such a solution will allow us to explain syncretic alignment and morphological containment in the cross-categorial \isi{paradigm} with these categories in a systematic way.


\section{Syncretisms with the declarative complementizer}

\subsection{Paradigm}\label{sec:paradajm}

The sample of languages in \tabref{table1} illustrates syncretic alignments consistent upon the complexity scale in \ref{BL:fseq}. \is{complementizer} \is{paradigm} \is{syncretism} The set in \tabref{table1} covers syncretisms with the nominal \isi{complementizer}, an equivalent of the \ili{English} \textit{that}, and excludes syncretisms with verbal complementizers, the categories that are derived from forms of assertive \isi{verb}s like `say'. We find verbal complementizers for instance in Yoruba, as seen in \ref{Yoruba}.

\ex. \ili{Yoruba} (\citealt[75]{Lawal1991})\label{Yoruba}
\ag. 
Ol\'u p\'e awon ti d\'e\\
Olu say they have arrived\\
\strut `Olu says they have arrived.'
\bg.
Ol\'u gb\`agb\'e p\'e B\'ol\'a ti j\'ade\\
Olu forget \textsc{comp} Bola \textsc{pfv} go.out\\
\strut `Olu forgot that Bola has gone out.'
\cg.
Ol\'u r\'ant\'i p\'e B\'ol\'a \'nsun\\
Olu remember \textsc{comp} Bola sleeping\\
\strut `Olu remembered that Bola was sleeping.' 

\begin{table}[h]
\caption{Syncretic alignment (continued)}
\label{table1}
\begin{tabular}[t]{ l l l l l l }
\lsptoprule
& \textsc{dem} 	& \textsc{comp} 	& \textsc{rel}  	& \textsc{wh}\\	
\midrule
\ili{English}: & that\cellcolor[gray]{0.9} & that\cellcolor[gray]{0.9} & that\cellcolor[gray]{0.9} 	& what\\
\ili{German}: & das\cellcolor[gray]{0.9} & dass\cellcolor[gray]{0.9} & das\cellcolor[gray]{0.9} & was\\
\ili{Dutch}: & dat\cellcolor[gray]{0.9} & dat\cellcolor[gray]{0.9} & dat\cellcolor[gray]{0.9} 	& wat\\
\ili{Afrikaans}: & dit & dat & wat\cellcolor[gray]{0.9} & wat\cellcolor[gray]{0.9}\\
\ili{Yiddish}: 	& jenc & vos,\cellcolor[gray]{0.9} & vos,\cellcolor[gray]{0.9} & vos\cellcolor[gray]{0.9}\\
		& 	  & az\cellcolor[gray]{0.75}    & az\cellcolor[gray]{0.75}\\	
\ili{Pite Saami}: & dat & att & mij\cellcolor[gray]{0.9} & mij\cellcolor[gray]{0.9}\\
\ili{Finnish}: & t\"a- & ett\"a & mi-\cellcolor[gray]{0.9} & mi-\cellcolor[gray]{0.9}\\	
Modern Greek: & ek\'ino 	& pu\cellcolor[gray]{0.9} 		& pu\cellcolor[gray]{0.9} 		& t\'i\\
\ili{Italian}: & quello 	& che\cellcolor[gray]{0.9} 	& che\cellcolor[gray]{0.9} 	& che\cellcolor[gray]{0.9}\\
\ili{Romanian}:	& acel	& c\v{a}	& ce\cellcolor[gray]{0.9}		& ce\cellcolor[gray]{0.9}\\
\ili{French}: & ce & que\cellcolor[gray]{0.9} & que\cellcolor[gray]{0.9} & que\cellcolor[gray]{0.9}\\
\ili{Basque}: & hura & -ela & -n & zer\\
\lspbottomrule
\end{tabular}
\end{table}

\citet{Lawal1991} shows that in \ili{Yoruba}, \textit{p\'e} is syncretic form for the verb `say' and serves as a complementizer for clauses embedded under assertive verbs like `say' as well as verbs of cognition like `forget' or \mbox{`remember',} as seen in \ref{Yoruba}. At the same time, \citet[76]{Lawal1991} argues that the distribution of \textit{p\'e} is that of a \isi{complementizer}, as it heads preposed \ili{English}-like \textit{that}-clauses, as in \ref{Yayaya}.\largerpage

\exg.
p\'e a jo lo d\'ara\\
\textsc{comp} we together went good\\
\strut `that we went together was good'\label{Yayaya}

Verbal \isi{complementizer}s are well-attested cross-lingustically (see for instance \citealt{DixonAik2006}) and they can co-exist with nominal complementizers within one language as for example in \ili{Hausa}. \ili{Hausa} has a verbal declarative complementizer \textit{c\^eewaa} based on `say', as in \ref{cewa}, which is not used after the \isi{verb}  \textit{c\^ee}, in which case  the nominal complemetizer \textit{wai} is used, as in \ref{wai}.

\ex. \ili{Hausa} (\citealt[96--97]{Dimm1989})\label{Hausa}
\ag. sun tabb\'ataa man\'a [ c\^eewaa niisan raanaa d\'aga nan yaa yi m\^il d\'a {yaw\'aa ]}\\
\textsc{3a} assure us {} \textsc{comp} distance sun from here \textsc{3a} do mile with {many }\\
\strut `They assured us that the sun is far away from here.'\label{cewa}
\bg. an c\^ee [ wai yaa bi wani {mac\v{i}ijii ]}\\
one say {} \textsc{comp} \textsc{3a} follow some {snake }\\
\strut `It was said that he followed some snake.' \label{wai}

The remainder of the discussion in this chapter focuses on the paradigm with the nominal \isi{complementizer} and completely disregards verbal complementizers.

\subsection{Analysis in \citeauthor{BaunazLander2017} (\citeyear{BaunazLander2017,BaunazLander2018})}

Baunaz \& Lander propose an analysis of the syncretic alignment shown in \tabref{table1} 
based on a complex underlying tree structure as in \ref{BL:tree}, whose left branch spells out as the prefix on a nominal base (marked here as the N triangle) and whose right branch spells out an invariant inflectional suffix (marked here as the $\phi$ triangle). Given the entries  for the \ili{English} morphemes \textit{wh} and \textit{th} as in \Next, they come out as prefixes on the nominal stem \textit{-a}, which is suffixed with the invariant inflectional marker \textit{-t}.

\ex. Lexical entries for the \ili{English} \textit{wh} and \textit{th} (1st approximation)\label{Eng:1stapprox}
\a. [ Wh [ \textit{n} ]] $\Leftrightarrow$ \textit{wh}\label{old-wh}
\b. [ Dem [ Comp [ Rel [ Wh \textit{n} ]]]] $\Leftrightarrow$ \textit{th}\label{old-th}

\noindent Using phrasal \isi{spell-out} and the \isi{Superset Principle}, the phrasal nodes DemP, CompP, and RelP all spell-out as \textit{th-} as they constitute, respectively, the superset and the subset structures of the lexical entry in \ref{old-th}. The WhP node, also a subset of the entry in \ref{old-th}, is spelled out as \textit{wh} on the strength of the Elsewhere clause, since \ref{old-wh} is a more specific match for the WhP node than \ref{old-th}. 

Two remarks are in place before we proceed. First, it is important to note that the labelling used in \ref{BL:tree} is a simplified way to illustrate Baunaz \& Lander's analysis, in the sense that a ``\isi{demonstrative} pronoun,  a ``\isi{complementizer}'', a ``\isi{relativizer}'', and a ``\isi{wh-pronoun}'' lexicalize all three branches of the tree \ref{BL:tree} in their analysis, irrespective of morphological complexity of these categories. This is a natural consequence of phrasal \isi{spell-out}. For instance, in Baunaz \& Lander's architecture, the \ili{Italian} \textit{che} is analyzed as a bi-morphemic \textit{ch-e}, where the \textit{ch-} \isi{morpheme} spells out both the left branch and the nominal stem of the representation in \ref{BL:tree} as a portmanteau while \textit{-e} spells out the right branch, the invariant $\phi$ suffix, as in \NNext.\footnote{The drawback of the analysis where \textit{ch-} is a portmanteau realization of two independent branches of an underlying representation is that the constituent that corresponds to the the morphological stem (the middle branch) cannot be overtly identi­fied, since its decomposition is not possible.
}%end of fn


\ex.\label{BL:tree} Lexicalization of the \ili{English} \textit{that} and \textit{what} in \cite{BaunazLander2017}\footnote{For the sake of concreteness, let us note that the nominal element at the bottom of the left branch of this tree, the stem for the merger of the Wh \isi{feature} labelled here as \textit{n}, is described as a classifier-like lexical noun in \cite{BaunazLander2018} and as non-lexical indeterminate noun in \cite{Baunaz-Lander-Glossa}. This issue is, however, orthogonal to what follows.}\\[-0.5ex]
\begin{forest}nice empty nodes, for tree={fit=tight,calign angle=63}
 [~, l=0pt [, l=0pt
 [\textsc{prefix:}\\{\hspace{27pt}DemP $\Rightarrow$ \textit{th}},tier=cat, l=0pt [Dem]
 [CompP  [Comp][RelP [Rel ][WhP [Wh] [\textit{n}]]{\draw (.east) node[right]{$\Rightarrow$ \textit{wh}}; }
 ]]
 ][\textsc{base:}\\{\hspace{20pt}N $\Rightarrow$ \textit{a}},tier=cat, l=0pt [{\color{white}$--$}, roof]
 ]]
 [\textsc{suffix:}\\{\hspace{20pt}$\phi$ $\Rightarrow$ \textit{t}},tier=cat, l=0pt,yshift=4pt  [{\color{white}$--$}, roof]]
 ]
\end{forest}


\ex.\label{Italian}
\a.  \ili{Italian} \isi{complementizer} \textit{che}\\[-1ex]
\begin{forest}nice empty nodes, for tree={l sep=0.65em,l=0,calign angle=63}
 [~, s sep=20pt
 [CompP
 [CompP  [Comp][RelP [Rel ][WhP [Wh] [\textit{n}]]
 ]][N [{\color{white}$--$}, roof]]]{\draw (.east) node[right]{$\Rightarrow$ \textit{ch}}; }
 [$\phi$ [{\color{white}$--$}, roof]]{\draw (.east) node[right]{$\Rightarrow$ \textit{e}}; }]
\end{forest}\\[2ex]
\b. \ili{Italian} relativizer \textit{che}\\[-1ex]
\begin{forest}nice empty nodes, for tree={l sep=0.65em,l=0,calign angle=63}
  [~, s sep=20pt
 [RelP
 [RelP [Rel ][WhP [Wh] [\textit{n}]]][N [{\color{white}$--$}, roof]]]{\draw (.east) node[right]{$\Rightarrow$ \textit{ch}}; }
 [$\phi$ [{\color{white}$--$}, roof]]{\draw (.east) node[right]{$\Rightarrow$ \textit{e}}; }]
\end{forest}\\[2ex]
\c. \ili{Italian} wh-pronoun \textit{che}\\[-1ex]
\begin{forest} nice empty nodes, for tree={l sep=0.65em,l=0,calign angle=63}
 [~, s sep=25pt
[WhP
[WhP [Wh] [\textit{n}]][N [{\color{white}$--$}, roof]]]{\draw (.east) node[right]{$\Rightarrow$ \textit{ch}}; }
[$\phi$ [{\color{white}$--$}, roof]]{\draw (.east) node[right]{$\Rightarrow$ \textit{e}}; }]
\end{forest}


\noindent The terminal nodes labelled as Dem, Comp, Rel, and Wh  should be understood here as subcomponents of the demonstrative, the \isi{complementizer}, etc., rather than features that solitarily encode the properties of the categories they head. \is{terminal node} For example, the spatial deictic contrast in \ili{English} demonstratives \textit{th-is}/\textit{th-at} is morphologically realized by \textit{-is}/\textit{-at}, not by the definite prefix \textit{th-}. For this reason, \citet{BaunazLander2018} describe the DemP in \ref{BL:tree} as an instantiation of the definite article, a subcomponent of the \isi{demonstrative} rather than the source of spatial deixis, an issue that will be taken up in a greater detail in what follows.
\par
The other thing to bear in mind is that the four categories -- Dem, Comp, Rel, and Wh -- should not be necessarily treated as inherently simplex  beyond the \isi{containment} relation that holds between them. For example, it is clear that the RelP-layer of structure that corresponds to the \isi{relativizer} (as a grammatical category) must be inherently complex enough to cover two types of relativizers found for instance in \ili{Polish}: the invariant \textit{co}, which is syncretic with the \isi{wh-pronoun} `what', and the case-inflected inflected \textit{kt\'ory}, which morphologically includes the person wh-pronoun \textit{kto} `who', but which, just like the invariant relativizer \textit{co}, is compatible with $+$/$-$person] and $+$/$-$animate head nouns, as in \Next.\pagebreak

\ex. \ili{Polish}\label{Pol:inv-co}
\ag. 
poci\k{a}g \{ co / {kt\'ory \}} przyjecha\l {} za p\'o\'zno\\
train.\textsc{nom} {} \textsc{rel}\textsubscript{inv} {} \textsc{rel.msc.nom}  arrived.\textsc{3sg.msc} too late\\
\strut `the train that arrived too late'
\bg.
dziewczyn\k{e} \{ co / {kt\'or\k{a} \}} widzieli\'smy w kinie\\
girl.\textsc{acc} {} \textsc{rel}\textsubscript{inv} {} \textsc{rel.fem.acc} saw.\textsc{1pl} in cinema\\
\strut `the girl that we saw in the cinema'  

While both \textit{co} and \textit{kt\'ory} can appear in subject and object relative clauses in \ili{Polish}, as in \ref{uno}, there are certain differences between \isi{relative clause}s with both types of \isi{relativizer}s.

\ex.\label{uno} \ili{Polish}
\ag. zegar \{ co / {kt\'ory \}} wybi\l {} dwunast\k{a}\\
clock {} \textsc{rel}\textsubscript{inv} {} \textsc{rel.msc.nom} struck.\textsc{3sg.msc} twelve\\
\strut `the clock which struck twelve o'clock'
\bg.
dziewczyna \{ co / {kt\'ora \}} widzia\l a nas w kinie\\
girl {} \textsc{rel}\textsubscript{inv} {} \textsc{rel.fem.nom} saw.\textsc{3sg.fem} us in cinema\\
\strut `the girl that saw us in the cinema'  

For instance, as noted in \cite{Mykowiecka2001}, the resumptive pronoun (the neuter accusative \textit{je} `it' in \pref{dwa}) must be adjacent to \textit{co} but it does not appear in \textit{kt\'ory}-relatives, as in \ref{trzy}: 

\exg.
 wino, co (je) Adam (*je) przyni\'os\l\\
wine \textsc{rel}\textsubscript{inv} \ \ it Adam \ \ it brought\\
\strut `the wine that Adam brought'\label{dwa}

\exg.
wino, kt\'ore (*je) Adam (*je) przyni\'os\l\\
wine \textsc{rel.neu.acc} \ \ it Adam \ \ it brought\\
\strut `the wine that Adam brought'\label{trzy}

As observed in \cite{Szczegielniak2005}, when the resumptive pronoun is embedded, it can appear in both types of relatives, as seen the following:

\exg. wino, \{ co / {kt\'ore \}} wszyscy wiedz\k{a}, \.ze (je) Adam przyni\'os\l\\
wine {} \textsc{rel}\textsubscript{inv} {} \textsc{rel.neu.acc} everybody know.\textsc{3pl} \textsc{comp} \ \ it Adam brought\\
\strut `the wine that everybody knows that Adam brought'

\noindent The degree of the inherent complexity of the categories Dem, Comp, Rel, and Wh is largely irrelevant to the containment relation which holds between them, though. That is, we find some cross-linguistic evidence beyond \isi{syncretism} for the claim that such a relation holds between these categories. For instance, in \ili{Hungarian}, the uninflected stem of the wh-pronoun \textit{mi-} `what' is morphologically contained within the stem of the \isi{relativizer} \textit{a-mi-}, as seen in \tabref{Ungarowie}.

\begin{table}
\caption{Hungarian paradigm\label{Ungarowie}}
\begin{tabular}[t]{ l l l l l l }
\lsptoprule
\textsc{dem} 	& \textsc{comp} & \textsc{rel}  	& \textsc{wh}\\	
\midrule
az- & hogy & a-mi- & mi-\\
\lspbottomrule
\end{tabular}
\end{table}

\noindent The following examples illustrate the use of \textit{mi-} as a \isi{wh-pronoun} and \textit{a-mi-} as a relativizer (both suffixed with the accusative \textit{-t}):

\ex. \ili{Hungarian} (\citealt[11]{Kenesei1998})
\ag.[]\hspace{-22pt}Mi-t 	tal\'alt 	mindenki?\\
\hspace{-22pt}what-\textsc{acc} 	found.\textsc{3sg} 	everyone\\
\hspace{-22pt}\strut `What did everyone find?'

\ex. Hungrian (\citealt[136]{Rounds2001})
\ag.[]\hspace{-22pt}Elolvostam a 	k\"onvet 	ami-t k\"uld\'et nekem.\\
\hspace{-22pt}sent.\textsc{1sg} the	book.\textsc{-acc}	\textsc{rel-acc}	sent.\textsc{2sg} me\\
\hspace{-22pt}\strut `I read the book that you sent me.'


\noindent The morphological \isi{containment} of Wh in Rel is an instance of a more general pattern in \ili{Hungarian}, where relativizers are formed by adding the prefix \textit{a-} to wh-pronouns other than `what', as for instance \textit{a-ki} `\textsc{rel}-who', \textit{a-melyik} `\textsc{rel-}which', or \textit{a-mennyi} `\textsc{rel-}how.many' (cf. \citealt[40]{Kenesei1998}).
This yields a structure of \textit{a-mi-} as in the following: 

\ex. [\textsubscript{RelP} a [\textsubscript{WhP} mi ]]

\noindent
However, while the containment of Wh inside Rel in \ili{Hungarian} is in agreement with the hierarchy in \ref{BL:fseq}, defined on the basis of cross-lin\-guis\-tic\-ally attested syncretisms, the morphological containment of a \isi{demonstrative} pronoun inside the remaining three categories that we find in \ili{Russian} and \ili{Serbo-Croatian} is not. 

\section{An ordering paradox with the demonstrative}

\noindent Assuming the way the facts are described and set up in \citeauthor{BaunazLander2017} (\citeyear{BaunazLander2017,BaunazLander2018}),
the Dem$=$Comp \isi{syncretism} found in certain languages, in particular in the \ili{West Germanic} subgroup (\ili{English}, \ili{Dutch}, and \ili{German}) as shown in \tabref{table1}, points to the hierarchy ``Dem\,$>$\,Comp\,$>$\,Rel\,$>$\,Wh''. Some other languages, however, indicate that the order between these categories is different.  In particular, a challenge to ``Dem\,$>$\,Comp\,$>$\,Rel\,$>$\,Wh'' comes from morphological \isi{containment} of Dem in the structure of the other three categories, which we find in \ili{Slavic} languages like \ili{Russian} or \ili{Serbo-Croatian}, as shown in \tabref{table2} (repeated below):

\begin{table}
\caption{Morphological containment of Dem}
\begin{tabular}[t]{ l l l l l l }
\lsptoprule
& \textsc{dem} 	& \textsc{comp} 	& \textsc{rel}  	& \textsc{wh}\\	
\midrule
\ili{Russian} & to & \v{c}-to\cellcolor[gray]{0.9} & \v{c}-to\cellcolor[gray]{0.9} & \v{c}-to\cellcolor[gray]{0.9}\\
\ili{Serbo-Croatian} & to & \v{s}-to,\cellcolor[gray]{0.9} & \v{s}-to\cellcolor[gray]{0.9} & \v{s}-to\cellcolor[gray]{0.9}\\
			& 	& da						&					& \\
\lspbottomrule
\end{tabular}
\end{table}

\noindent 
\makebox[\linewidth][s]{The \ili{Russian} \isi{paradigm} has the neuter singular \isi{demonstrative} pronoun \textit{to} in-}\\
cluded in the structure of all three remaining categories. The \ili{Serbo-Croatian} shows a slightly different paradigm in that \textit{\v{s}to} serves as a \isi{complementizer} with only a subset of \isi{verb}s selecting for declarative clauses. For instance, as shown in the following, the complementizer \textit{\v{s}to} heads clauses embedded under the \isi{verb} \textit{smetati} `bother, annoy' while the complementizer that heads declarative clauses introduced by the verb \textit{misliti} `think' is \textit{da}.

\ex. \ili{Serbo-Croatian} (\citealt[114]{Mihalicek})
\ag. Ani smeta \{ {\v{s}to \,/} {*da \}} Marko stalno spava.\\
Ana.\textsc{dat} bother.\textsc{3sg} {} {\textsc{comp}} {\textsc{\phantom{l}comp}}  Marko.\textsc{nom} always sleep.\textsc{3sg}\\
\strut `It bothers Ana that Marko is always sleeping.' 
\bg. Ana misli \{ {*\v{s}to \,/} {da \}} Marka spava.\\
Ana.\textsc{nom} think.\textsc{3sg} {} {\phantom{l}\textsc{comp}} {\textsc{comp}}  Marko.\textsc{nom} sleep.\textsc{3sg}\\
\strut `Ana thinks that Marko is sleeping.'

\noindent Descriptively speaking, the morphological containment of Dem within Comp, Rel, and Wh is paradoxical -- or counter-intuitive at best -- if the \isi{demonstrative} pronoun is the structurally biggest category in the paradigm.\largerpage[2]
\par
This problem is recognized in \cite{BaunazLander2018}, who propose to solve it by eliminating demonstratives without definiteness marking (Dem\textsubscript{indef} for short) from the sequence so that it applies only to languages with morphologically marked definiteness on demonstratives (Dem\textsubscript{def} for short). The updated complexity scale looks now as in:

\ex.\label{dem-def} Dem\textsubscript{def}\,$>$\,Comp\,$>$\,Rel\,$>$\,Wh

More precisely, \cite{BaunazLander2018} argue that only Dem\textsubscript{def} projects as the top layer of the left branch of the tree in \ref{BL:tree} and in languages like \ili{Russian} and \ili{Serbo-Croatian} Dem\textsubscript{indef} is restricted to the nominal stem, i.e. the middle branch of the tree in \ref{BL:tree} marked as ``N''. 
\par
However, such a solution creates a paradox: on the one hand the hierarchy in \ref{dem-def} 
applies to the categories that are supposed to always \isi{spell-out} all three branches of the tree in \ref{BL:tree} (either synthetically as in \ili{English} or as a portmanteau in \ili{Italian}), on the other hand it is defined only on the basis of the left branch of that tree, excluding the middle and the right branch.
\par
 In order to keep the \isi{demonstrative} pronouns that are not marked for definiteness in the picture (i.e. in \ili{Slavic} languages like \ili{Russian}, \ili{Polish}, or \ili{Czech} that lack definiteness morphology), unless indicated otherwise, I will use the ``Dem'' label more broadly so that it describes both kinds of demonstrative pronouns. Whenever it will be needed to differentiate between demonstratives with and without definiteness morphology, I will refer to them specifically as Dem\textsubscript{def} and Dem\textsubscript{indef}, respectively.
 \par
Since the \ili{Russian} \textit{\v{c}to} covers three cells of the \isi{paradigm} in \tabref{table2}  and, unlike the \ili{Serbo-Croatian} \textit{\v{s}to}, is the only possible form of the declarative \isi{complementizer}, I will be focusing mostly on the \ili{Russian} paradigm. To the extent that I can tell, the result for the \ili{Russian} \textit{\v{c}to}, however, carries over to the \ili{Serbo-Croatian} paradigm with the syncretic Wh/Rel/Comp \textit{\v{s}to}, too.

\section{Low indefinite demonstratives}\largerpage

\noindent It appears that what constitutes an obstacle in resolving the ordering paradoxes for the sequence in \ref{dem-def} is that it describes the categories realized by the three branches of the tree in \ref{BL:tree} while the sequence applies only to the properties of the left branch. Let us, thus, consider what happens if we relax Baunaz \& Lander's constraint that a \isi{demonstrative}, a complementizer, a \isi{relativizer} and a \isi{wh-pronoun} are always realizations of the three branches of the tree in \ref{BL:tree}.

\subsection{Severing spatial deixis from definiteness}

I have argued elsewhere (\citealt{Wiland-PSiCL}) that the base for the formation of the pronoun `what' in \ili{Slavic} is the indefinite \isi{demonstrative}, which constitutes the bottom of a monotonically growing singleton projection line, as in:

\ex.\label{wh-dem}
\begin{forest}nice empty nodes, for tree={l sep=0.65em,l=0,calign angle=63}
 [WhP
 [Wh ] [\hspace{15pt}Dem\textsubscript{indef} 
 [Dem] [NP ]]]]
 \end{forest}

More precisely, I have argued there that the base for the formation of the \ili{Polish} \textit{co} `what' and \ili{Russian} \textit{\v{c}to} `what' is the medial demonstrative \textit{to}. The evidence comes from the decomposition of spatial deixis into three categories: the proximal (close to speaker), the medial (close to hearer), and the distal (far from speaker and hearer) advanced in \cite{Lander-Haegeman2016}, who argue that such a three-way contrast reflects a universal syntactic structure, as in \ref{LH:Dem}\,(where Deix\textsubscript{n} stands for an abstract spatial deictic \isi{feature}).

\ex.\label{LH:Dem}
\begin{forest}nice empty nodes, for tree={l sep=0.65em,l=0,calign angle=63}
 [DistP
 [Deix$_3$ ] [MedP 
 [Deix$_2$] [ProxP 
 [Deix$_1$] ]]]]
 \end{forest} 

\noindent In a phrasal spell-out approach made a case for in the present work, deictic morphology is the realization of the subset(s) or the superset of that representation. For example, the proximal-medial-distal contrast in \ili{Japanese} is realized sui generically by three distinct \isi{morpheme}s.

\ex. \ili{Japanese} (\citealt[97]{Hoji-etal2003})
\ag.[]\hspace{-22pt}ko- / so- / a-\\
\hspace{-22pt}\textsc{prox} {} \textsc{med} {} \textsc{dist}\\

This reveals that \ili{Japanese} has the lexical entries for \textit{ko}, \textit{so} and \textit{a} as specified in:

\ex. Lexical entries for the \ili{Japanese} \textit{ko}, \textit{so}, and \textit{a}
\a. [\textsubscript{ProxP} Deix$_{1}$ ] $\Leftrightarrow$ \textit{ko}
\b. [\textsubscript{MedP}  Deix$_{2}$ [\textsubscript{ProxP} Deix$_{1}$ ]] $\Leftrightarrow$ \textit{so}
\c. [\textsubscript{DistP}  Deix$_{3}$ [\textsubscript{MedP} Deix$_{2}$ [\textsubscript{ProxP} Deix$_{1}$ ]]] $\Leftrightarrow$ \textit{a}

which results in each layer of the tree in \ref{LH:Dem} being lexicalized unequivocally, as indicated in the following:

\ex.\label{Jap:s}Spell-out of the tree in \ref{LH:Dem} in \ili{Japanese}\\[1ex]
\begin{forest}nice empty nodes, for tree={l sep=0.65em,l=0,calign angle=63}
 [DistP 
 [Deix$_{3}$ ] [MedP 
 [Deix$_{2}$] [ProxP
 [Deix$_{1}$] ]{ \draw (.east) node[right]{$\Rightarrow$ \textit{ko}}; }
 ]{ \draw (.east) node[right]{$\Rightarrow$ \textit{so}}; }
 ]{ \draw (.east) node[right]{$\Rightarrow$ \textit{a}}; }
\end{forest}


\noindent Languages differ with respect to the number of exponents which realize the representation in \ref{LH:Dem}. For instance, the proximal-medial-distal contrast is realized in \ili{French} by a singleton lexical item \textit{ce} (and its \is{allomorphy} allomorphs), as in:\footnote{The \ili{French} syncretic Prox$=$Med$=$Dist \isi{demonstrative} \textit{ce} modifies masculine nous that begin with a consonant, the other two allomorphs are \textit{cet}, which modifies masculine nouns that begin with a vowel, as in (i) and \textit{cette}, which modifies feminine nouns, as in (ii):

\noindent\parbox{\linguexfootnotewidth}{\exg.[(i)] 
cet oncle\\
\textsc{prox/med/dist} uncle.\textsc{msc}\\
\strut `this/that uncle'

\exg.[(ii)] 
cette taverne\\
\textsc{prox/med/dist} tavern.\textsc{fem}\\
\strut `this/that tavern'

}} %end of fn 

\ex. \ili{French}\label{French}
\ag.[]\hspace{-22pt}ce journal\\
\hspace{-22pt}\textsc{prox/med/dist} newspaper.\textsc{msc}\\
\hspace{-22pt}\strut `this/that newspaper'

Such a one-to-many relation indicates that the \ili{French} \textit{ce} is specified for a superset of features which describe the proximal--medial--distal contrast, as indicated in \ref{gilotyna}.

\ex.\label{gilotyna} Lexical entry for the \ili{French} \textit{ce}\\[1ex]
[\textsubscript{DistP}  Deix$_{3}$ [\textsubscript{MedP} Deix$_{2}$ [\textsubscript{ProxP} Deix$_{1}$ ]]] $\Leftrightarrow$ \textit{ce}

In fact, if we follow Baunaz \& Lander's bi-morphemic analysis of the \ili{Italian} \textit{che} as in \ref{Italian} for a little longer and extend it to the \ili{French} \textit{ce}, it is only the \textit{c-} morpheme that appears to realize the spatial deictic contrast while the \textit{-e} is an invariant ``$\phi$-agreement'' suffix. Hence, on the strength of the \isi{Superset Principle}, the \ili{French} lexical item \textit{c-} spells out either the superset or any subset of that tree, as in \ref{francuz}, resulting in its different readings depending on its size, as indicated in \ref{francuz}.


\ex.\label{francuz}
\a. \ili{French} distal \textit{ce}\\
\begin{forest}nice empty nodes, for tree={l sep=0.85em,l=0,calign angle=63}
 [
 [DistP 
 [Deix$_{3}$ ] [MedP 
 [Deix$_{2}$] [ProxP
 [Deix$_{1}$] ]
 ]
 ]{\draw (.east) node[right]{$\Rightarrow$ \textit{c}}; }
 [$\phi$[{\color{white}$--$}, roof]]{\draw (.east) node[right]{$\Rightarrow$ \textit{e}}; }
 ]
\end{forest}
\b. \ili{French} medial \textit{ce}\\
\begin{forest}nice empty nodes, for tree={l sep=0.8em,l=0,calign angle=63}
 [, s sep=25pt
 [MedP, s sep=15pt
 [Deix$_{2}$] [ProxP
 [Deix$_{1}$] ]
 ]{ \draw (.east) node[right]{$\Rightarrow$ \textit{c}}; }
 [$\phi$[{\color{white}$--$}, roof]]{\draw (.east) node[right]{$\Rightarrow$ \textit{e}}; }
 ]
\end{forest}
\c. \ili{French} proximal \textit{ce}\\
\begin{forest}nice empty nodes, for tree={l sep=0.8em,l=0,calign angle=63}
 [, s sep=35pt
 [ProxP
 [Deix$_{1}$] ]{ \draw (.east) node[right]{$\Rightarrow$ \textit{c}}; }
 [$\phi$[{\color{white}$--$}, roof]]{\draw (.east) node[right]{$\Rightarrow$ \textit{e}}; }
 ]
\end{forest}

\noindent Just like the \ili{English} \textit{this} and \textit{that}, \ili{Polish} and \ili{Russian} have two distinct pronouns that realize the three-way deictic contrast. The \ili{Polish} \textit{to} describes closeness to speaker and hearer, while \textit{tamto} univocally describes remoteness from both speaker and hearer, as seen in \Next. 

\ex. \ili{Polish}\label{P}
\ag.[]\hspace{-22pt}to / tamto auto\\
\hspace{-22pt}\textsc{prox/med} {} \textsc{dist} car.\textsc{neu.nom}\\

Unlike in \ili{Polish}, the \ili{Russian} \textit{eto} univocally describes closeness to the speaker while the \ili{Russian} \textit{to} describes closeness to the hearer and remoteness from both speaker and hearer, as for instance in \ref{R}:

\ex. \ili{Russian}\label{R} 
\ag.[]\hspace{-22pt}\`eto / to okno\\
\hspace{-22pt}\textsc{prox} {} \textsc{med/dist} window.\textsc{neu.nom}\\

This clearly shows that the only subset of the tree in \ref{LH:Dem} which is realized by both \ili{Polish} and \ili{Russian} \textit{to} is the medial subtree, as in \Next, the observation that will become important in what follows. 

\ex.\label{simplified:to}  Simplified representation of the medial \isi{demonstrative} pronoun \textit{to} in \ili{Polish} and \ili{Russian} \\[1ex]
\begin{forest}nice empty nodes, for tree={l sep=0.8em,l=0,calign angle=63}
 [MedP 
 [Deix$_{2}$] [ProxP
 [Deix$_{1}$] ]
 ]{ \draw (.east) node[right]{$\Rightarrow$ \textit{to} }; }
\end{forest}

\noindent Before the representation of \textit{to} in \ref{simplified:to} is refined into a separate stem \textit{t-} and an inflection suffix \textit{-o}, a short excursus about the structure of the \ili{Polish} distal demonstrative \textit{tamto} is called for here. Namely, it morphologically contains the proximal/medial \textit{to} along the distal locative \textit{tam} `there'. I have argued in \cite{Wiland-PSiCL} that \textit{tam-to} is in fact an instance of a reinforcer-\isi{demonstrative} construction, a pattern more widely attested in \ili{Romance} and \ili{Germanic} (see e.g. \citealt{Bernstein1997}), as for instance in \ili{Afrikaans}, where the locative reinforcer is prefixed onto the demonstrative in the pre-nomininal position, as seen in the following.

\ex. \ili{Afrikaans} (\citealt[226--227]{Roehrs2010})
\ag.[]\hspace{-22pt}hier-die mooi meisie\\
\hspace{-22pt}here.this pretty girl\\
\hspace{-22pt}\strut `this pretty girl'

The argument for the reinforcer-\isi{demonstrative} analysis of the \ili{Polish} \textit{tam-to} is based on the observation that there is a contrast between the distribution of the \ili{Polish} proximal locative \textit{tu} `here' and distal locative \textit{tam} `there' with demonstrative pronouns. While \textit{tu} `here' can be optionally placed after the  proximal/medial demonstrative pronoun \textit{to} as in \ref{pikaczu}  (just like \textit{here} in a substandard \ili{English} \textit{this here big house}), \textit{tam} `there' cannot function as free form reinforcer placed in the distal demonstrative \textit{tam-to}, which contains it, as seen in \ref{pikapika}:\largerpage

\exg. to \{ tu / {tam \}} dziecko\\
\textsc{prox/med} {} here {} there child.\textsc{neu.nom}\\
\strut `this here child'\label{pikaczu}

\exg. tamto (*tam) dziecko\\
\textsc{dist} \phantom{cl}there child.\textsc{neu.nom}\\
\strut intended `that there child'\label{pikapika}

At the same time, *\textit{tu-to} `here-\textsc{prox/med}' is ill-formed in \ili{Polish}, a scenario which indicates that only the distal demonstrative \textit{tam-to} but not the proximal/medial demonstrative \textit{to} includes a locative reinforcer in its structure. Thus, the structure of the distal \textit{tam-to} appears to be derived along the lines of \citeauthor{Leu2007}'s \citeyearpar{Leu2007} analysis of \ili{Germanic} demonstratives, whereby the locative \textit{tam} raises from its canonical pre-nominal position to the pre-\isi{demonstrative} position yielding the reinforcer-demonstrative item, as indicated in the following:
 
 \ex.
 \ag. to tam dziecko \\
\textsc{prox/med} there child.\textsc{neu.nom}\\
\strut intended `that there child'
\setlength{\arrowht}{2ex}
\newcommand*\cgdepthstrut{{\vrule height 0pt depth \arrowht width 0pt}}
\renewcommand\eachwordone{\cgdepthstrut\rmfamily}
\renewcommand\glt{\vskip -\topsep}
\let\trans=\glt
\newcommand\arrowex{\setlength{\arrowht}{1ex}\ex}
\b. \tikzmark{tam}-to {\ } \tikzmark{t} {\ } dziecko  
 \arrow{t}{tam}

\vskip 0.25cm
\noindent Before we turn the observation that \textit{to} spells out the medial layer in both \ili{Polish} and \ili{Russian} into a solution to the problem of morphological \isi{containment} of the demonstrative \textit{to} inside the \ili{Russian} \textit{\v{c}-to}, let us first refine the representation of the demonstrative pronoun in \ref{simplified:to}.
\par
It is clear that spatial deixis is not inherently pronominal, a point also made explicit in \cite{Lander-Haegeman2016}. For instance, the \ili{Japanese} spatial deictic markers \textit{ko-}, \textit{so-}, and \textit{a-} can merge with pronominal, determiner, and adverbial stems, as seen in \tabref{Jap:stems}, forming demonstrative pronouns, demonstrative determiners, and \isi{demonstrative} adverbs.\footnote{The stem \textit{-re}, as in \textit{so-re} in  \tabref{Jap:stems}, means `thing' and the stem \textit{-ko}, as in \textit{so-ko}, means `place'. \ili{Japanese} demonstratives can also merge directly with other nominal stems, as e.g. \textit{ko-tira} `\textsc{prox}-way', \textit{so-tira} `\textsc{med}-way', \textit{a-tira} `\textsc{dist}-way', or \textit{ko-itu} `\textsc{prox}-guy', \textit{so-itu} `\textsc{med}-guy', \textit{a-itu} `\textsc{dist}-guy' (\citealt[97]{Hoji-etal2003}). 
} %end of fn 

\begin{table}[H]
\caption{Categories of demonstratives in \ili{Japanese} (\citealt{Kuno1973})}
\label{Jap:stems} 
\begin{tabular}[h]{ l l l l l l }
\lsptoprule	
		& pronoun 	& determiner 	& adverb\\\midrule
proximal	& ko-re		& ko-no		& ko-ko\\
medial	& so-re		& so-no		& so-ko\\
distal		& a-re		& a-no		& a-soko\\
\lspbottomrule
\end{tabular}
\end{table}

In turn, what indicates that spatial deixis in the \ili{Polish} and \ili{Russian} \isi{demonstrative} pronoun \textit{to} merges with a nominal stem is the fact that it is inflected for case, which shows up in the obligatory case concord between the demonstrative pronoun and the head noun. This is illustrated in \ref{przezlampe} on the example of the \ili{Polish} singular accusative suffix of the feminine declension and instrumental suffix of the masculine declension.

\ex.\label{przezlampe}
\ag. przez t-\k{e} lamp-\k{e}\\
by \textsc{prox/med-acc.fem.sg} lamp-\textsc{acc.fem.sg}\\
\strut `by this/that lamp'
\bg. t-ym klucz-em\\
\textsc{prox/med-inst.msc.sg} key-\textsc{inst.msc.sg}\\
\strut `with this/that key'

The \textit{-o} suffix in the bi-morphemic \textit{t-o} is a syncretic marker for neuter nominative and accusative, as indicated in the singular declension \isi{paradigm}s in \tabref{to:decl}.
  
  
\begin{table}
\caption{Declension of \textit{to} in \ili{Polish} (left) and \ili{Russian} (right)}
\label{to:decl} 
\begin{tabular}[t]{ l l l l l l }
\lsptoprule	
& \textsc{msc} & \textsc{fem} & \textsc{neu}\\\hline
  \textsc{nom} & t-en & t-a & t-o\cellcolor[gray]{0.9}\\
  \textsc{acc}  & t-ego & t-\k{e} & t-o\cellcolor[gray]{0.9}\\
  \textsc{gen} & t-ego & t-ej & t-ego\\
  \textsc{dat} & t-emu & t-ej & t-emu\\
  \textsc{loc}  & t-ym & t-ej & t-ym\\
  \textsc{inst} & t-ym & t-\k{a} & t-ym\\
  \lspbottomrule
\end{tabular}\hspace{.05\linewidth}
% \caption{Declension of \textit{to} in \ili{Russian}}
% \label{to:Rus-decl}
\begin{tabular}[t]{ l l l l l l }
\lsptoprule	
& \textsc{msc} & \textsc{fem} & \textsc{neu}\\\hline
  \textsc{nom} & t-ot & t-a & t-o\cellcolor[gray]{0.9}\\
  \textsc{acc}  & t-ogo & t-u & t-o\cellcolor[gray]{0.9}\\
  \textsc{gen} & t-ogo & t-oj & t-ogo\\
  \textsc{dat}  & t-omu & t-oj & t-omu\\
  \textsc{loc} & t-om & t-oj & t-om\\
  \textsc{inst} & t-im & t-oj & t-im\\
  \lspbottomrule
\end{tabular}
\end{table}

\par
At this point, let us return for a moment to the inventory of \ili{Russian} demonstratives shown in \ref{R}, involving the proximal \textit{\`eto} and the medial/distal \textit{to}.  Given that the \ili{Russian} \textit{\`eto} is realizing a subset structure of \textit{to} and the description of the \textit{-o} as a suffix, the morphological structure of the \ili{Russian} proximal pronoun appears to be \textit{\`et-o}. The alternative with a tri-mor\-phe\-mic \textit{\`e-t-o} would require a substantially different analysis of the \ili{Russian} \isi{demonstrative}s (plus perhaps controlling for the fact that \textit{\`e-} does not appear in a related context elsewhere). I will therefore cautiously assume that the \ili{Russian} \textit{\`et-} is a singleton \isi{morpheme}.
\par
The presence of the case suffix in the structure of \textit{t-o} indicates that the \textit{t-} is not a ``pure'' marker of spatial deixis like the \ili{Japanese} \textit{ko-}, \textit{so-}, and \textit{a-} are, but that it realizes both spatial deixis and a stem which is inflected for case. The two kinds of stems that form case inflected categories in \ili{Polish} and \ili{Russian} are nouns and adjectives (these two classes obviously include not only lexical nouns and adjectives but also the categories that are based on nominal and adjectival \isi{root}s, such as case inflected numerals and quantifiers). Along personal pronouns, case inflected \textit{to} can serve as a pro-form for noun phrases rather than adjective phrases, as illustrated by the following example from \ili{Polish}:\footnote{The presence of a locative \textit{tym} in \ref{pro-to} is not accidental as it gives us a clearer example of a nominal pro-form than a neuter singular \textit{to} does. The latter form can both serve as a sentential pro-form, as for instance in the \ili{Polish} 

\noindent\parbox{\linguexfootnotewidth}{\exg. {\ldots} \ ale \textbf{to} nie mo\.ze by\'c prawda\\
{} but it not can be truth\\
\strut  `\ldots \ but it cannot be true'

} and it is also syncretic with (what can be pre-theoretically described as) an invariant \isi{particle} present in a range of sentences including foci, topics, and clefts, as partially illustrated in:

\noindent\parbox{\linguexfootnotewidth}{\ex. \ili{Polish} \textit{to} in sentences with a focused object (\citealt[147]{Wiland2016})
\ag.[]\hspace{-22pt}\textbf{To} Mari\k{e} okradli jej s\k{a}siedzi\\
\hspace{-22pt}\textsc{prt} Mary.\textsc{acc.foc}  robbed her neighbors.\textsc{nom}\\
\hspace{-22pt}\strut `Mary's neighbors robbed her.'

\ex. \ili{Polish} \textit{to} in cleft sentences (\citealt[354]{Tajsner2008})
\ag.[]\hspace{-22pt}Marka \textbf{to} Ania spokta\l a w kinie\\
\hspace{-22pt}Marek.\textsc{acc.top} \textsc{prt} Ania.\textsc{nom} met in cinema\\
\hspace{-22pt}\strut `It was Marek that Ania met in the cinema.'  

} For analyses of clauses with the sentential \textit{to} in \ili{Polish} see for instance \citeauthor{Tajsner2008} (\citeyear{Tajsner2008,Tajsner2015,Tajsner2018}) and \cite{Mokrosz2014}; for a related discussion of the sentential \textit{to} in \ili{Czech} see \cite{Simik2009}.      
} %end of fn on invariant 'to'

\exg.
Opowiedzia\l {} ze szczeg\'o\l ami o \textbf{twoim} \textbf{problemie}, mimo \.ze mia\l {} zakaz nawet o \{ \textbf{nim} / {\textbf{tym} \}} wspomina\'c.\\
 told.\textsc{3sg}  with details about your problem.\textsc{loc.sg} despite \textsc{comp} had.\textsc{3sg} ban even about {} it {} \textsc{dem-loc.sg} mention.\textsc{inf}\\
\strut `He told about your problem with details, even though he had a ban on even mentioning \{ it / that \}.'\label{pro-to}

\noindent For this reason, it is more more plausible to go along with the idea that, apart from spatial deixis, \textit{to} contains a nominal rather than adjectival ingredient (though nothing in what follows is going to rely on that particular choice).\footnote{In other words, what needs to be accommodated in the representation of the demonstrative pronoun is the source of case, which deictic features Deix\textsubscript{n} in \ref{simplified:to} are not. In \ili{Polish} and \ili{Russian} this source of case can be attributed to the presence of either a nominal or an  adjectival stem, which is reflected by what is often described as nominal or adjectival case declensions (cf. \citealt[130--131, 146]{Nagorko1998}).
} %end of fn
\par
This nominal ingredient is responsible for the projection of a separate case \isi{fseq} on its top (marked below as K$_{1}$, a stand-in for neuter nominative singular), in agreement with \citeauthor{Caha2009}'s \citeyearpar{Caha2009} case representation discussed in \sectref{caseaba}. All these layers are merged in the one and only projection line, as in the structure with a bare Dem\textsubscript{indef} in \ref{Pol-and-Rus-a} and WhP in \ref{Pol-and-Rus-b}, a refined version of \ref{wh-dem}:


\begin{multicols}{2}
\ex.\label{Pol-and-Rus}
\a.\label{Pol-and-Rus-a}
\begin{forest}nice empty nodes, for tree={l sep=0.65em,l=0,calign angle=63}
 [K$_{1}$P [K$_{1}$]
 [\hspace{10pt}Dem\textsubscript{indef} 
 [Dem] [NP ]]]
 \end{forest}\\[1ex]
\b.\label{Pol-and-Rus-b}
\begin{forest}nice empty nodes, for tree={l sep=0.65em,l=0,calign angle=63}
 [K$_{1}$P [K$_{1}$]
 [WhP
 [Wh ] [\hspace{10pt}Dem\textsubscript{indef} 
 [Dem] [NP ]]]]]
 \end{forest}
 
 \end{multicols}

\vskip -0.25cm
\noindent To wrap it up, under the decomposition analysis of the demonstrative into three deictic \isi{feature}s detailed in \cite{Lander-Haegeman2016}, the \ili{Polish} and \ili{Russian} \textit{to} in \ref{Pol-and-Rus-a} realizes the following sequence:

\ex.
\begin{forest}nice empty nodes, for tree={l sep=0.5em,l=0,calign angle=63}
 [K$_{1}$P [K$_{1}$]
 [MedP [Deix$_{2}$]
 [ProxP 
 [Deix$_{1}$] [NP ]]]]
 \end{forest}

\noindent
Note, however, that while decomposing the Dem\textsubscript{indef} layer into separate features that describe the spatial deictic contrast enables us to better identify the \ili{Polish}/\ili{Russian} \textit{t-} as an exponent of the medial, our main point merely relies on the fact that the \textit{t-} is an exponent of a certain \isi{demonstrative} pronoun without a definiteness marker. For this reason, I will continue to represent such demonstratives in this chapter and onwards simply as ``Dem\textsubscript{indef} headed by Dem'' since the argument is not based on the degree of its internal decomposition.

\subsection{Lexicalization in Polish and in Russian}

Let us consider how the structures in \ref{Pol-and-Rus} are lexicalized in \ili{Polish}, a language with bi-morphemic forms for all four categories, as shown in \tabref{tab:Pol}. These forms reveal that \ili{Polish} has the following list of the lexical entries:\pagebreak

\ex. Lexical entries in \ili{Polish}\label{lex:Pol}
\a. [ Dem NP ] $\Leftrightarrow$ \textit{t}\label{lex:Pol:t}
\b. [ Rel [ Wh [ Dem NP ]]] $\Leftrightarrow$ \textit{c}\label{lex:c}
\c. [ Comp [ Rel [ Wh [ Dem NP ]]]] $\Leftrightarrow$ \textit{\.z}\label{lex:z}
\c. [ K$_{1}$ ] $\Leftrightarrow$ \textit{o}\label{lex:o}

\begin{table}[h]
\caption{Polish paradigm\label{tab:Pol}}
\begin{tabular}[h]{ l l l l l l }
\lsptoprule
\textsc{dem} 	& \textsc{comp} 	& \textsc{rel}  	& \textsc{wh}\\	
\midrule
t-o & \.z-e & c-o\cellcolor[gray]{0.9} & c-o\cellcolor[gray]{0.9}\\
\lspbottomrule
\end{tabular}
\end{table}


In \ili{Polish}, the \isi{spell-out} of the ``Wh\,$>$\,Dem\textsubscript{indef}'' subsequence involves a simple over-riding: the merger of the Wh \isi{feature} on top of Dem\textsubscript{indef} is spelled out by \textsc{stay}, the first step of the algorithm. Given the lexical entries in \ref{lex:Pol:t} and \ref{lex:c}, the spell-out of the WhP-layer over-rides \is{Cyclic Over-ride} the earlier spell-out of Dem\textsubscript{indef}, as in:\largerpage[2]

\ex.\label{sp:wh}\vspace*{-\baselineskip}
\begin{forest} nice empty nodes, for tree={l sep=0.65em,l=0,calign angle=63}
 [K$_{1}$P [K$_{1}$]
 [WhP
 [Wh ] [\hspace{15pt}Dem\textsubscript{indef} 
 [Dem] [NP ]]{\draw (.east) node[right]{$\Rightarrow$ \textit{t}}; }
 ]{\draw (.east) node[right]{$\Rightarrow$ \textit{c}}; }
 ]]
 \end{forest}
 
\noindent 
In turn, the spell-out of K$_{1}$ requires the evacuation movement of its complement, as in \ref{sp:K1}, in a typical way in which nominative is lexicalized in \ili{Slavic}, as illustrated on the example of \textit{win-o} `wine-\textsc{nom}' in \ref{mergerandspellout} in \sectref{sec:Starke2018}.\footnote{\label{FN:ze}The case suffix on the \isi{complementizer} \textit{\.z-e} does not require a separate lexical entry other than the one for \textit{-o} in \ref{lex:o}. As \cite{BaunazLander2018} point out, the suffix \textit{-o} /o/ shifts into \textit{-e} /e/ after a soft consonant \textit{\.z}- /ʒ/. 
}%end of FN

\ex.\label{sp:K1}
\begin{forest} nice empty nodes, for tree={l sep=0.65em,l=0,calign angle=63}
[~, s sep=35pt [WhP, l=1pt, name=tgt
 [Wh ] [\hspace{15pt}Dem\textsubscript{indef} 
 [Dem] [NP ]]{\draw (.east) node[right]{$\Rightarrow$ \textit{t}}; }
 ]{\draw (.east) node[right]{$\Rightarrow$ \textit{c}}; }
 [K$_{1}$P, l=1pt [K$_{1}$], s sep=-30pt [\dots, name=t ]]{\draw (.east) node[right]{$\Rightarrow$ \textit{o}}; }
 ]
\draw[dashed,->,>=stealth,overlay] (t) [in=-155,out=-125,looseness=1.75]  to (tgt);
\end{forest}

\noindent There is no need to postulate a second branch (e.g. the N triangle in \pref{BL:tree}) if  Dem\textsubscript{indef} is already part of Wh\,$>$\,Dem\textsubscript{indef}. With the lexical entries in \ref{lex:Pol}, the lexicalization of Rel and Comp layers takes place, again, by spelling out the one and only projection line:

\ex. Lexicalization of the sequence in \ili{Polish}\label{sp:Pol2}\\[0.5ex]
\begin{forest}nice empty nodes, for tree={l sep=0.65em,l=0,calign angle=63}
 [K$_{1}$P [K$_{1}$]
 [CompP [Comp]
 [RelP [Rel]
 [WhP
 [Wh ] [\hspace{15pt}Dem\textsubscript{indef} 
 [Dem] [NP ]]{\draw (.east) node[right]{$\Rightarrow$ \textit{t}}; }
 ]]{\draw (.east) node[right]{$\Rightarrow$ \textit{c}}; }
 ]{\draw (.east) node[right]{$\Rightarrow$ \textit{\.z}}; }
 ]
\end{forest}
  
\noindent Note that the hypothesis that there is a single underlying projection line for the sequence ``Comp\,$>$\,Rel\,$>$\,Wh\,$>$\,Dem\textsubscript{indef}'' does not exclude the possibility that it may have to be reshaped in order to facilitate spell-out. This is a natural consequence of the spell-out procedure but it does not equal the idea that a reshaped tree is base generated as anything more complex than a singleton sequence of heads.
\par
 As detailed in Chapter \ref{chapter:nanosyntax}, the essence of \citeauthor{Starke2018}'s   \citeyearpar{Starke2018} contribution is that the subderivation of the left branch takes place as a last resort operation which facilitates \isi{spell-out} only after \textsc{stay} and \textsc{move} (cyclic and snowballing movements) do not lead to lexical insertion. This is precisely the source of the difference between the pattern we see in \ili{Polish} and \ili{Russian} (and \ili{Serbo-Croatian}), as argued for in \citet{Wiland-PSiCL}. That is, while the shape of the lexical entries in \ili{Polish} allow the \isi{fseq} in \ref{sp:Pol2} to be spelled-out by \textsc{stay} (ignoring case), the shape of the lexical entry for the \ili{Russian} \textit{\v{c}}- as in \Next requires the formation of the left branch. 

\ex.\label{lex:Ru} Lexical entry in \ili{Russian}\\[0.25ex]
 [ Comp [ Rel [ Wh Dem ]]] $\Leftrightarrow$ \textit{\v{c}}

If the lexical entries for the demonstrative \textit{t-} and the neuter case suffix \textit{-o} are identical in \ili{Polish} and \ili{Russian}, then the  lexicalization of Wh, Rel, and Comp will require the formation of the left branch in \ili{Russian}, given the entry for \textit{\v{c}-} in \ref{lex:Ru}. In contrast to \ili{Polish}, only the bottom Dem\textsubscript{indef} of the \isi{fseq} in \ref{sp:Pol2} can be spelled out by \textsc{stay} (as \textit{t-}) and none of the available movement operations of the updated \isi{spell-out algorithm} (cyclic, snowballing, extraction) are able to reshape the tree in \ref{sp:Pol2} in such a way that it matches (the subset or the superset of) the entry for \textit{\v{c}-} in \ref{lex:Ru}, either. As discussed in \sectref{sec:Starke2018}, the final available option is to launch a subderivation by providing the \isi{feature} from the mainline, e.g. the Dem feature of Dem\textsubscript{indef}, as the basis for the merger of the Wh feature. Such a merger will result with a binary foot, as in \Next, and will require a separate lexical entry to be spelled out.

\ex. 
\begin{forest}nice empty nodes, for tree={l sep=0.75em,l=0,calign angle=63}
[WhP [Wh][Dem]]
\end{forest}

Upon the merger of this subderivation with Dem\textsubscript{indef}, the resulting structure comes out as a bi-morphemic \textit{\v{c}-t-} (ignoring, again, the neuter case suffix \textit{-o}):

\ex.
\begin{forest}nice empty nodes, for tree={l sep=0.75em,l=0,calign angle=63}
 [~, s sep=20pt
 [WhP [Wh][Dem]]{\draw (.east) node[right]{$\Rightarrow$ \textit{\v{c}}}; } 
 [\hspace{10pt}Dem\textsubscript{indef} 
 [Dem] [NP ]]{\draw (.east) node[right]{$\Rightarrow$ \textit{t}}; }
 ]]]]
\end{forest}

\makebox[\linewidth][s]{Subsequent mergers of features forming RelP and CompP will extend (what}\\
 comes out as) the left branch, yielding \ref{sp:Russian}.


\ex.\label{sp:Russian} Lexicalization of the sequence in \ili{Russian}\\[-1ex]
\begin{forest}nice empty nodes, for tree={l sep=0.75em,l=0,calign angle=63}
 [~ 
 [CompP [Comp][RelP [Rel][WhP [Wh][Dem]]]]{\draw (.east) node[right]{$\Rightarrow$ \textit{\v{c}}}; } 
 [\hspace{15pt}Dem\textsubscript{indef} 
 [Dem] [NP ]]{\draw (.east) node[right]{$\Rightarrow$ \textit{t}}; }
 ]]]]
\end{forest}

\noindent If this analysis is on the right track, then the contrast in the shapes of the lexical items in \ili{Polish} and \ili{Russian} directly implies that the \ili{Polish} pattern is more basic, in the sense that the  lexicalization of the same \isi{fseq} is achieved by \textsc{stay}, while its lexicalization in \ili{Russian} requires \textsc{subderive}, the last resort. We can, thus, conclude that the underlying fseq comprises the indefinite demonstrative at its bottom, as in \ref{comp>dem}. 

\ex.\label{comp>dem} 
Comp\,$>$\,Rel\,$>$\,Wh\,$>$\,Dem\textsubscript{indef}
 
The geometry of the tree in \ref{sp:Russian} resembles the structure for the \ili{Russian} \textit{\v{c}-t-} as in  \textit{\v{c}to} in \cite{BaunazLander2018}, where it is based on a complex underlying tree in \ref{BL:tree}. Note, however, that there are two essential differences between these two representations. The first one is that in Baunaz \& Lander's analysis the \ili{Russian} \textit{t-} is an invariant nominal core, a kind of base component, while the \textit{t-} in \ref{sp:Russian} is the medial demonstrative pronoun (modulo the case suffix). The second difference concerns the nature of both representations. In Baunaz and Lander's analysis, the bi-morphemic \textit{\v{c}-t-} realizes the nominal base and the prefix branch of complex representation in \ref{BL:tree}. In the alternative  in \ref{sp:Russian}, the bi-moprhemic \textit{\v{c}-t-} is created solely as a result of the \isi{spell-out algorithm}, hence, there is technically no base component or a pre-defined prefix branch; instead, the underlying representation is a simple projection line just like it is in \ili{Polish} (or any other language, for that matter).
\par At this point let us note that while the \textit{t-} stem of the inflected \isi{demonstrative} \textit{t-o} is retained in the \ili{Russian} Comp and nominative and accusative forms of the Wh and the Rel \textit{\v{c}to}, it disappears in non-nominative forms of the Wh and the Rel, as shown in \tabref{tab:ruskie}.

\begin{table}
\caption{Declension of the \ili{Russian} \textit{\v{c}to}}
\label{tab:ruskie} 
\begin{tabular}[t]{ l l l l l l }
\lsptoprule	
  \textsc{nom} & \v{c}-t-o\\
  \textsc{acc}  & \v{c}-t-o / \v{c}-evo (informal)\\
  \textsc{gen} & \v{c}-evo\\
  \textsc{dat} & \v{c}-emu\\
  \textsc{loc}  & \v{c}-om\\
  \textsc{inst} & \v{c}-em\\
  \lspbottomrule
\end{tabular}
\end{table}

\noindent The disappearing \textit{t-} stem is found in \ili{Slavic} beyond \ili{Russian} and \ili{Polish}, too, and targets also forms of person wh-pronoun `who'. For example, as noted in \cite{Wiland-PSiCL}, if we follow the logic of decomposing \textit{\v{c}to} into \textit{c-t-o} and analyze \textit{kto} `who-\textsc{nom}' as \textit{k-t-o}, the same form in \ili{Russian} and \ili{Polish}, \textit{t-} disappears in all other cases, as shown in \tabref{tab:kogo}.

\begin{table}
\caption{Declension of the \ili{Russian} and \ili{Polish} \textit{kto} `who'}
\label{tab:kogo} 
\begin{tabular}[t]{ l l l l l l }
\lsptoprule	
   			& \ili{Russian}			& \ili{Polish}\\\hline
  \textsc{nom} 	& k-t-o			& k-t-o\\
  \textsc{acc}  	& k-ovo			& k-ogo\\
  \textsc{gen} 	& k-ovo			& k-ogo\\
  \textsc{dat} 	& k-omu			& k-omu\\
  \textsc{loc}  	& k-om			& k-im\\
  \textsc{inst} 	& k-em			& k-im\\
  \lspbottomrule
\end{tabular}
\end{table}

If we consider the case hierarchy in \ref{case-fseq} in \sectref{caseaba},
the \textit{t-} stem in \isi{wh-pronoun}s disappears in cases that are all bigger than nominative in the complexity scale.
This suggests that the disappearing \textit{t-} is a result of spell-out of cases bigger than nominative (perhaps involving \isi{backtracking}). In the remainder of the chapter, I will restrict the discussion to the nominative form of \textit{\v{c}to} only, as it is the only attested form of the declarative complementizer, and  will not offer an analysis of the disappearing \textit{t-} in forms other than the nominative.
\par
The sequence in \ref{comp>dem} is enough to cover languages like \ili{Polish} or \ili{Russian}, but it needs to be updated with definite demonstratives in order to describe languages like \ili{English}. This issue essentially reduces to the question about the place of definiteness morphology among the other categories in \ref{comp>dem}.

\section{High definite demonstratives}\label{sec:hdd}

\noindent There are at least two scenarios to consider. The first one is a variant of \ref{comp>dem} in which definiteness (indicated below as Def) is projected as a separate category at the bottom of the sequence, as in:

\ex.\label{bigDem} Comp\,$>$\,Rel\,$>$\,Wh\,$>$\,Dem\textsubscript{def}\,$>$\,Def

Initially, this looks like an attractive option since not only does it suggest that definiteness applies directly to the nominal \isi{root}, as in \Next, but it also reflects the fact that definite markers can be contained in the structure of a \isi{demonstrative} pronoun (e.g. \ili{English} \textit{th-at} or \ili{Italian} \textit{quel-lo}).\largerpage[2]
 
 \ex.
\begin{forest} nice empty nodes, for tree={l sep=0.75em,l=0,calign angle=63}
 [\hspace{15pt}Dem\textsubscript{def} 
 [Dem] [DefP [Def][NP ]]
 ]]]
 \end{forest}

The idea that definiteness applies to the nominal \isi{root} also parallels with the situation observed with lexical nouns, as e.g. \textit{the car}, where the definite article can appear without demonstrative morphology.
\par
However, extending such a structure into WhP, RelP, and CompP leads to the \isi{*ABA} violation: if the \ili{English} definiteness marker \textit{th-} and the medial/distal \isi{demonstrative} marker -\textit{at} spell out such a structure, the demonstrative \textit{-at} will come out as the suffix, following the evacuation movement of DefP, as indicated in the following:

 \ex. \label{DEF-movement}
\begin{forest} nice empty nodes, for tree={l sep=0.75em,l=0,calign angle=63}
 [, s sep=30pt [DefP, name=tgt [Def][NP ]]{\draw (.east) node[right]{$\Rightarrow$ \textit{th}}; }
 [\hspace{20pt}Dem\textsubscript{def} 
 [Dem] [\dots, name=t]]{\draw (.east) node[right]{$\Rightarrow$ \textit{at}}; }
 ]
\draw[dashed,->,>=stealth,overlay] (t) [in=-150,out=-150,looseness=2]  to (tgt);
 \end{forest}\vspace*{2\baselineskip}

The structure obtained by the Def-movement in \ref{DEF-movement}  appears to give a desired result. However, if the remainder of the sequence is ``Comp\,$>$\,Rel\,$>$\,Wh'', then the addition of these layers will result in the \isi{*ABA} pattern by sandwiching the \textit{wh-} for Wh between a lower \textit{th-} for Def and a higher \textit{th-} for Rel and Comp (i.e. the *ABA-violating ``\textit{th}\textsubscript{Comp}\,$>$\,\textit{th}\textsubscript{Rel}\,$>$\,\textit{wh}\textsubscript{Wh}\,$>$\,\textit{at}\textsubscript{Dem}\,$>$\,\textit{th}\textsubscript{Def}'').
 \par
In the alternative scenario, definiteness applies to the entire \isi{fseq} with the nominal \isi{root} at its bottom, as indicated in the following:\largerpage[-1]

  
\ex.\label{updated-fseq} The updated singleton \isi{fseq}\\[1ex]
\begin{forest}nice empty nodes, for tree={l sep=0.65em,l=0,calign angle=63}
 [\hspace{10pt}Dem\textsubscript{def} 
 [Def]
 [CompP [Comp]
 [RelP [Rel]
 [WhP [Wh]
 [\hspace{10pt}Dem\textsubscript{indef} [Dem][NP]]]]]]
 \end{forest}
 
 This sequence differs from the one that applies to both \ili{Polish} and \ili{Russian} (cf. \pref{sp:Pol2}) only by the top layer and captures the fact that the deictic demonstrative is a stem for the formation of all higher categories.\footnote{This option, shown in \Next below without the intermediate Wh, Rel, Comp layers, is in essence compliant with \citet[\S2]{Leu2015}. 

\noindent\parbox{\linguexfootnotewidth}{\ex.
\begin{forest} nice empty nodes, for tree={l sep=0.65em,l=0,calign angle=63}
 [\hspace{10pt}Dem\textsubscript{def}, s sep=0pt
 [Def]
 [\hspace{15pt}Dem\textsubscript{indef} ]]]
 \end{forest}
 
} Leu's work makes a case for the architecture of the \ili{Germanic} definite \isi{demonstrative} which contains the definite article and a proper deictic element\,---\,an abstract \textsc{here/there} in \citeauthor{Leu2015}'s \citeyearpar[15]{Leu2015} analysis of \ili{German} \textit{der Tisch} `the table', as shown in:
 
\noindent\parbox{\linguexfootnotewidth}{\ex.
\begin{forest}nice empty nodes, for tree={l sep=0.65em,l=0,calign angle=63}
[DP [Dem [\textit{der} \textsc{there}, roof]][[D][NP [\textit{Tisch}, roof]]]]
 \end{forest}

}} %end of FN on Leu (2015)

Given the shape of the \ili{English} lexical items as in \Next, the spell-out of the updated fseq in \ili{English} requires the formation of the complex left branch, as shown in \ref{Eng:LB}.

\ex. Lexical entries in \ili{English} (2nd approximation, replaces \pref{Eng:1stapprox})\label{th+wh:2nd}
\a.\label{lex:Dem}[ Def [ Comp [ Rel [ Wh Dem ]]]] $\Leftrightarrow$ \textit{th}
\b.\label{lex:Wh} [ Wh Dem ] $\Leftrightarrow$ \textit{wh}
\c.\label{lex:at} [ Dem NP ]  $\Leftrightarrow$ \textit{at}


\ex.\label{Eng:LB} Lexicalization of the \ili{English} \textit{wh-at} and \textit{th-at}\\[-1ex]
\begin{forest}nice empty nodes, for tree={l sep=0.65em,l=0,calign angle=63}
 [~, s sep=10pt
 [\hspace{10pt}Dem\textsubscript{def} [Def][CompP [Comp][RelP [Rel]
 [WhP [Wh][Dem]]{\draw (.east) node[right]{$\Rightarrow$ \textit{wh}}; }
 ]]]{\draw (.east) node[right]{$\Rightarrow$ \textit{th}}; } 
 [Dem\textsubscript{indef} 
 [Dem] [NP ]]{\draw (.east) node[right]{$\Rightarrow$ \textit{at}}; }
 ]]]]
\end{forest}

Thus, with the addition of Def, the lexicalization of the updated fseq in \ref{updated-fseq} in \ili{English} mimics what we see in \ili{Russian} in \ref{sp:Russian}, modulo the Def added on top.
\par 
To sum up, defining the sequence as in \ref{updated-fseq} leads to the reordering in the \isi{paradigm}s of languages without definiteness marking, which should be represented as in \tabref{table3}.

\begin{table}
\caption{English via-\`a-vis Russian}
\label{table3}
\begin{tabular}[t]{ l l l l l l }
\lsptoprule
 & \textsc{dem}\textsubscript{def} & \textsc{comp} 	& \textsc{rel}  	& \textsc{wh} & \textsc{dem}\textsubscript{indef}\\	
 \midrule
\ili{English} & th-at\cellcolor[gray]{0.9} & th-at\cellcolor[gray]{0.9} & th-at\cellcolor[gray]{0.9} & wh-at & -at \\
\ili{Russian} & & \v{c}-to\cellcolor[gray]{0.9} & \v{c}-to\cellcolor[gray]{0.9} & \v{c}-to\cellcolor[gray]{0.9} & to\\
\lspbottomrule
\end{tabular}
\end{table}

\noindent The \textit{-at} morpheme in \textit{th-at} /ðæt/ and in \textit{wh-at} /wɑt/ has different exponents, even across the varieties of \ili{English} involving also /wɔt/ but not */wæt/. This contrasts with what we observe in \ili{Russian}, where \textit{to} is syncretic in all four forms. This fact does not seem to result in an ABA pattern \is{*ABA} in \tabref{table3} but\,---\,on the proviso that the contrast in the phonological shape of the stem \textit{-at} in \textit{th-at} and in \textit{wh-at} as /æt/ vs. /ɑt/ or /ɔt/ is not an instance of a purely phonologically conditioned \is{allomorphy} allomorphy\,---\,it may suggest that the syntactic size of stem in Wh, Rel, Comp, and Dem\textsubscript{indef} is not constant throughout the \ili{English} paradigm. That is, the \ili{English} /ɑt/ and /æt/ may reflect the subset-superset relation that is realized by different exponents, a plausible scenario given that the Dem\textsubscript{indef} stem is internally complex. I will return to the issue of the variable size of the bottom constituent in the next chapter on the example of the Latvian \textit{kas}, a syncretic form for pronominal `what' and `who'.\footnote{\label{FN:differentbottoms}The complexity of Dem\textsubscript{indef} concerns both the spatial deictic contrast as in Lander \& Haegeman's \citeyearpar{Lander-Haegeman2016} decomposition in \ref{LH:Dem} but also its (pro)nominal component, marked in \ref{updated-fseq} and elsewhere in this chapter as the NP constituent at the bottom of the \isi{fseq} in \ref{updated-fseq}. In \cite{Wiland-PSiCL} I have explored a possibility where the \ili{Russian} and \ili{Polish} NP \textit{t-} of the bi-morphemic \textit{t-o} spells out subsets of a nominal sequence specified for Thing and Person (in the sense of \citeauthor{Cysouw2004} \citeyear{Cysouw2004,Cysouw2005}), a scenario more transparently visible in the \ili{English} forms \textit{wh-at} and \textit{wh-o} rather than in the \ili{Russian} \textit{\v{c}-to} `what' and \textit{k-to} `who' with a syncretic stem \textit{to}. I will discuss the distinction between pronominal Person and Thing in wh-queries in Latvian in the next chapter.
} % end of FN



\section{Summary}

\noindent Cross-categorial \isi{syncretism}s with the declarative \isi{complementizer} discussed in \citeauthor{BaunazLander2017} (\citeyear{BaunazLander2017,BaunazLander2018,Baunaz-Lander-Glossa}) indicate that the \isi{wh-pronoun}, the \isi{relativizer}, the complementizer, and the definite demonstrative pronoun form an \isi{fseq}. Thus, morphological \isi{containment} of indefinite \isi{demonstrative} pronouns in the structure of the wh-pronoun, the relativizer, and the complementizer in languages like \ili{Russian} poses a problem for such an fseq in that it does not apply uniformly to languages with and without definiteness marking.
\par
This problem can be resolved by inserting indefinite demonstratives at the bottom of this fseq to the effect that the definite demonstrative is a category which syntactically ranges from the indefinite demonstrative, through Wh, Rel, Comp, and is closed up by a high Def. This result is possible to achieve if the underlying representation of these categories is simplified to a single projection line and its partition into multiple \isi{morpheme}s is solely a result of the \isi{spell-out} procedure, not the geometry of a tree in an underlying representation.
