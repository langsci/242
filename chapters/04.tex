\chapter{Deriving the verb stem alternation}\label{chapter:explaining}


\section{Introduction}

The domain which arguably exhibits the \isi{reduction} in the number of \isi{morpheme}s is the alternation between  \isi{semelfactive} and \isi{iterative} \isi{verb} stems found in \ili{Czech} and \ili{Polish}, which is illustrated in the following: 


\ex. \ili{Czech} \label{Cz:sem-act}
\ag. 
kop-n-ou-t -- kop-a-t\\
kick-\textsc{n-ou}\textsubscript{theme}-\textsc{inf} {} kick-\textsc{aj}\textsubscript{theme}-\textsc{inf}\\
\strut `give a kick' \hskip 1.4cm `be giving kicks'
\bg.
\v{s}t\v{e}k-n-ou-t -- \v{s}t\v{e}k-a-t\\
bark-\textsc{n-ou}\textsubscript{theme}-\textsc{inf} {} bark-\textsc{aj}\textsubscript{theme}-\textsc{inf}\\
`give a bark' \hskip 1.45cm `bark repeatedly'

\ex. \ili{Polish}\label{P:sem-act}
\ag. 
kop-n-\k{a}-\'c -- kop-a-\'c\\
kick-\textsc{n-ou}\textsubscript{theme}-\textsc{inf} {} kick-\textsc{aj}\textsubscript{theme}-\textsc{inf}\\
\strut `give a kick' \hskip 1.4cm `be giving kicks'
\bg. 
liz-n-\k{a}-\'c -- liz-a-\'c\\
lick-\textsc{n-ou}\textsubscript{theme}-\textsc{inf} {} lick-\textsc{aj}\textsubscript{theme}-\textsc{inf}\\
\strut `give a lick' \hskip 1.4cm `lick repeatedly'

The alternation involves a tri-morphemic  \isi{semelfactive} stem and a bi-morphemic \isi{iterative} stem. The semelfactive stem, which can be roughly defined as one-time event, comprises a \isi{root}, the \textit{-n} suffix (with the light \isi{verb} meaning Give \is{light Give}), and a \isi{thematic suffix} \textit{-ou} (realized as \textit{ou} in \ili{Czech} and as a nasalized vowel \textit{\k{a}} in \ili{Polish}). The corresponding bi-morphemic iterative stem, roughly defined as an event involving a repetition of a one-time event, comprises a root and the \isi{thematic suffix} \textit{-aj} (here realized simply as \textit{a} due to a rule in \ili{Slavic} phonology whereby a glide becomes truncated \is{glide truncation} before a consonant).
\par
The fact that the \isi{iterative} stem is morphologically less complex than a \isi{semelfactive} is paradoxical given the account of iteratives as more complex in syn-sem terms as than the second. If so, then the extension of structurally smaller \isi{semelfactive}s into bigger iteratives results in the \isi{reduction} in the amount of \isi{morpheme}s. In what follows, I explore the possibility to derive this  reduction by \isi{subextraction} and compare it with an alternative analysis based on \isi{backtracking}. 
\par
Let us begin with an overview of the structure of the \ili{Slavic} \isi{verb} stem and the properties of the  alternation. 

\section{Background: the verb stem in \ili{Czech} and \ili{Polish}}

\subsection{Verb stem morphology}

The morphological make-up of the \isi{verb} in \ili{Slavic} is to a large degree templatic, as shown below on the example of the \ili{Czech} verb \textit{d\v elat} `do' in \Next and the \ili{Polish} \isi{verb} \textit{zamyka\'c} `close' in \ref{zamykac}.

\ex.\label{verb}	
\textsc{(prefix) -- root -- theme -- participle -- agr} 
\ag.
u -- d\v el -- a -- l -- a (active: L-participle)\label{1a} \\
\textsc{pfv} -- do -- \textsc{aj} -- \textsc{part} -- \textsc{fem.sg} \\
\strut `(she) did' 
\bg. u -- d\v el -- \'a -- n -- o (passive: N/T-participle)\label{1b}\\
\textsc{pfv} -- do -- \textsc{aj} -- \textsc{part} -- \textsc{neu.sg} \\
\strut `(it was) done' 

\ex.\label{zamykac}
\textsc{(prefix) -- root -- theme -- participle -- agr}
\ag.
za -- myk -- a -- \l {} -- em (active: L-participle)\label{1a} \\
\textsc{pfv} -- close -- \textsc{aj} -- \textsc{part} -- \textsc{1sg.msc} \\
\strut `(I) closed' 
\bg. za -- myk -- a -- n -- y \hskip 0.15cm (passive: N/T-participle)\label{1b}\\
\textsc{pfv} -- close -- \textsc{aj} -- \textsc{part} -- \textsc{msc.sg} \\
\strut `(being) closed' 

The \isi{verb} structure comprises a \isi{root}, optionally preceded by lexical and/or aspectual prefix, which is followed by a \isi{thematic suffix} (the so-called theme vowel), the participle \isi{morpheme}
(L in active non-present tense, and N/T in passive), and the subject agreement suffix.\footnote{Such a  representation of the \ili{Slavic} \isi{verb} has its origin in Jakobson's \citeyearpar{Jakobson1948} analysis of the \ili{Russian} conjugation, which has opened up the possibility to provide a structural description of the \isi{verb} in all \ili{Slavic}. For some alternative ways of classifying \ili{Slavic} verbs into conjugation classes see e.g. \cite{Laskowski1975}, \cite{towjand}, \cite{czayk1988}, \cite{Jablonska2007}, and the references cited there.
}%end of fn
\par
Before we take a look at the list of theme vowels in the structure of the \ili{Czech} and \ili{Polish} \isi{verb}, a terminological distinction between \isi{root}s and stems should be made clear. Unless specified differently in a particular context, I will refer to the ``root'' as an item understood pre-theoretically as in the following:

\ex. A \isi{root} is an open class lexical item that can form \isi{verb}s, adjectives or nouns.

In line with this definition, a `verbal root' is an open class lexical item that forms \isi{verb}s, an `adjectival root' an open-class item that forms adjectives, and a `nominal root' an open class item that forms nouns. 
In turn, I will use the term `verb stem' in the way that is common in the literature on the \ili{Slavic} verb (and, in fact,  often used in the context of \isi{verb} morphology in general, too) as in the following:

\ex. A \isi{verb} stem is a (simplex or complex) morphological form that is subject to inflection. 

This definition implies that a \ili{Slavic} verb stem can in principle be morphologically more complex than a root, a situation that will be illustrated shortly. 


\subsection{Thematic suffixes}

The thematic affixes \is{thematic suffix} in \ili{Slavic} are verbalizers that come in between the \isi{root} and the inflectional suffix (see e.g. \citealt{Isacenko1962,Halle1963,Flier1972,Lightner1972} for \ili{Russian}; \citealt{towjand} and \citealt{Komarek2006} for \ili{Czech}; \citealt{Laskowski1975,GP1979,Rubach1984,czayk1988} and \citealt{Szpyra1989} for \ili{Polish}; \citealt[181--188]{Sven2004} for a comprehensive overview). The list of themes in \ili{Czech} and \ili{Polish} is given in Table \ref{tab:thvs}. Together with root they merge with, thematic affixes form \isi{verb} stems, which encode the verbal \isi{argument structure}. The verbalizing property of thematic affixes is clear as we do not find them in present day \ili{Czech} and \ili{Polish} nouns or adjectives. 

\begin{table}
\caption{Thematic affixes \is{thematic suffix} in \ili{Czech} and \ili{Polish}}
\label{tab:thvs}
\begin{tabular}{clll}
\lsptoprule		
thematic affix	& 	\ili{Czech}					&	\ili{Polish}				& gloss\\\hline	
$\emptyset$	&	n\'es-$\emptyset$-t	&	nie-$\emptyset$-\'s\'c	& carry\\
a			&	ps-\'a-t					&	pis-a-\'c				& write\\
\v{e}/e		&	vid-\v{e}-t					& 	widzi-e-\'c				& see \\
aj			&	klek-a-t					&	kl\k{e}k-a-\'c			& kneel\\
ej			& 	kamen-e-t					&	kamieni-e-\'c			& become stone\\
ov			&	kup-ov-a-t					&	kup-ow-a-\'c			& buy\\
i			&	pros-i-t					&	pros-i-\'c				& ask\\
nou/n\k{a}	&	kop-nou-t					&	kop-n\k{a}-\'c			& kick\\	
\lspbottomrule
\end{tabular}
\end{table}

\par
Whereas three \is{thematic suffix} theme vowels, the null theme, \textit{-a}, and \textit{-ov} produce a range of different aspectual and argument-structural \is{argument structure} classes of \isi{verb} stems, the other theme vowels contribute to the properties of \isi{verb} stems  in a more predictable way.\footnote{Following the tradition of \ili{Slavic} philology, largely shaped by the work done on \ili{Old Church Slavonic} and modern \ili{Russian}, \ili{Polish} \isi{verb} stems with the null theme vowel are sometimes referred to as consonantal stems rather than stems comprising a \isi{root} and a null theme vowel (e.g. \citealt{Rubach1984,czayk1988,Jablonska2007}). The nature of such stems, however, is orthogonal to the following discussion of  \isi{semelfactive}s.  
} %end of fn on C-stems
 For example, the null theme and the \textit{-a} theme build both \isi{activity} and process \isi{verb}s that belong to different argument-structural classes, e.g. the \ili{Czech} transitive activity verbs \textit{n\'es-$\emptyset$-t} `carry' or \textit{ps-\'a-t} `write', or the \ili{Polish} unaccusative \textit{pa\'s-$\emptyset$-\'c} `fall' or \textit{u-mier-a-\'c} `die'. 
\par
The same holds true about the \textit{-ov} theme, which also builds (broadly understood) \isi{activity} stems, but there is a caveat about its distribution. \is{thematic suffix} Namely, one characteristic property of the \textit{-a} theme is that it merges with verbal \isi{root}s. Maintaining the approach to \isi{spell-out} whereby every \isi{morpheme} is a lexical realization of a phrasal constituent in syntax, we can represent a verbal root simply as the VP in the structure of \textit{a}-stems as in the following:   

\ex. 
[[\textsubscript{VP} \isi{root} ] \textit{-a} ]

The term `verbal root', represented above as a morphological \isi{root} (a particular lexical item) with the VP status in syntax, is descriptively understood in this context simply as an open-class lexical item that forms \isi{verb}s but does not form adjectives or nouns.
For example, neither the \ili{Czech} root \textit{pis-} `write' nor the \ili{Polish} root \textit{mar-} `die' can form adjectival or nominal stems. These and other roots can form adjectival participles and nominalizations, e.g. the \ili{Czech} \textit{ps-a-n-\'y} `written' or the \ili{Polish} \textit{u-mier-a-nie} `dying'. These forms, however, are derived by suffixes that are all external to the \isi{verb} stem, as indicated in \ref{verb}--\ref{zamykac}. 
In order for nominal \isi{root}s to form a \isi{verb} stem with the \textit{-a} theme they must be extended by the \textit{-ov} suffix, which can be illustrated by the \ili{Polish} examples such as \textit{matk-a} `mother-\textsc{nom.fem}' -- \textit{matk-ow-a-\'c} `to mother someone', \textit{st\'o\l} {} `table.\textsc{nom.msc}' -- \textit{sto\l-ow-a-\'c} `to be eating out', \textit{panik-a} `panic-\textsc{nom.fem}' -- \textit{panik-ow-a-\'c} `to panic'. At the same time, the -\textit{a} theme does not form \isi{verb} stems by a direct merger with nominal roots, as in the unattested forms \textit{*matk-a-\'c}, \textit{*sto\l-a-\'c}, \textit{*panik-a-\'c}. \is{thematic suffix} This pattern is productive and holds in both \ili{Czech} and \ili{Polish} borrowings, as for instance \textit{forward} -- \textit{forward-ov-a-t} `to forward an email', \textit{skype} -- \textit{skyp-ov-a-t} `to skype', \textit{biwak} `bivouac.\textsc{nom.msc}' -- \textit{biwak-ow-a-\'c} `to bivouac', with bare nominal roots impossible to form \textit{-a}-stems, as shown by the unattested \textit{*forward-a-t}, \textit{*skyp-a-t}, or \textit{*biwak-a-\'c}. 
The resulting picture is that the merger of a nominal \isi{root} with the \textit{-ov} is a morphologically complex realization of the verbal root, as in \Next, which makes such a structure fit to merge with the \textit{-a} theme. \is{thematic suffix}

\ex. [[\textsubscript{VP} [\textsubscript{NP} \isi{root} ] \textit{-ov} ] \textit{-a} ]

Hence, what is traditionally described as the -\textit{ova} theme vowel in the literature on \ili{Slavic} comes out as a sequence of two separate suffixes, \textit{-ov} and \textit{-a}, whose distribution can be best understood when considered jointly with the categories of roots they merge with.\footnote{This is not an exhaustive description of the \textit{-ov} theme since it also can merge with a subset of adjectival roots. The \textit{-ov-a} \isi{verb} stems that are formed in this way are statives rather than activities, e.g. the \ili{Polish} \textit{chor-y} `sick-\textsc{adj.nom.msc}' -- \textit{chor-ow-a-\'c} `be sick'.
} %end of fn on ADJ+OVA
\par 
Unlike the null theme, the other thematic suffixes form \isi{verb} stems whose syn-sem properties can be predicted more accurately. \is{thematic suffix}
\par
 For instance, the \textit{-e} theme builds stative stems, e.g. the \ili{Czech} \textit{sed-\v{e}-t} `sit' \textit{bol-e-t} `hurt', or the \ili{Polish} \textit{le\.z-e-\'c} `lie (on a surface)', including what is sometimes classified as its subclass, namely verbs of perception and production of sounds, e.g. the \ili{Polish} \textit{s\l ysz-e-\'c} `hear', \textit{becz-e-\'c} `bleat', \textit{rycz-e-\'c} `roar', \textit{burcz-e-\'c} `growl', \textit{brz\k{e}cz-e-\'c} `buzz', or \textit{krzycz-e-\'c} `shout'. 
On top of that, \textit{-e} can also form \isi{activity} stems, e.g. the \ili{Czech} \textit{b\v{e}\v{z}-e-t} `run', \textit{let-\v{e}-t} `fly', \textit{s\'az-e-t} `plant'.
\par 
In turn, both \textit{-aj} and \textit{-i} themes build \isi{activity} \isi{verb}s. \is{thematic suffix} As stated earlier, the \textit{-aj} theme forms \isi{iterative}s, habituals and frequentives, while the \textit{-i} theme builds a fairly wide range of transitives, e.g. the \ili{Polish} \textit{pal-i-\'c} `burn, smoke', \textit{rob-i-\'c} `do', and reflexive \isi{verb}s like the \ili{Czech} \textit{modl-i-t se} `pray', among other \isi{activity} verbs with different argument-structural \is{argument structure} properties. Notably, however, the \textit{-i} theme is also a formative of `make X do Y' \isi{causative}s such as the \ili{Czech} \textit{posad-i-t} `make somebody sit', as in:

\ex. \ili{Czech}
\ag.[]\hspace{-22pt}Petr posad-i-l d\'it\v{e} na \v{z}idli.\\
\hspace{-22pt}Petr.\textsc{nom} sat-\textsc{i-part} baby.\textsc{neu} on chair\textsc{.loc}\\
\hspace{-22pt}\strut `Petr sat the baby on the chair.'

\noindent
The \textit{-ej} theme \is{thematic suffix} builds a subset of the so-called  degree achievements \isi{verb}s, an aspectual category that can be approximately described as a change of state that does not reach the endpoint (cf. \citealt{dowty79,hklevin1999,rothstein2004}), e.g. the \ili{Czech} \textit{\v{s}ediv-\v{e}-t} `become grey', \textit{kamen-\v{e}-t} `be turning into stone' or the \ili{Polish} \textit{\l ysi-e-\'c} `become bald', \textit{rdzewi-e-\'c} `get rusty'.\footnote{In the same way as in the case of the \textit{-aj} theme, the final glide in \textit{-ej} becomes deleted in front of a consonant of the following suffix due to the Glide Truncation rule given in \is{glide truncation} footnote \ref{fn:glidetruncation} in Chapter \ref{chapter:nanosyntax}. The \textit{-ej} suffix will surface in its entirety in non-past forms, e.g. the \ili{Polish} \textit{\l ysi-ej-emy} `we are getting bald' or in imperatives, e.g. \textit{\l ysi-ej} `get bald'. These are also examples of environments that allow us to morphologically distinguish \textit{-ej} from the theme \is{thematic suffix} vowel \textit{-e}, which forms statives, as discussed above. 
} %end of fn on glide truncation in -ej
\par
A large subset of degree achievement \isi{verb}s is also formed by the \textit{-n-ou} complex, which is analyzed in \cite{NU} as a sequence of two distinct \isi{morpheme}s only the second of which is a genuine theme vowel.\footnote{The description of the \isi{thematic suffix} as \textit{-ou} is based on \ili{Czech}. In \ili{Polish}, the theme vowel \textit{-ou} surfaces as a nasalized vowel \textit{\k{a}}, as in \textit{marz-n-\k{a}-\'c} `get cold'. Nasalization in \ili{Polish} has been analyzed as a consequence of  the presence of an underlying sequence of vowel and a nasal consonant in the coda, which suggests the \ili{Polish} exponent is \textit{-on} (cf. \citealt{Guss1980,Rubach1984}). \cite{czayk1988} suggests a different analysis involving a nasal diphthong comprising  a vowel and a nasal glide. Since this purely phonological difference is orthogonal to the syn-sem properties of the \isi{thematic suffix} in the \textit{-n-ou} sequence, I will continue to use the \textit{-ou} notation in reference to both \ili{Czech} and \ili{Polish}. 
} %end of fn
To a large extent, the list of \isi{root}s forming  degree achievements \textit{-n-ou} stems is common to \ili{Czech} and \ili{Polish}, e.g. \textit{bled-n-ou-t}/\textit{bled-n-\k{a}-\'c} `become pale', \textit{hluch-n-ou-t}/\textit{g\l uch-n-\k{a}-\'c} `get deaf', \textit{ho\v{r}k-n-ou-t}/\textit{gorzk-n-\k{a}-\'c} `get bitter', \textit{m\v{e}k-n-ou-t}/\textit{mi\k{e}k-n-\k{a}-\'c} `soften', \textit{vad-n-ou-t}/\textit{wi\k{e}d-n-\k{a}-\'c} `wither', \textit{mok-n-ou-t}/\textit{mok-n-\k{a}-\'c} `get wet', 
\textit{hub-n-ou-t}/\textit{chud-n-\k{a}-\'c} `lose weight, get thinner', to name a few. Nevertheless, certain roots that form degree achievement \textit{-n-ou} stems in \ili{Czech} form degree achievement \textit{-ej} stems in \ili{Polish}, e.g. \textit{hloup-n-ou-t} vs.\ \textit{g\l upi-e-\'c} `get stupid', \textit{hrub-n-ou-t} vs.\ \textit{grubi-e-\'c} `get fat', \textit{hloup-n-ou-t} vs.\ \textit{g\l upi-e-\'c} `get stupid', \textit{rud-n-ou-t} vs.\ \textit{rudzi-e-\'c} `redden'.
\par
Importantly, the \textit{-n-ou} sequence forms also  \isi{semelfactive}s, the category of \isi{verb}s that can be approximately described as single-stage events. The list of roots forming semelfactive \textit{-n-ou} stems also largely overlaps in \ili{Czech} and \ili{Polish}, e.g. \textit{kop-n-ou-t}/\textit{kop-n-\k{a}-\'c} `give a kick', \textit{kous-n-ou-t}/\textit{k\k{a}s-n-\k{a}-\'c} `give a bite', \textit{\v{s}t\v{e}k-n-ou-t}/\textit{szczek-n-\k{a}-\'c} `bark once', \textit{dotk-n-ou-t}/\textit{dotk-n-\k{a}-\'c} `give a touch', \textit{couv-n-ou-t}/\textit{cof-n-\k{a}-\'c} `mo-ve back once', \textit{mrk-n-ou-t}/\textit{mrug-n-\k{a}-\'c} `wink once'.
Despite the fact that the surface morphological forms of  degree achievement and \isi{semelfactive} \isi{verb}s are identical, the internal structures of the \isi{morpheme}s they are made of are different.
In the following section, I outline the description and analysis of these two \isi{verb} stems given \cite{NU}, which will serve as a starting point for the discussion of the  semelfactive-\isi{iterative alternation}, which involves the \isi{reduction} in the number of morphemes.

\section{Degree achievements vs.\ semelfactives in \cite{NU}}\label{sec:DAS}

The major idea of \cite{NU} is that while both  degree achievements and  \isi{semelfactive}s comprise the \isi{root} and the \textit{-n-ou} sequence, all three \isi{morpheme}s exhibit different syn-sem properties in these categories. 

\subsection{Adjectival vs.\ nominal \isi{root}s}\label{sec:roots}

The first contrast between these two \isi{verb} classes targets the lexical category of the root. The root in degree achievement stems is  adjectival  (an adjective modulo the case suffix \textit{-\'y}) as for instance in the \ili{Czech} \textit{bled-\'y} `pale' -- \textit{bled-n-ou-t} `get pale' (glossed in \Next), \textit{hluch-\'y} `deaf' -- \textit{hluch-n-ou-t} `get deaf',  \textit{ho\v{r}k-\'y} `bitter' -- \textit{ho\v{r}k-n-ou-t} `get bitter', or the \ili{Polish} \textit{blad-y} `pale' -- \textit{bled-n-\k{a}-\'c} `get pale', \textit{chud-y} `thin' -- \textit{chud-n\k{a}-\'c} `lose weight, get thinner'. 

\ex. Degree achievement (\ili{Czech})\label{bled}
\ag.[]\hspace{-22pt}bled -\textbf{n} -\textbf{ou} -t \\
\hspace{-22pt}pale\textsubscript{Adj} {\color{white}-}\textsc{get} {\color{white}-}\textsc{ou}\textsubscript{theme} 
{\color{white}-}\textsc{inf}\\
\hspace{-22pt}\strut `get pale'

\noindent
In turn, the \isi{semelfactive} stems are all based on a nominal root (a noun modulo the case suffix), e.g. the \ili{Czech} \textit{kop} `kick' -- \textit{kop-n-ou-t} `give a kick' (glossed in \Next), \textit{p\'isk} `a whistle' -- \textit{p\'isk-n-ou-t} `whistle once', \textit{vzlyk} `a sob' -- \textit{vzlyk-n-ou-t} `give a sob', or the \ili{Polish} \textit{pisk} `a squeak' -- \textit{pisk-n-\k{a}-\'c} `give a squeak', \textit{krzyk} `a scream' -- \textit{krzyk-n-\k{a}-\'c} `scream once', \textit{dotyk} `a touch' -- \textit{dotk-n-\k{a}-\'c} `touch once', etc. 

\ex. Semelfactive (\ili{Czech})\label{gloss:sem}
\ag.[]\hspace{-22pt}kop -\textbf{n} -\textbf{ou} -t \\
\hspace{-22pt}kick\textsubscript{N} {\color{white}-}\textsc{give} {\color{white}-}\textsc{ou}\textsubscript{theme} 
{\color{white}-}\textsc{inf}\\
\hspace{-22pt}\strut `(give a) kick'

The formation of  \isi{semelfactive} \textit{-n-ou} stems applies also to a subset of borrowed nominal roots, e.g. the  \ili{Czech} \textit{klik} `a click' -- \textit{klik-n-ou-t} `to click once'.
\par
There is a few important remarks that need to be made about the \textit{-n-ou} \isi{semelfactive}s. First, only a subset of \ili{Czech} and \ili{Polish} nominal \isi{root}s form such stems. For example, roots of such \ili{Polish} nouns as \textit{matk-a} `mother-\textsc{fem.nom}', \textit{st\'o\l} `table.\textsc{msc.nom}', among many others, will not form \textit{-n-ou}  semelfactives, i.e. \textit{*matk-n-\k{a}-\'c}, \textit{*sto\l-n-\k{a}-\'c} (these particular roots will forms \textit{-ov-a} activities \textit{matk-ov-a-\'c}, \textit{sto\l-ov-a-\'c}, as discussed in the previous section).
\par
 Second, some other genuine \textit{-n-ou} \isi{semelfactive}s, such as for instance the \ili{Czech} \textit{mrk-n-ou-t} `give a wink' or the \ili{Polish} \textit{pac-n-\k{a}-\'c} `give a smack' or \textit{mach-n-\k{a}-\'c} `wave once', do not have a simple noun formed only from the corresponding \isi{root} with an added case suffix, i.e. the unattested *\textit{mrk}, *\textit{pac}, *\textit{mach}.\footnote{Again, let us disregard nominalizations (the attested \textit{mrknut\'i} `winking', \textit{pacanie} `smacking', \textit{machanie} `waving') as even adjectival roots, like in the \ili{Polish} \textit{blad-y} `pale-\textsc{adj.nom.msc}', can form nominalizations, e.g. \textit{bledn\k{e}cie} `turning pale'.
 } %end of fn on nominalizations
The fact that the \textit{-n-ou} stems based on nominal roots may not have a corresponding noun is not limited to \isi{semelfactive}s since examples of  degree achievement \isi{verb}s that do not have a corresponding adjective are also attested. For instance, the \ili{Czech} degree achievement \isi{verb} \textit{plih-n-ou-t} `get limp' or the \ili{Polish} \textit{wi\k{e}d-n-\k{a}-\'c} `wither' do not have corresponding adjectives \textit{*plih-\'y} `limp-\textsc{adj.msc.nom}', \textit{*wi\k{e}d-y} `wither-\textsc{adj.msc.nom}'. However, when prefixed, these \isi{root}s still can still form adjectival L-participles, as in:

\ex.
\ag.  
z-plih-l-\'y\\
from-limp-\textsc{part}-\textsc{adj.msc.nom}\\
\strut `limp'
\bg.
z-wi\k{e}d-\l-y\\
from-wither-\textsc{part}-\textsc{adj.msc.nom}\\
\strut `withered'

\noindent
This contrast regarding the ability of nominal roots to form  \isi{semelfactive}s indicates that there exists a syntactically sensitive typology of nominal roots which singles out eventive and countable nouns as candidates for the formation of \isi{semelfactive} stems. Importantly, `eventive' and `countable' appear to be necessary but not sufficient \isi{feature}s of nominal \isi{root}s to qualify them as bases for the formation of \isi{semelfactive} \textit{-n-ou} stem. For instance, the \ili{Polish} \textit{op\'or} `resistance' or \textit{skarga} `complaint' do not form such stems (\textit{*opor-n-\k{a}-\'c}, \textit{*skarg-n-\k{a}-\'c}) but both can form semelfactives in different ways. The first one forms a periphrastic semelfactive with the \isi{verb} \textit{da\'c} `give' as in:

\exg. 
da\'c  op\'or w\l adzy\\
give.\textsc{inf}  resistance.\textsc{nom} authority.\textsc{dat}\\
\strut `to give resistance to the authority'

The second one can merge with the \isi{activity} theme \textit{-i} and with the perfectivizing  prefix \textit{za-}, as in \Next, which results in the formation of what \cite[116]{Bacz2012} describes as an inchoative \isi{semelfactive}, the one that marks the beginning of a new event or state. For the sake of explicitness, let us follow \citet[\S6.5]{Klein1994} and define perfectivity construed by prefixation with \textit{za-} as location of the run time of the event denoted by the predicate within the time interval.\footnote{See also \cite{DickeyJanda2009} for construing semelfactivity with perfctivizing prefixes in \ili{Russian}, the point of departure in \citeauthor{Bacz2012}'s \citeyearpar{Bacz2012} analysis of \isi{semelfactive}s  derived by prefixation in \ili{Polish}. For a related discussion concerning perfectivization by prefixation in \ili{Polish} see also \cite{Grzegorczykowa1997} and \citet[187--189]{Willim2006}. For a related discussion of the interplay of perfectivizing function of verbal prefixes and theme vowels \is{thematic suffix} see \citeauthor{Jablonska2004} (\citeyear{Jablonska2004,Jablonska2007}).
}%end of fn on perfectivizing prefixes in \ili{Slavic}

\exg. 
za-skar\.z-y-\'c decyzj\k{e}\\
\textsc{pfv}-complaint-\textsc{i}\textsubscript{theme}-\textsc{inf} decision.\textsc{acc}\\
\strut `to file a complaint against a decision'\label{zaskar}

\noindent
Let us point out that the situation where a subset of nominal roots does not form  \isi{semelfactive} \textit{-n-ou} stems does not have a bearing on the descriptive generalization that such stems are exclusively formed with nominal roots (in the same way as the situation where only a subset of adjectival roots form  degree achievement \textit{-n-ou} stems does not have a bearing on the generalization that such stems are only based on adjectival roots).
This is also reflected by the fact that there exist a small group of \textit{-n-ou} stems that are formed on what can be classified as verbal roots, in the sense that they only form \isi{verb} stems rather than nouns or adjectives (other than nominalizations or adjectival participles), such as e.g. the \ili{Czech} \textit{ply-n-ou-t} `flow, pass', \textit{vi-n-ou-t} `wind, wrap', \textit{\v{z}-n-ou-t} `mow, cut', \textit{tisk-n-ou-t} `print', or the \ili{Polish} \textit{p\l y-n-\k{a}-\'c} `swim',  \textit{ci\k{a}g-n-\k{a}-\'c} `drag, pull', or \textit{p\l o-n-\k{a}-\'c} `burn'. The verbal status of such roots is also reflected by their ability to merge with typically verbal prefixes such as the completive \textit{prze-}, as in the \ili{Polish} \textit{prze-p\l yn\k{a}\'c} lit. `complete a certain distance swimming' or the perfective \textit{za-}, as in the \ili{Czech} \textit{za-vinout} `swaddle', or the \ili{Polish} \textit{za-ci\k{a}gn\k{a}\'c} `pull onto', \textit{za-p\l on\k{a}\'c} `inflame' (cf. also \ref{zaskar}, where \textit{za-} merges with the verbal \textit{i}-stem rather than with a nominal root as in the unattested \textit{*za-skarga}). All these stems that are based on verbal roots are activities rather than \isi{semelfactive}s or degree achievements, as predicted by the generalization about nominal and adjectival status of roots in the \textit{-n-ou} stems.  

\subsection{Get vs. Give}

The \is{light Get} difference \is{light Give} in the lexical category of \isi{root}s the  degree achievement and  \isi{semelfactive} stems are based on carries over to the readings of these categories. The reading of the degree achievements is described in \cite{NU} as the light \isi{verb} Get applied to the property denoted by the adjectival root, which makes these categories essentially equivalent to \ili{English} analytic degree achievements such as \textit{get pale} or \textit{get dark} (a subset of which also have synthetic variants, e.g. \textit{darken}, \textit{redden}, making it even more descriptively close to the ones in \ili{Czech} and \ili{Polish}). 
\par
In turn, the reading of the \textit{-n-ou} \isi{semelfactive}s is described as the light \isi{verb} Give applied to the (caseless) noun, a fairly close equivalent of \ili{English} analytic semelfactives such as \textit{give a kick}, \textit{give a shout}, etc. The source of the light \isi{verb} semantics that applies to the roots in both kinds of stems is argued there to be the \textit{-n} \isi{morpheme}, which leaves \textit{-ou} to be a verbalizer, just like the other theme vowels are.
\par Even under the analysis of \textit{-ou} as a verbalizing theme vowel that turns the `Adj-root $+$ Get' \is{light Get} and the `N-root $+$ Give' complexes into, respectively,  degree achievement and \isi{semelfactive} \isi{verb} stems, the \textit{-ou} theme is not identical in both kinds of stems, either. This is due to the generalization inferred from a corpus study on \ili{Czech} and \ili{Polish} reported in \cite{NU} which states that degree achievement \textit{-n-ou} verbs are all unaccusative, while semelfactive \textit{-n-ou} verbs are either transitive/accusative or unergative.\footnote{\label{fn:L}The reported diagnostic for distinguishing between unaccusatives and unergatives is the formation of adjectival passive participles, arguably the only reliable test for unaccusativity that can be applied to both \ili{Czech} and \ili{Polish}. Unaccusative \isi{verb}s can form adjectival L-participles, while unergative and transitive verbs can form only N- or T-participles (cf. Cetnarowska \citeyear{Cetnarowska2002b}, \citeyear{Cetnarowska2002a}).  For instance, unaccusatives like \textit{vlhnout} `get wet' (Cz) or \textit{g\l uchn\k{a}\'c} `get deaf' (Pol) can form L-based adjectival participles \textit{z-vlh-l-\'y} `wet' or \textit{o-g\l uch-\l-y} `deaf', while unergative verbs like \textit{dupnout} `stamp' (Cz) or \textit{cofn\k{a}\'c} `move back (once)' (Pol) cannot: \textit{*dup-l-\'y, *cof-\l-y}. For an account of this contrast see \cite{LTN}.
} %fn on L-participles
 Thus, under the assumption that argument-stuctural properties are associated with the verbal structure, this contrast is realized by the \isi{thematic suffix} \textit{-ou}. This is not to say that theme vowels, including \textit{-ou}, are solely responsible for encoding the argument-structural properties of \isi{verb} stems. As written above, \isi{argument structure} is a property of the stem in the sense that it depends on the combination of a theme vowel and a \isi{root}. However, identifying different lexical categories of roots in different classes of stems opens up the possibility to understand the nature of the association between roots and theme vowels from the perspective of the argument structure in a more transparent way, the line of inquiry opened up by \cite{Jablonska2007}.
\par The description of the syn-sem properties of both kinds of stems are summarized in  \tabref{tab:summary}.

\begin{table}
\caption{Properties of degree achievement and  \isi{semelfactive} \textit{-n-ou} stems in \ili{Czech} and \ili{Polish} in \cite{NU}}
\label{tab:summary}
\begin{tabular}{ l  c  c  c }
\lsptoprule		
				& \textsc{root} 	& \textsc{light verb reading}	& \textsc{argument structure}\\\hline
deg.\ achievement: 	& Adj	& Get  				& unaccusative\\
 semelfactive: 		& N   	& Give 				& accusative, unergative\\
\lspbottomrule
\end{tabular}
\end{table}



The fact that with adjectival \isi{root}s the \textit{-n} suffix contributes the Get-reading \is{light Get} and with nominal roots it contributes the Give-reading \is{light Give} as well as the fact that the \textit{-ou} theme is present in unaccusative, transitives, and unergative \textit{-n-ou} stems is analyzed as instances of \isi{syncretism}.

\subsection{Light verb theory of \textit{-n}}

More precisely, the analysis  of the syn-sem structure of the \textit{-n} affix in \ili{Czech} and \ili{Polish} follows the decomposition of the \ili{English} lexical verb \textit{give} into the sequence of light \isi{verb}s involving `Give\,$>$\,Get' argued for in \cite{Richards2001}.
\par
\cite{Richards2001} considers \ili{English} idioms which include the lexical \textit{give}, like in \Next, and shows that in such idioms the idiomatic part is smaller than \textit{give DP}.

\ex. 
\a. The Count gives Mary the creeps.
\b. Mary gave John the sack.
\c. Mary gave Susan the boot.

Richards observes that the idiom is preserved with the lexical \isi{verb} \textit{get}, as in:

\ex. 
\a. Mary got the creeps.
\b. John got the sack.
\c. Susan got the boot.

This leads to a conclusion where the lexical structure of Get \is{light Get} is a subset of Give. \is{light Give}
\par 
Note also that \textit{give}-idioms are broken with the \textit{to}-dative variant:

\ex.
\a. *The Count gives the creeps to Mary.
\b. *Mary gave the sack to John.
\c. *Mary gave the boot to Susan.

\cite{Richards2001} takes this fact to indicate that double object constructions do not comprise a separate possessive functor (the abstract \isi{verb} Have) and instead, the possessive is an integral component of a ditransitive \textit{get}. As pointed out in \citet[\S4.2.3]{NU}, the \isi{containment} structure of the light `Give\,$>$\,Get' is not restricted only to the change-of-possession relation and is retained also with the change-of-state Get. \is{light Get} We can see this on the example of the idioms that are preserved with the lexical \isi{verb} \textit{get}, as in:

\ex.
\a. Mary got sacked.
\b. Mary got booted.
\c. Mary got evil eyed (by John).

This fact is taken to indicate that the core component of Get-readings is the change itself: change-of-possession in the case of the \ili{English} lexical \isi{verb}s \textit{get}, \textit{give} and change-of-state in the case of the lexical \textit{get} but not \textit{give}. This makes the correct prediction about the status of the Get-readings \is{light Get} in \ili{Czech} and \ili{Polish}  degree achievements, which denote change-of-state, not change-of-possession. 
\par
Since we find both Get- and \is{light Give} Give-readings \is{light Get} in the combinations of roots with the \textit{-n} suffix, this is taken to indicate that the light \isi{verb} structure is realized synthetically in \ili{Czech} and \ili{Polish} by the \textit{-n} morpheme, whose lexical entry can be minimally described as in:\footnote{Minimally, since in the few \isi{activity} \textit{-n-ou} stems listed above which are based on verbal roots, such as \textit{ply-n-ou-t} `swim' (Cz), \textit{vi-n-ou-t} `wind, wrap' (Cz), \textit{ci\k{a}g-n-\k{a}-\'c} `drag, pull' (Pol), etc., we do not have the \isi{light Get}- or Give-reading yet we do have the \textit{-n} suffix. Unless verbal roots trigger semantic neutralization of Get \is{light Get} and Give, a scenario I do not find immediate evidence for, this fact suggests that verbal roots such as \textit{ply-}, \textit{vi-}, \textit{ci\k{a}g-}, etc. form activity \textit{-n-ou} stems with the \textit{-n} suffix whose superstructure syntactically contains the [\,Give\,[\,Get\,]] structure given in \ref{lex:n}. The exhaustive description of the \textit{-n} superstructure, however, will not have a bearing on the following analysis of the \isi{iterative alternation}. \is{iterative}
} %end of fn on lexical entry for -n

\ex.\label{lex:n} Lexical entry for the light \textit{-n} in \ili{Czech} and \ili{Polish}\\[0.5ex]
[ Give [ Get ]]  $\Leftrightarrow$ \textit{n}

The Get-subset \is{light Get} of the structure realized by the \textit{-n} \isi{morpheme} is present in degree achievements, as illustrated on the example of \textit{bled-n-ou-t} `get pale', where it applies to the adjectival \isi{root}, as shown in \Next. More precisely, the change that is the core component of the Get-reading applies to the state denoted by the adjectival root, resulting in the perceived change-of-state.

\ex. 
\begin{forest}nice empty nodes, for tree={l sep=0.7em,l=0,calign angle=63}
 [GetP [Get][AdjP [\textit{bled}\\`pale', roof]]] 
 \end{forest} 

\vskip 0.25cm
As shown in \Next, following the \isi{spell-out} motivated movement of the root node, GetP becomes lexicalized as \textit{-n} on the strength of the \isi{Superset Principle} and surfaces as the suffix. 

\ex. Partial spell-out of a  degree achievement stem \textit{bled-n} `get pale'\label{bled-n}\\[0.5ex]
\begin{forest}nice empty nodes, for tree={l sep=0.7em,l=0,calign angle=63}
[GetP, s sep=13pt [AdjP, name=tgt [\textit{bled}\\`pale', roof]][GetP [Get][\ldots, name=t]
]{\draw (.east) node[right]{$\Rightarrow$ \textit{n}}; }
]
 \draw[dashed,->,>=stealth] (t) to[out=south west,in=south west,looseness=2.5] (tgt);
 \end{forest} 

\vskip -0.25cm
In turn, the superset of \isi{feature}s listed in the lexical entry for \textit{-ou} in \ref{lex:n} is present in \isi{semelfactive}s,  the categories construed by the merger of the \is{light Give} with the nominal root, as illustrated on the example of the \ili{Czech} \textit{kop-n-ou-t} `give a kick' in \Next. Unlike in degree achievements where the \isi{light Get} applies to a state denoted by the adjectival \isi{root}, in semelfactives, Get applies to an object of possession, which is denoted by the nominal root, a structure that projects into the GiveP after subsequent merger of the \is{light Give} \isi{feature} (see the discussion in \citealt[\S4.2.3--4.3]{NU}).\footnote{Let us take note of the fact that the feature Give serves in the structure in \ref{tree-GiveP} as a stand-in for a semantic \isi{feature} that extends the GetP subset into  GiveP. If we follow \citeauthor{dowty79}'s \citeyearpar{dowty79} description of the \ili{English} lexical \textit{give} as [ Cause [ Become [ Have ]]], our Give feature will correspond to a functor that introduces causation \is{causative} to a change-of-possession constituent GetP construed by the merger of Get \is{light Get} and a nominal \isi{root}, a feasible scenario which due to the purposes of this chapter I will not explore here further.
} %end of fn on Dowty 1979

\ex.\label{tree-GiveP} 
\begin{forest}nice empty nodes, for tree={l sep=0.7em,l=0,calign angle=63}
 [GiveP [Give][GetP [Get][NP [\textit{kop}\\`kick', roof]]]]
 \end{forest} 

As shown in \Next, the \isi{spell-out} of GiveP takes place following two movements, the complement movement and the spec-to-spec movement at the next cycle, to the effect that \textit{-n} comes out, again, as the suffix on the nominal root.


\ex. Partial \isi{spell-out} of a \isi{semelfactive} stem \textit{kop-n} `give a kick'\label{so:kop-n}\\[0.5ex]
\begin{forest}nice empty nodes, for tree={l sep=0.7em,l=0,calign angle=63}
[GiveP, s sep=13pt [NP, name=NP [\textit{kop}\\`kick', roof]]
[GiveP [Give][GetP [\ldots, name=T] [GetP [Get][\ldots, name=t]]]
]{\draw (.east) node[right]{$\Rightarrow$ \textit{n}}; }
]
 \draw[dashed,->,>=stealth] (t) to[out=south west,in=south west,looseness=1.75] (T);
 \draw[dashed,->,>=stealth] (T) to[out=south west,in=south west,looseness=2] (NP);
 \end{forest} 

\noindent Let us also point out that the association of the \ili{Czech}/\ili{Polish} light \textit{-n} morpheme with the \ili{English} \textit{give} and \textit{get} is based not only on the proximity of the readings but also on valency identity between the synthetic forms of both kinds of stems in \ili{Slavic} and the forms attested in \ili{English}.

\subsection{\textit{-Ou} as layers of the VP structure}

These argument-structural \is{argument structure} correlations are easily observed between the \ili{English} periphrastic degree achievements like e.g. \textit{get dumber}, \textit{get soft}, \textit{get blind}, etc., which correspond to the \ili{Czech}/\ili{Polish}  synthetic unaccusative `Adj-root \textit{-n}' structures, as for instance in:

\ex. \ili{Czech}
\ag.[]\hspace{-22pt}Petr hloup-n-u-l.\\
\hspace{-22pt}Petr.\textsc{nom} stupid-\textsc{get}-\textsc{ou}-\textsc{part.msc.sg}\\
\hspace{-22pt}\strut `Petr was getting more and more stupid.'

\ex. \ili{Polish}
\ag.[]\hspace{-22pt}Kartofle mi\k{e}k-n-\k{a}-$\emptyset$ podczas gotowania.\\
\hspace{-22pt}potatoes-{\sc nom} soft-\textsc{get}-\textsc{ou}-\textsc{pres.3pl} during cooking\\
\hspace{-22pt}\strut `Potatoes soften during cooking.'

\noindent
Likewise, the \ili{English} \isi{causative}s with the lexical \textit{give} correspond to the causative `N-root \textit{-n}' structures. The second is particularly transparent in the narrow subset of \ili{Slavic} periphrastic \isi{semelfactive}s which feature the lexical \isi{verb} \textit{da\'c} `give' followed by an accusative direct object as for instance in \ref{a}, a close equivalent of the synthetic \textit{-n-ou} semelfactive in \ref{b}.


\ex. \ili{Polish}\label{dackopa}
\ag. Jan da\l {} kop-a Karol-owi.\label{a}\\
Jan.\textsc{nom} gave  kick-\textsc{acc} Karol-\textsc{dat}\\
\bg. Jan kop-n-\k{a}-\l {} Karol-a.\label{b}\\
Jan.\textsc{nom} kop-\textsc{give-ou-part} Karol-\textsc{acc}\\
\strut `Jan gave Karol a kick.'\label{Karol}

Of course, \isi{semelfactive}s that do not have periphrastic variants like \textit{kopn\k{a}\'c}/\textit{da\'c kopa} in \ref{dackopa} can be transitive/accusative, too, e.g. \textit{bod-n-ou-t} `stab' in \ref{budnul} or even  double transitive, e.g. \textit{sk\v{r}\'ip-n-ou-t} `squeeze' in \ref{skrip}:

\ex. \ili{Czech} (\citet[ex. 82]{NU})
\ag.
Petr bod-n-u-l Karl-a. \\
Petr\textsc{.nom} stab-\textsc{give}-\textsc{ou}-\textsc{part.msc.sg} Karl-\textsc{acc}\\
\strut `Petr stabbed Karel (once).'\label{budnul}
\bg. 
Karel sk\v r\'ip-n-u-l Petr-ovi prst do dve\v{r}-\'i. \\
Karel-{\sc nom} squeeze-\textsc{give}-\textsc{ou}-\textsc{part.msc.sg} Petr-\textsc{dat} finger.\textsc{acc} into door-\textsc{gen} \\ 
\strut `Karel squeezed Petr's finger into the door.'\label{skrip}
 
\noindent The other category of the \ili{Czech}/\ili{Polish} \textit{-n-ou} semelfactives are unergatives, the equivalents of \ili{English} \isi{semelfactive}s such as \textit{sneeze} or \textit{bark}, which denote a single stage event in sentences like in:

\ex. 
\a. The baby sneezed once at 8 o'clock.  
\b. The dog suddenly barked at me.

In \ili{English}, such \isi{verb}s are usually homonymous with activities: \isi{iterative}s as in \ref{sneeze} and habituals as in \ref{barked} (cf. \citealt{Carlson2012}).

\ex. 
\a. The baby sneezed for a few minutes.\label{sneeze}
\b. The dog barked for several minutes (every Friday).\label{barked}

Contrary to \ili{English}, the unergative \textit{-n-ou} verbs such as the \ili{Polish} \textit{kich-n-\k{a}-\'c} `sneeze (once)', \textit{wark-n-\k{a}-\'c} `gnarl (once)', \textit{ziew-n-\k{a}-\'c} `yawn' or the \ili{Czech} \textit{m\'av-n-ou-t}  `wave (once)', \textit{syk-n-ou-t} `hiss (once)', \textit{dup-n-ou-t} `stamp', etc. are unambiguously \isi{semelfactive}.\footnote{As explained in footnote \ref{fn:L}, the fact that these \ili{Czech} and \ili{Polish} \isi{verb}s do not form adjectival L-passives confirms that they are unergatives rather than unaccusatives (cf. \textit{*kich-\l-y}, \textit{*wark-\l-y}, \textit{*ziew-\l-y}, \textit{*m\'av-l-y}, \textit{*syk-l-y}, \textit{*dup-l-y}, etc.).
} % end of fn
\par
Dividing the \textit{-n-ou} part of the stem into a sequence of the light \textit{-n} and the genuine theme vowel \textit{-ou} allows us to associate the argument-structural properties of degree achievement and  \isi{semelfactive} stems with their syntactic representations in a way which captures the fact that all theme vowels are verbalizers. \is{thematic suffix} However, since the degree achievement stems are unaccusative and the semelfactive stems are either transitive/accusative or unergative, representing the \textit{-ou} theme as a simplex verbalizing head in syntax (such as the minimalist ``little v'') does not lead to predictions about the relation between the geometry of their syntactic representations and received \isi{argument structure}s.
\par
The alternative is a representation of the \textit{-ou} theme as a monotonically growing sequence of heads which realizes the `unergative\,$>$\,accusative\,$>$\,unaccusative' hierarchy. For the purposes of our discussion of the \isi{iterative alternation}, let us represent the eventive verbal structure simply as an articulated VP, as in \Next, where V\textsubscript{n} heads indicate levels of embedding.\footnote{This is an approximation of the representation of the argument structure discussed in \cite{NU}, which is argued there to include case positions. Although important from the perspective of argument realization, the syntactic representation of the `unergative\,$>$\,accusative\,$>$\,unaccusative' hierarchy as in \ref{argstr} is sufficient for present purposes. 
} %end of fn on case peels


\ex.\label{argstr} 
\begin{forest}nice empty nodes, for tree={l sep=0.65em,l=0,calign angle=63}
[V$_{3}$P, s sep=11pt [V$_{3}$]
[V$_{2}$P, s sep=9pt [V$_{2}$][V$_{1}$P, s sep=-10pt [V$_{1}$][\ldots]]{\draw (.east) node[right]{unaccusative}; }
]{\draw (.east) node[right]{accusative}; }
]{\draw (.east) node[right]{unergative}; }
]
 \end{forest} 
 
Such a representation reflects structural proximity between unergatives and accusatives based on the observation that external arguments of unergatives and accusatives \is{argument structure} are event initiators, which are introduced by higher heads than arguments of unaccusatives are (e.g. \citealt{LevinandRapp1995} and \citealt{Ramchand08}). In the domain of \textit{-n-ou} stems, this sequence reflects the fact that a subset of \isi{semelfactive}s can be either unergative or accusative but never unaccusative, such as for instance the \ili{Polish} \textit{gwizd-n-\k{a}-\'c}. In \ref{gwizd-a}, it has a literal meaning `whistle' when unergative and in \ref{gwizd-b}, where it occurs with an accusative object, it has a non-literal meaning `steal'.

\ex. \ili{Polish} (\citealt[ex. 89]{NU})
\ag. Jan gwizd-n-\k{a}-\l.\\
Jan.\textsc{nom} whistle-\textsc{give}-\textsc{ou}-\textsc{part}\\
\strut `Jan whistled (once).'\label{gwizd-a}
\bg. Jan {gwizd-n-\k{a}-\l}  kred-\k{e} z klasy.\\
Jan.\textsc{nom} {whistle-\textsc{give}-\textsc{ou}-\textsc{part}}  chalk-\textsc{acc} from classroom\\
\strut `Jan has stolen the chalk from the classroom.'\label{gwizd-b}

What follows from the representation of the verbal \isi{argument structure} as in \ref{argstr} and the fact that \textit{-ou} is an exponent of the eventive verbal structure in three kinds of argument-structural \textit{-n-ou} stems is the shape of the lexical entry as in:

\ex. Lexical entry for the \textit{-ou} theme in \ili{Czech} and \ili{Polish}\label{lex:ou}\\[0.5ex]
[ V$_{3}$ [ V$_{2}$ [ V$_{1}$ ]]]  $\Leftrightarrow$ \textit{-ou}

\noindent
The smallest subset of the VP structure that can be lexicalized as \textit{-ou} is present in degree achievements, a class of \textit{-n-ou} \isi{verb}s that are, let us restate, exclusively unaccusative, as for instance in the \ili{Czech} example in \Next.

\exg. 
Jan bled-n-u-l.\\
Jan.\textsc{nom} pale-\textsc{get}-\textsc{ou}-\textsc{part}\\
\strut `Jan was getting pale.'

The merger of the partially derived  \isi{semelfactive} stem like \textit{bled-n} `get pale' in \ref{bled-n} with the verbal \isi{feature} V$_{1}$ is followed by the \isi{spell-out} procedure, as shown in:


\ex. Spell-out of \textit{-ou} in an unaccusative degree achievement stem \textit{bled-n-ou} `get pale'\\[0.75ex]
\begin{forest}nice empty nodes, for tree={l sep=0.7em,l=0,calign angle=63}
[V$_{1}$P [V$_{1}$][GetP, s sep=10pt [AdjP [\textit{bled}\\`pale', roof]][GetP [Get]
]{\draw (.east) node[right]{$\Rightarrow$ \textit{n}}; }
]]
\end{forest} 
\hskip 0.25cm $\leadsto$ \hskip -1.75cm
\begin{forest}nice empty nodes, for tree={l sep=0.7em,l=0,calign angle=63}
[V$_{1}$P, s sep=37pt [GetP, s sep=10pt, name=tgt [AdjP [\textit{bled}\\`pale', roof]][GetP [Get]
]{\draw (.east) node[right]{$\Rightarrow$ \textit{n}}; }] 
[V$_{1}$P [V$_{1}$][\ldots, name=t]
 ]{\draw (.east) node[right]{$\Rightarrow$ \textit{ou}}; }
 ]
\draw[dashed,->,>=stealth] (t) [in=-150,out=-120,looseness=2.75]  to (tgt);
\end{forest}

\vskip -1.25cm

\noindent Following snowballing, \textit{-ou} becomes spelled out as the smallest subset of \ref{lex:ou} and ends up as the external suffix on the adjectival \isi{root} \textit{bled}.
\par
In the case of transitive/accusative \isi{semelfactive}s, like the \ili{Czech}/\ili{Polish} \textit{kop-n-\k{a}-\'c} `kick' in \ref{Karol}, a bigger subset of the verbal structure is present, the one that includes \isi{feature}s V$_{1}$ and V$_{2}$. Each merger of the verbal \isi{feature} triggers the \isi{spell-out} procedure, as outlined in \Next:

\ex. Spell-out of \textit{-ou} in an accusative semelfactive stem \textit{kop-n-ou} `give a kick'\label{so:acc:kop}

\hskip 0.25cm \begin{forest}nice empty nodes, for tree={l sep=0.65em,l=0,calign angle=63}
[V$_{2}$P, s sep=10pt [V$_{2}$][V$_{1}$P, s sep=-5pt [V$_{1}$][GiveP, s sep=5pt [NP [\textit{kop}\\`kick', roof]]
[GiveP, s sep=-5pt [Give][GetP [Get]]
]{\draw (.east) node[right]{$\Rightarrow$ \textit{n}}; }
]]]
\end{forest} 
$\leadsto$ \hskip -2.25cm
\begin{forest}nice empty nodes, for tree={l sep=0.65em,l=0,calign angle=63}
[V$_{2}$P, s sep=20pt [GiveP, s sep=5pt, name=tgt [NP [\textit{kop}\\`kick', roof]]
[GiveP [Give][GetP [Get]]]{\draw (.east) node[right]{$\Rightarrow$ \textit{n}}; }] 
[V$_{2}$P [V$_{2}$][ [\ldots, name=t][V$_{1}$P [V$_{1}$][\ldots, name=V1]
]]]{\draw (.east) node[right]{$\Rightarrow$ \textit{ou}}; }
]
\draw[dashed,->,>=stealth] (t) [in=-150,out=-120,looseness=2.7]  to (tgt);
 \draw[dashed,->,>=stealth] (V1) to[out=south west,in=south west,looseness=1.75] (t);
\end{forest}

\vskip -1cm
\noindent Following snowballing at the first cycle and spec-to-spec movement at the second cycle, the \textit{-ou} theme \is{thematic suffix} spells out the accusative V$_{2}$P structure and comes out as the outer suffix.
\par
In turn, the derivation of unergative \isi{semelfactive}s, like the \ili{Czech} \textit{syk-n-ou-t} `hiss' or the \ili{Polish} \textit{gwizd-n-\k{a}-\'c} `whistle' in \ref{gwizd-a} involves the merger of the full set of V-features, resulting in the formation of the unergative superstructure, the structure that is a notch bigger than accusative semelfactive. \is{feature}
As shown in \Next on the example of \textit{gwizd-n-\k{a}-\'c}, the merger of each V-feature is, again, followed by \isi{spell-out}. 

\ex. Spell-out of \textit{-ou} in an unergative \isi{semelfactive} stem \textit{gwizd-n-\k{a}} `whistle'\label{so:unacc:gwizd}

\hspace{-30pt}\begin{forest}nice empty nodes, for tree={l sep=0.65em,l=0,calign angle=63}
[V$_{3}$P, s sep=30pt [GiveP, s sep=8pt, name=tgt [NP [\textit{gwizd}\\`whistle', roof]]
[GiveP [Give][GetP [Get]]]{\draw (.east) node[right]{$\Rightarrow$ \textit{n}}; }] 
[V$_{3}$P, s sep=8pt [V$_{3}$]
[ [\ldots, name=V3]
[V$_{2}$P, s sep=8pt [V$_{2}$][ [\ldots, name=t][V$_{1}$P [V$_{1}$][\ldots, name=V1]
]]]]]{\draw (.east) node[right]{$\Rightarrow$ \textit{ou}}; }
]
\draw[dashed,->,>=stealth] (V3) [in=-150,out=-120,looseness=2.7]  to (tgt);
 \draw[dashed,->,>=stealth] (t) to[out=south west,in=south west,looseness=2.05] (V3);
 \draw[dashed,->,>=stealth] (V1) to[out=south west,in=south west,looseness=1.65] (t);
\end{forest}

\vskip -1cm
\noindent Following the movements of the derived \textit{-n} stem, the GiveP constituent, the \textit{-ou} theme spells out the unergative V$_{3}$P superstructure and, like before, comes out as the outer suffix on the nominal root.

\section{Properties of the alternation}

There are two key properties of the  alternation between \textit{-n-ou} and \textit{-aj} stems. Namely, the alternation targets perfective stems and it preserves the \isi{argument structure} of the stem. 

\subsection{Perfective stems}

The \isi{semelfactive} stems are inherently perfective, which means that the event they express is bounded, hence countable (\citealt{Renat1979,Bach1986,deSwart1998,Willim2006,Dickey2016}). A bounded (countable) event denoted by a semelfactive stem can be iterated, which is reflected in the  alternation illustrated on the example of a few \ili{Czech} and \ili{Polish} \isi{verb}s in the following. 

\ex. Examples of \isi{semelfactive}-\isi{iterative alternation} in \ili{Czech}
\ag. 
kop-\textbf{n-ou}-t -- kop-\textbf{a}-t\\
kick-\textsc{give-ou}-\textsc{inf} {} kick-\textsc{aj}-\textsc{inf}\\
\strut `give a kick' \hskip 32pt `be giving kicks, kick repeatedly'
\bg. 
mrk-\textbf{n-ou}-t -- mrk-\textbf{a}-t\\
wink-\textsc{give-ou}-\textsc{inf} {} wink-\textsc{aj}-\textsc{inf}\\
\strut `give a wink' \hskip 32pt `be giving winks, keep winking'
\cg.
kous-\textbf{n-ou}-t -- kous-\textbf{a}-t\\
bite-\textsc{give-ou}-\textsc{inf} {} bite-\textsc{aj}-\textsc{inf}\\
\strut `give a bite' \hskip 33pt `bite repeatedly, keep biting'

\ex. Examples of semelfactive-\isi{iterative alternation} in \ili{Polish}\label{polisz}
\ag. 
liz-\textbf{n-\k{a}}-\'c -- liz-\textbf{a}-\'c\\
lick-\textsc{give-ou}-\textsc{inf} {} lick-\textsc{aj}-\textsc{inf}\\
\strut `give a lick' \hskip 32pt `be giving licks, lick repeatedly'
\bg. 
dotk-\textbf{n-\k{a}}-\'c -- dotyk-\textbf{a}-\'c\\
touch-\textsc{give-ou}-\textsc{inf} {} touch-\textsc{aj}-\textsc{inf}\\
\strut `give a touch' \hskip 33pt `touch repeatedly'
\cg.
bek-\textbf{n-\k{a}}-\'c -- bek-\textbf{a}-\'c\\
burp-\textsc{give-ou}-\textsc{inf} {} burp-\textsc{aj}-\textsc{inf}\\
\strut `burp once' \hskip 38pt `keep burping, burp repeatedly'



\noindent The \textit{-aj} \isi{iterative}s retain the Give-readings \is{light Give} of  semelfactive \textit{-n-ou} stems, which is expected if iteratives denote a repetition of the single stage event denoted by the corresponding \isi{semelfactive} stem.\footnote{This comes with a caveat regarding the extensions of the \isi{iterative} readings denoted by the \textit{-aj} stems into habitual and/or frequentative readings, a class broadly labeled as activities. The morphological form of the three types of \isi{activity} \isi{verb}s is identical and includes the \textit{-aj} theme to the effect that iterative, habitual, and frequentative readings can be differentiated by adverbial modifiers, in a similar way as in \ili{English}, as for instance in \Next (see \citealt{Carlson2012}). 

\ex. 
\a. The dog barked \textbf{for the whole night}. \hfill (\isi{iterative})
\b. The dog barked \textbf{every time he was hungry}.  \hfill (habitual/frequentative)

Unless in the unlikely scenario that the distinction between iteratives, habituals, and frequentatives is not part of lexical aspect, this points to an analysis of \textit{-aj} -- as well as the \ili{English} \isi{verb}s like \textit{bark}, \textit{cough}, \textit{wink} -- as \isi{morpheme}s that are overspecified with respect to the \isi{feature}s forming these aspectual categories, in a similar way the \textit{-ou} theme is \is{thematic suffix} overspecified for argument-structural properties, the \textit{-n} morpheme for the light Get and Give, etc. \is{light Give}
} %end of fn
\par
Although the  alternation targets a considerable subset of nominal roots that form \textit{-n-ou} semelfactives, certain roots that form such semelfactives will not form \isi{iterative} \textit{-aj} stems. For instance, nominal roots such as the \ili{Polish} \textit{krzyk-} `a scream' or \textit{ryk-} `a roar' build  semelfactives \textit{krzyk-n-\k{a}-\'c} `give a scream', \textit{ryk-n-\k{a}-\'c} `give a roar' but they alternate with stative \textit{-e} stems \textit{krzycz-e-\'c} `to scream', \textit{rycz-e-\'c} `to roar' rather than with \isi{iterative} \textit{-aj} stems (the unattested *\textit{krzyk-a-\'c}, *\textit{ryk-a-\'c}). This, however, is expected under a proviso that there is a syntactically sensitive typology of roots that goes beyond the basic distinction into lexical categories of N vs. Adj vs. V, a scenario we need to assume anyways in order to control for the fact that not all nominal roots form \isi{semelfactive} \textit{-n-ou} stems in the first place (let us recall here the discussion of unattested  semelfactives with nominal roots such as \textit{matk-} `mother' or \textit{st\'o\l-} `table' from \sectref{sec:roots}).
Given the fact that \textit{krzyk-} and \textit{ryk-} are nouns of perception and production of sounds we correctly expect them to produce \textit{-e} stems, which typically form this subclass of statives, rather than \isi{iterative}s. Thus, in the case of such \isi{root}s it is safe to state that they simply form bases for semelfactive \textit{-n-ou} stems and \textit{-e} stems but there is no derivational relation between \isi{semelfactive}s and \textit{-e} statives.
\par
Unlike in the case of semelfactives, bare roots of degree achievement \textit{-n-ou} stems do not undergo the  \isi{iterative alternation}, as illustrated by the following examples.

\ex. \ili{Czech} 
\ag.	
{bled-\textbf{n-ou}-t} -- *bled-\textbf{a}-t\\
{pale-\textsc{get-ou}-\textsc{inf}}\\
\strut `get pale'
\bg. 
{hluch-\textbf{n-ou}-t} -- *hluch-\textbf{a}-t\\
{deaf-\textsc{get-ou}-\textsc{inf}}\\
\strut `get deaf'
\cg. 
{mrz-\textbf{n-ou}-t} -- *mrz-\textbf{a}-t\\
{freeze-\textsc{get-ou}-\textsc{inf}}\\
\strut `get frozen'

\ex. \ili{Polish}\label{Pol:da:not}
\ag.	
{mok-\textbf{n-\k{a}}-\'c} -- *mocz-\textbf{a}-\'c\\
{wet-\textsc{get-ou}-\textsc{inf}}\\
\strut`get wet'
\bg.
{sch-\textbf{n-\k{a}}-\'c} -- *sch-\textbf{a}-\'c \\
{dry-\textsc{get-ou}-\textsc{inf}}\\
\strut `get dry'
\cg.
{chud-\textbf{n-\k{a}}-\'c} -- *chud-\textbf{a}-\'c \\
{slim-\textsc{get-ou}-\textsc{inf}}\\
\strut `get slim, loose weight'

\noindent
This contrast follows from the fact that  degree achievement stems are imperfective, which means that the event they express is unbounded, hence uncountable. An unbounded (uncountable) event denoted by such a stem cannot be iterated. However, once a degree achievement stem has a prefix which makes it perfective, such a stem can undergo the  \isi{iterative alternation} quite regularly, as shown in the following: \is{iterative}


\ex. \ili{Czech}
\ag. vy-bled-\textbf{n-ou}-t -- vy-bled-\textbf{a}-t\\
\textsc{pfv}-pale-\textsc{get-ou}-\textsc{inf} {} \textsc{pfv}-pale-\textsc{aj}-\textsc{inf}\\
\strut `get pale' \hskip 62pt `pale out repeatedly'
\bg. u-vad-\textbf{n-ou}-t -- u-vad-\textbf{a}-t\\
\textsc{pfv}-wither-\textsc{get-ou}-\textsc{inf} {} \textsc{pfv}-wither-\textsc{aj}-\textsc{inf}\\
\strut `wither' \hskip 77.5pt `wither repeatedly'
\cg.za-mrz-\textbf{n-ou}-t -- za-mrz-\textbf{a}-t\\
\textsc{pfv}-freeze-\textsc{get-ou}-\textsc{inf} {} \textsc{pfv}-freeze-\textsc{aj}-\textsc{inf}\\
\strut `get frozen' \hskip 60pt `freeze repeatedly'

\ex. \ili{Polish}
\ag.	
za-mok-\textbf{n-\k{a}}-\'c -- za-mak-\textbf{a}-\'c \\
\textsc{pfv}-wet-\textsc{get-ou}-\textsc{inf} {} \textsc{pfv}-wet-\textsc{aj}-\textsc{inf}\\
\strut `get wet' \hskip 62pt `moisten repeatedly or gradually'
\bg.
wy-sch-\textbf{n-\k{a}}-\'c -- wy-sych-\textbf{a}-\'c \\
\textsc{pfv}-dry-\textsc{get-ou}-\textsc{inf} {} \textsc{pfv}-dry-\textsc{aj}-\textsc{inf}\\
\strut `get dry' \hskip 61.5pt `get dry repeatedly or gradually'
\cg.
wy-mi\k{e}k-\textbf{n-\k{a}}-\'c -- wy-mi\k{e}k-\textbf{a}-\'c\\
\textsc{pfv}-soft-\textsc{get-ou}-\textsc{inf} {} \textsc{pfv}-soft-\textsc{aj}-\textsc{inf}\\
\strut `chicken out' \hskip 42pt `chicken out repeatedly'


\subsection{Argument structure preservation}

The other essential property of the  \isi{iterative alternation} with \textit{-n-ou} stems is the preservation of the \isi{argument structure}. As shown in \ref{pres1}--\ref{pres2}, accusative \isi{semelfactive} \textit{-n-ou}  stems will form accusative iterative \textit{-aj} stems. \is{iterative}

\ex.\label{pres1} \ili{Czech}
\ag.[]\hspace{-22pt}Jan \{ kop\textbf{nu}l  {/} {kop\textbf{a}l \}} m\'i\v{c}.\\
\hspace{-22pt}Jan.\textsc{nom} {}  kicked\textsubscript{semel} {}  {kicked\textsubscript{iter}}  ball.\textsc{acc}\\ 
\hspace{-22pt}\strut `Jan kicked the ball once / repeatedly.'

\ex.\label{pres2} \ili{Polish}\label{dotykal}
\ag.[]\hspace{-22pt}Jan \{ dotk\textbf{n\k{a}}\l {} {/} {dotyk\textbf{a}\l \ \}} detonator.\\
\hspace{-22pt}Jan.\textsc{nom} {}  touched\textsubscript{semel} {}  touched\textsubscript{iter}  detonator.\textsc{acc}\\ 
\hspace{-22pt}\strut `Jan touched the detonator once / repeatedly.'

Unergative \isi{semelfactive} \textit{-n-ou} stems will form unergative \textit{-aj} stems, as shown in the following:

\ex. \ili{Czech}
\ag.[]\hspace{-22pt}Pes \{ \v{s}t\v{e}k\textbf{nu}l  {/} {\v{s}t\v{e}k\textbf{a}l \}}.\\
\hspace{-22pt}dog.\textsc{nom} {} barked\textsubscript{semel} {} {barked\textsubscript{iter}}\\
\hspace{-22pt}\strut `The dog barked once / repeatedly.'


\ex. \ili{Polish}
\ag.[]\hspace{-22pt}Jan \{ mrug\textbf{n\k{a}}\l {} {/} {mrug\textbf{a}\l \ \}}.\\
\hspace{-22pt}Jan.\textsc{nom} {} winked\textsubscript{semel} {} {winked\textsubscript{iter}}\\
\hspace{-22pt}\strut `Jan winked once / repeatedly.'

The \isi{argument structure} preservation holds also in the case of anticausative semelfactives, such as the \ili{Czech}/\ili{Polish} \isi{verb} \textit{couvnout}/\textit{cofn\k{a}\'c} `move back', as illustrated for \ili{Polish} in the following:

\exg.
Motor si\k{e} \{ cof\textbf{n\k{a}}\l {}  {/} {cof\textbf{a}\l {} \}}.\\
motorcycle.\textsc{nom} \textsc{refl} {} moved.back\textsubscript{semel} {}  {moved.back\textsubscript{iter}}\\ 
\strut `The motorcycle moved back once / repeatedly.'

\noindent
Argument structure is also preserved in \isi{iterative}s formed with perfectivized stems of degree achievements prefixed with \textit{wy-} such the \ili{Polish} \textit{wymi\k{e}ka\'c} `chicken out repeatedly':

\exg.
Nasi zawodnicy nie mog\k{a} \{ wy-mi\k{e}k\textbf{n\k{a}}\'c {/} {wy-mi\k{e}k\textbf{a}\'c \}.}\\
our players.\textsc{nom.pl} not can {} chicken.out\textsubscript{deg.ach} {} chicken.out\textsubscript{iter}\\
\strut `Our players must not chicken out this time / repeatedly.'



\section{Representation}

The properties of the  alternation between perfective (bounded/countable) \isi{verb}s and \isi{iterative}s can be explained if we follow a strand of work on aspectual categories that argues for a compositional relation between these two types of verbs. 
More specifically, the properties of the alternation can be captured if iterative \textit{-aj} stems are structurally bigger than perfective (bounded/countable) stems. This can be generally represented as in the following, where the relevant size difference is pre-theoretically marked as an extra \isi{iterative}-forming Asp head on top of the perfective stem:

\ex.\label{AspP}
\begin{forest}nice empty nodes, for tree={l sep=0.7em,l=0,calign angle=63}
	[\hspace{5pt}AspP\textsubscript{iter} [Asp] [perfective 
	[~~~~~~, roof]]]
	\end{forest}

\vskip -0.35cm
For \isi{semelfactive}s, this means the iterative Asp \isi{feature} will apply to the \textit{-n-ou} stem that contains the light \isi{verb} Give \textit{-n}. \is{light Give} Since both accusative and unergative semelfactives undergo the \isi{iterative alternation}, the stem that the Asp feature applies to must include, respectively, the V$_{2}$P subset or the V$_{3}$P superset of \textit{-ou}. The addition of the \isi{iterative} \isi{feature} Asp to both types of semelfactives  is shown on the example of an accusative \textit{kop-n-ou-t} `give a kick' and an unergative \textit{gwizd-n-\k{a}-\'c} `whistle' in the following representations, which show the stages before the \isi{spell-out} of AspP as \textit{-aj} will over-ride the \textit{-n-ou} sequence:

	\ex.\label{kop-aj} Iterative stems based on \isi{semelfactive}s before the \isi{spell-out} of AspP as \textit{-aj} 
	\a.\label{50a} accusative \textit{kop-n-ou-t} `give a kick' (Cz)\\[0.5ex]
	\begin{forest}nice empty nodes, for tree={l sep=0.65em,l=0,calign angle=63}
	[AspP, s sep=-2pt [Asp] [V$_{2}$P, s sep=10pt [GiveP
	[NP  [\textit{kop}\\`kick', roof]
	] 
	[GiveP, s sep=5pt
	[\textit{-n}, roof] ]] 
	[V$_{2}$P 
	[\textit{-ou}, roof]]]]
	\end{forest}\\[0.25ex]
	\b.\label{50b} unergative \textit{gwizd-n-\k{a}-\'c} `whistle' (Pol)\\[0.5ex]
	\begin{forest}nice empty nodes, for tree={l sep=0.65em,l=0,calign angle=63}
	[AspP, s sep=-2pt [Asp] [V$_{3}$P, s sep=10pt [GiveP
	[NP  [\textit{gwizd}\\`whistle', roof]
	] 
	[GiveP, s sep=5pt
	[\textit{-n}, roof] ]] 
	[V$_{3}$P 
	[\textit{-ou}, roof]]]]
	\end{forest}

For degree achievements perfectivized with a prefix, this means the \isi{iterative} Asp will apply to the \textit{-n-ou} stem that contains the Get \is{light Get} subset of light \isi{verb} \textit{-n}, and the V$_{1}$P subset of the \textit{-ou}, which is present in unaccusatives. This is illustrated on the example of the \ili{Czech} \textit{za-mrz-n-ou-t} `get frozen', which alternates with \textit{za-mrz-a-t} `freeze repeatedly' in \Next. \is{containment}
As for the perfectivizing prefix \textit{za-}, which is represented below simply as the realization of the Perf(ective)P, which I will assume to merger directly with the adjectival root of a degree achievement (marked as the AP). 

\ex.\label{zamarzaj} Iterative stem \textit{za-mrz-a-t} `freeze repeatedly' (Cz) based on the root of a degree achievement before the \isi{spell-out} of AspP as \textit{-aj}\\[0.5ex]
	\begin{forest}nice empty nodes, for tree={l sep=0.65em,l=0,calign angle=63}
	[AspP, s sep=-2pt [Asp] [V$_{1}$P, s sep=30pt [GetP, s sep=0pt
	[PerfP [PerfP [\textit{za-}, roof]][AP  [\textit{mrz}\\`freeze', roof]]
	] 
	[GetP
	[Get]]{\draw (.east) node[right]{$\Rightarrow$ \textit{n}}; }
	] 
	[V$_{1}$P 
	[V$_{1}$]]{\draw (.east) node[right]{$\Rightarrow$ \textit{ou}}; }
	]]
	\end{forest}

This assumption about \textit{za-} is in agreement with observations about its low position in \ili{Polish} in \cite{Sven2004} (who credits Patrycja Jab\l{}o\'nska with this insight), \cite{Wiland2012}, and in Slovenian in \cite{Zaucer2005}. More generally, the idea that verbal prefixes in \ili{Czech} are base generated as sisters to the \isi{root} is compliant with \citeauthor{Caha-Zikova}'s \citeyearpar{Caha-Zikova} claim that prefixed \isi{verb} stems in \ili{Czech} have an underlying structure as in \Next, the proposal first put forth for \ili{Slavic} in \cite{Sven2004b}.

\ex. [[ pref root ] theme ] \is{thematic suffix} \is{root}

Apart from the formation of an \isi{iterative} based on a prefixed root of a  degree achievement stem, an inferential argument in favor of the size relation between iteratives and (unprefixed) semelfactives 	
is based on the fact that we can construe an iterative reading of a \isi{semelfactive} \textit{-n-ou} \isi{verb} by adding a frequency adverbial. 
This is illustrated for \ili{Polish} by the following examples:

\ex. 
\ag. 	Jan kop-n-\k{a}-\l {} pi\l k\k{e}. \hskip 0.35cm (\isi{semelfactive})\\
	Jan.\textsc{nom} kick-\textsc{give-ou-part} ball.\textsc{acc}\\
	\strut `Jan kicked the ball once.'
\bg. 	Jan kop-n-\k{a}-\l {} pi\l k\k{e} \textit{pi\k{e}\'c razy} (\isi{iterative}) \\
	Jan kick-\textsc{give-ou-part} ball {five times}\\
	\strut `Jan kicked the ball five times.'

\ex. 
\ag. 	Jan kaszl-n-\k{a}-\l. \hskip 0.8cm (\isi{semelfactive})\\
	Jan.\textsc{nom} cough-\textsc{give-ou-part}\\
	\strut `Jan coughed once.'
\bg. 	Jan kaszl-n-\k{a}-\l {} \textit{pi\k{e}\'c razy} (\isi{iterative}) \\
	Jan cough-\textsc{give-ou-part} {five times}\\
	\strut `Jan coughed five times.'

The opposite, that is the addition of a punctual adverbial to an iterative \textit{-aj} \isi{verb}, does not result in the  semelfactive reading of the \textit{-aj} verb, as illustrated for \ili{Polish} in the following:

\exg. Jan kop-{a}-\l {} pi\l k\k{e} (\textit{o 5-tej}). \hskip 0.25cm (\isi{iterative})\\
	Jan.\textsc{nom} kick-\textsc{aj-part} ball.\textsc{acc} { \ at five}\\
	\strut `Jan kicked the ball repeatedly at 5 o'clock.'
	
\exg.	Jan kaszl-a-\l  {} (\textit{o 5-tej}). \hskip 1.25cm (\isi{iterative})\\
	Jan.\textsc{nom} cough-\textsc{aj-part} { \ at five}\\
	\strut `Jan coughed repeatedly at 5 o'clock.'

\noindent
While there is no agreement in the literature about the identification of the semantic content of what is represented in \ref{AspP} as the Asp head, the syn-sem representation of iteratives as bigger than \isi{semelfactive}s is in line with a strand of work on the semantics of aspectual classes that describes semelfactives as a subset structure of \isi{iterative} activities. For example, in approaches that extend Vendler's \citeyearpar{Vendler1967} description of aspectual classes, both semelfactives and activities are described as [$+$dynamic] situations, with activities additionally described as [$+$durative] (e.g. \citealt{Smith1997,Olsen1994,Olsen1997,Beavers2008}). 
\par
In \cite{XM2004}, where the \isi{activity} class is split such that iteratives constitute a separate category, iteratives that correspond to the \ili{English} \isi{verb}s like in:

\ex. He coughed \textbf{for 5 minutes}.

are classified as derived \isi{semelfactive}s, as opposed to basic semelfactives like in:

\ex. He coughed \textbf{once}.

In turn, in a non-Vendlerian approach such as \cite{Egg2017}, \isi{iterative}s are derived either by lexical construction or aspectual coercion applied to \isi{semelfactive}s. Egg's \citeyearpar{Egg2017} analysis stands in opposition to \cite{rothstein2004}, who proposes that iteratives are more basic than semelfactives, which effectively makes semelfactives a subclass of \isi{activity} predicates, a scenario not compatible with the syn-sem description of both categories in \ref{kop-aj}. Egg shows, among others that, contrary to the predictions of Rothstein's proposal, \isi{iterative}s are composed of minimal eventualities. For instance, iteratives like \textit{tremble} clearly denote back and forth movements whereas \textit{tremble 5 times} denotes iterations of such movements only.\footnote{See also \citet[\S4.1]{NU} for challenges in applying Rothstein's proposal to the morpho-semantic description of \ili{Czech} and \ili{Polish} semelfactives.
}% end of fn
\par Assuming the structures in \ref{kop-aj}--\ref{zamarzaj} represent the \isi{iterative} \textit{-aj} stems that alternate with \textit{-n-ou} stems, let us attempt to spell out the AspP in these structures following the spell-out procedure discussed in the previous chapter.

\section{Spelling out \textit{-aj} stems with subextraction}\label{sec:aj-stems}

We need to apply \isi{spell-out} operations to the trees in \ref{kop-aj}--\ref{zamarzaj} in such a way that we preserve the root in \isi{semelfactive}s and the prefix-root constituent in perfectivized stems of degree achievements and make sure the spell-out of the Asp head will over-ride the earlier spell-outs of \textit{-n} and \textit{-ou} in these structures -- the procedure that will derive the \isi{reduction} in the number of affixes. For the illustration of the application of the spell-our procedure recapped in \sectref{section:Starke2018} to our structures, let us first work with the  semelfactive \textit{kop-n-ou-t} in \ref{50a}.

\subsection{Deriving the reduction}

The first step of the \isi{spell-out} algorithm, \textsc{stay}, does not lead to the spell-out of Asp in \ref{AspP} since the insertion of \textit{-aj} in the AspP node would over-ride the entire stem including the root, counter fact. \is{spell-out algorithm} The second step, the spec-to-spec movement of GiveP shown in \Next, does not lead to its spell-out either, since it results in the formation of an unattested stem \textit{*kop-n-aj}. (Let us recall from \ref{so:acc:kop} that GiveP is the constituent that moves at the cycle directly preceding the merger of Asp).


\ex.\label{kop-n-aj}
	\begin{forest}nice empty nodes, for tree={l sep=0.65em,l=0,calign angle=63}
	[AspP [Asp] [V$_{2}$P, s sep=10pt [GiveP, s sep=5pt
	[NP  [\textit{kop}\\`kick', roof]][GiveP 
	[\textit{-n}, roof] ]] 
	[V$_{2}$P 
	[\textit{-ou}, roof]]]]
	\end{forest}
	 $\leadsto$\hskip -0.75cm 
	\begin{forest}nice empty nodes, for tree={l sep=0.65em,l=0,calign angle=63}
	 [\textit{*kop-n-aj}, s sep=12pt [GiveP, s sep=5pt, name=tgt 
	 [NP  [\textit{kop}\\`kick', roof]] 
	[GiveP [\textit{-n}, roof] ]] [AspP, s sep=8pt [Asp] 
	[V$_{2}$P [..., name=t] 
	[V$_{2}$P [\textit{-ou}, roof]]]
	]{\draw (.east) node[right]{$\Rightarrow$ *\textit{aj}}; }]
	 \draw[dashed,->,>=stealth] (t) to[out=south west,in=south west,looseness=2.6] (tgt);
	\end{forest}

\vskip -0.65cm
Although the evacuation of GiveP \textit{kop-n} in \ref{kop-n-aj} allows Asp to be spelled out in such a way that the insertion of \textit{-aj} in the sister node to the landing site of \is{light Give} GiveP over-rides the \isi{spell-out} of the VP \textit{-ou}, \textit{-aj} surfaces here as the second suffix on the \isi{root}, counter fact. In other words, spec-to-spec movement does not derive the cutback in the number of suffixes we observe in the alternation between \isi{semelfactive} \textit{-n-ou} and \isi{iterative} \textit{-aj} stems. 
\par
In this case we need to backtrack \is{backtracking} by trying snowballing, the third step of the algorithm, as shown in: \is{spell-out algorithm}

\ex.\label{kop-n-ou-aj}
	\begin{forest}nice empty nodes, for tree={l sep=0.65em,l=0,calign angle=63}
	[AspP [Asp] [V$_{2}$P, s sep=10pt [GiveP, s sep=5pt
	[NP  [\textit{kop}\\`kick', roof]][GiveP 
	[\textit{-n}, roof] ]] 
	[V$_{2}$P 
	[\textit{-ou}, roof]]]]
	\end{forest}
	$\leadsto$ \hskip 0.5cm
	\begin{forest}nice empty nodes, for tree={l sep=0.65em,l=0,calign angle=63}
	[\textit{*kop-n-ou-aj}, s sep=15pt
	[V$_{2}$P, s sep=12pt, name=tgt [GiveP, s sep=5pt
	[NP  [\textit{kop}\\`kick', roof]][GiveP 
	[\textit{-n}, roof] ]] 
	[V$_{2}$P 
	[\textit{-ou}, roof]]][AspP [Asp][\dots, name=t]
	]{\draw (.east) node[right]{$\Rightarrow$ *\textit{aj}}; }
	]
	 \draw[dashed,->,>=stealth] (t) [in=-110,out=-115,looseness=1.6]  to (tgt);
	\end{forest}
	
Snowballing, however, also does not derive the desired result either since now \textit{-aj} ends up as the third suffix in the unattested stem \textit{*kop-n-ou-aj}.
Let us note here that the application of the truncation rule in \ili{Slavic} phonology as in \Next, whereby whereby a vowel in a cyclic \isi{morpheme} (essentially, a suffix) becomes deleted before a vowel, does not help, either.\footnote{There is a long tradition of applying the vowel deletion \is{vowel truncation} rule in \ref{VTr}, originally discovered to hold in \ili{Russian} conjugation in \cite{Jakobson1948}, in the derivation of surface forms throughout \ili{Slavic}, including \cite{Lightner1972}, \cite{Guss1980}, \citeauthor{Rubach1984} (\citeyear{Rubach1984,Rubach1993}), \cite{NH2009}, among others.
} %end of fn

\ex. \textbf{Vowel truncation}\label{VTr}\\[0.5ex]
V $\rightarrow$ $\emptyset$ / \_ V \is{vowel truncation}

This is so since the deletion of \textit{-ou} in front of \textit{-aj} as in:

\ex. kop-n-ou-aj $\rightarrow$ kop-n-$\emptyset$-aj

derives the unattested surface form \textit{*kop-n-aj}, the same result as in \ref{kop-n-aj}.
\par
Snowballing exhausts the list of movement operations in the \isi{spell-out} procedure discussed in \cite{Starke2018} with the subsequent \textsc{subderive} resulting in the formation of a prefix. 
As suggested in \sectref{sec:subext}, a
logical solution to the problem of spelling out Asp is to extend the list of movement operations by \textsc{subextract} and order it before \textsc{subderive}. \is{subextraction} When applied to our representation in \Next, the extraction of the NP \textit{kop} from the complex specifier GiveP \textit{kop-n} appears to derive the desired result.

\ex.\label{solved}
	\begin{forest}nice empty nodes, for tree={l sep=0.65em,l=0,calign angle=63}
	[AspP [Asp] [V$_{2}$P, s sep=10pt [GiveP, s sep=5pt
	[NP  [\textit{kop}\\`kick', roof]][GiveP 
	[\textit{-n}, roof] ]] 
	[V$_{2}$P 
	[\textit{-ou}, roof]]]]
	\end{forest}
	$\leadsto$
	{\small \begin{forest}nice empty nodes, for tree={l sep=0.65em,l=0,calign angle=63}
	[\textit{kop-aj}, s sep=18pt [NP, name=tgt  [\textit{kop}\\`kick', roof]]
	[AspP, s sep=5pt [Asp] [V$_{2}$P, s sep=10 [GiveP, s sep=5pt
	[\ldots, name=t][GiveP 
	[\textit{-n}, roof] ]] 
	[V$_{2}$P 
	[\textit{-ou}, roof]]]]{\draw (.east) node[right]{$\Rightarrow$ \textit{aj}}; }]
	\draw[dashed,->,>=stealth] (t) to[out=south west,in=south west,looseness=1.5] (tgt);
	\end{forest}}

\vskip -0.65cm
Following the extraction of the NP \textit{kop}, the \isi{spell-out} of its sister node AspP as \textit{-aj} over-rides the earlier spell-outs of both \textit{-n} and \textit{-ou}, resulting in the formation of \textit{kop-aj}, a bi-morphemic stem with a portmanteau suffix. The extraction preserves the nominal root and derives the  \isi{reduction} in the number of morphemes in the iterative \textit{-aj} stem with respect to the syntactically less complex \isi{semelfactive} \mbox{\textit{-n-ou}} stem. Let us also point out that the lexicalization of the complex AspP as the \textit{-aj} suffix in \ref{solved} adheres to Starke's \citeyearpar{Starke2018} contrast between `pre-'  vs. `post-' placement in terms of a binary vs. a unary foot in their syntactic representations (cf. the discussion in \sectref{sec:Starke2018}). This is so since the \isi{subextraction} that facilitates \isi{spell-out} in a derivation like in \ref{solved} does not appear to create a syntactically relevant trace (i.e. an object relevant for reconstruction), which makes it identical  to \isi{spell-out} driven movement that involves a specifier or a complement with this respect.
\par
The \isi{subextraction} of the root node will give a similar result when it applies to the representation with the unergative  semelfactives \textit{gwizd-n-\k{a}-\'c} `whistle once' in \ref{50b}. As shown in \Next, the spell-out of the remnant AspP as \textit{-aj} produces the desired \textit{gwizd-a-\'c} `whistle repeatedly' (modulo the infinitive suffix \textit{-\'c}).


\ex. \label{unerg:gwizd}
\begin{forest}nice empty nodes, for tree={l sep=0.65em,l=0,calign angle=63}
	[\textit{gwizd-aj}, s sep=15pt [NP, name=tgt  [\textit{gwizd}\\`whistle', roof]]
	[AspP, s sep=0pt [Asp] 
	[V$_{3}$P, s sep=12 [GiveP, s sep=5pt
	[\ldots, name=t][GiveP 
	[\textit{-n}, roof] ]] 
	[V$_{3}$P [\textit{-ou}, roof]]]
	]{\draw (.east) node[right]{$\Rightarrow$ \textit{aj}}; }]]
	\draw[dashed,->,>=stealth] (t) [in=-160,out=-135,looseness=1.55]  to (tgt);
\end{forest}

\vskip -0.75cm
Likewise, the \isi{subextraction} of the node containing the prefixed \isi{root} can apply to the representation based on the degree achievement \textit{za-mrz-n-ou-t} `get frozen' in \ref{zamarzaj}. As shown in \Next, such a movement will create a remnant AspP, which can be spelled out as \textit{-aj} in the desired \textit{za-mrz-a-t} `freeze repeatedly'. \is{containment}

\ex.\label{so:zamarzaj} 
\begin{forest}nice empty nodes, for tree={l sep=0.65em,l=0,calign angle=63}
	[\textit{za-mrz-aj}, s sep=15pt 
	[PerfP, s sep=5pt, name=tgt [PerfP [\textit{za-}, roof]][AP  [\textit{mrz}\\`freeze', roof]]
	]
	[AspP, s sep=0pt [Asp] 
	[V$_{1}$P, s sep=35 [GetP, s sep=5pt
	[\ldots, name=t][GetP 
	[Get]]{\draw (.east) node[right]{$\Rightarrow$ \textit{n}}; }
	] 
	[V$_{1}$P [V$_{1}$]]{\draw (.east) node[right]{$\Rightarrow$ \textit{ou}}; }
	]
	]{\draw (.east) node[right]{$\Rightarrow$ \textit{aj}}; }]]
	\draw[dashed,->,>=stealth] (t) [in=-150,out=-140,looseness=2]  to (tgt);
\end{forest}



\vskip -0.75cm
\noindent Let us observe that while we are able to obtain the \isi{reduction} of a sequence of two affixes to one with \textsc{subextract}, we need to control for the fact that \mbox{\textit{-aj}} spells out three different subtrees. \is{subextraction} In \ref{solved}, \textit{-aj} spells out AspP that contains \is{containment} GiveP and the accusative V$_{2}$P; in \ref{unerg:gwizd}, it spells out AspP that contains \is{light Give} GiveP and the unergative V$_{3}$P; in \ref{zamarzaj}, it spells out AspP that contains GetP and the unaccusative V$_{1}$P, the smallest subset of the \textit{-ou} theme. \is{containment} This raises the question: what is the shape of the lexical entry for \textit{-aj} such that it can be inserted in these three different-looking nodes? This issue is non-trivial since the lexical insertion mechanism that is regulated by the \isi{Superset Principle} requires a syntactic node to be a (sub-)constituent of a lexically stored tree. In the case we are considering, \textit{-aj} is inserted into AspP that \textbf{dominates} (sub-)constituents of two lexically stored trees: one for \textit{-n} and the other for \textit{-ou}. In other words, \textit{-aj} is inserted into a syntactic tree that can shrink in the middle rather than on top. \is{shrinking}
This issue can be resolved if the lexical entry for \textit{-aj} includes \isi{pointer}s to the lexical items \textit{-n} and \textit{-ou} rather than to syntactic nodes these exponents realize.

\subsection{Pointers}

\noindent
In \sectref{sec:pointers} we stated that the cyclicity of \isi{spell-out} enables the insertion mechanism to make reference to lexical items inserted at earlier cycles, a result achieved through a tool called a \isi{pointer}. Let us consider how such a  lexicalization scenario applies to the lexical entry for \textit{-aj} if it includes a pointer structure as in the following.

\ex. Lexical entry for the \textit{-aj} theme\label{final:aj}\\[0.5ex]
\begin{forest}nice empty nodes, for tree={l sep=0.7em,l=0,calign angle=63}
	[AspP [Asp] 
	[[\textit{n}, edge+={->, line width=0.15mm}]
	[\textit{ou}, edge+={->, line width=0.15mm}]
	]]{\draw (.east) node[right]{$\Leftrightarrow$ \textit{aj}}; }]]
\end{forest}

The entry for \textit{-aj} defined in such a way means that it can be inserted in AspP that contains \is{containment} \isi{feature} Asp and a \isi{pointer} structure with two particular lexical items, \textit{-n} and \textit{-ou}, which were inserted at earlier cycles. The item \textit{-aj} can, thus, spell out the following syntactic representations, which involve either the superset or the subset structures of \textit{-n} and \textit{-ou}: 



\ex.
\a.\label{pt:a}
\begin{forest}nice empty nodes, for tree={l sep=0.65em,l=0,calign angle=63}
	[AspP [Asp] 
	[, s sep=12 [GiveP, s sep=5pt
	[Give][GetP[Get]]] 
	[V$_{3}$P\textsubscript{unerg}, s sep=15pt [V$_{3}$]
	[V$_{2}$P\textsubscript{acc}, s sep=15pt [V$_{2}$][V$_{1}$P\textsubscript{unacc} [V$_{1}$]]]]]]
\end{forest}
\b.\label{pt:b}
\begin{forest}nice empty nodes, for tree={l sep=0.65em,l=0,calign angle=63}
	[AspP [Asp] 
	[, s sep=12 [GiveP, s sep=5pt
	[Give][GetP[Get]]] 
	[V$_{2}$P\textsubscript{acc}, s sep=15pt [V$_{2}$][V$_{1}$P\textsubscript{unacc} [V$_{1}$]]]]]
\end{forest}
\c.\label{pt:c}
\begin{forest}nice empty nodes, for tree={l sep=0.65em,l=0,calign angle=63}
	[AspP [Asp] 
	[, s sep=10 [GiveP, s sep=5pt
	[Give][GetP[Get]]] 
	[V$_{1}$P\textsubscript{unacc} [V$_{1}$]]]]
\end{forest}
\d.\label{pt:d}
\begin{forest}nice empty nodes, for tree={l sep=0.65em,l=0,calign angle=63}
	[AspP [Asp] 
	[, s sep=10 [GetP, s sep=5pt
	[Get]] 
	[V$_{1}$P\textsubscript{unacc} [V$_{1}$]]]]
\end{forest}

\noindent
The \textit{-aj} theme \is{thematic suffix} which spells out the unergative V$_{3}$P superstructure in \ref{pt:a} is present in stems like \textit{gwizda\'c} (Pol) `whistle repeatedly' in \ref{unerg:gwizd}. The \textit{-aj} with the accusative V$_{2}$P subset structure in \ref{pt:b} is present in stems like \textit{kopat} `kick repeatedly' (Cz, Pol) in \ref{solved}. 
In turn, while \textit{-aj} can also spell out the tree in \ref{pt:c}, that tree does not correspond to an attested syntactic representation. This is so since unaccusative \textit{-n-ou} stems only form degree achievements, which include the \is{light Get}. Thus, the unaccusative V$_{1}$P does not merge with GiveP but with its GetP subset -- the attested structure in \ref{pt:d}.
\par
To sum up, the \isi{reduction} in the number of \isi{morpheme}s can be derived with \isi{subextraction} from a complex specifier followed by the \isi{spell-out} of the remnant node. In the illustration of such a \isi{reduction} with the \isi{iterative alternation} that involves \textit{-n-ou} stems, the desired result of the over-riding of two smaller affixes  with one bigger affix can be obtained using the lexical insertion mechanism that makes reference to lexical items inserted at earlier cycles. \is{iterative}

\section{Subextract vs. backtracking}

An alternative way of obtaining a \isi{reduction} in the amount of \isi{morpheme}s based on \isi{backtracking} has been outlined in \sectref{sec:back}. \is{subextraction} According to the \isi{spell-out} logic we have been working with so far, an attempt to spell-out a \isi{feature} becomes undone if there is no lexical item that matches a tree structure and a different \isi{spell-out} option is attempted. In a backtracking derivation, this may mean moving back several cycles. To illustrate how the \isi{backtracking} derivation outlined in \sectref{sec:back} applies to the \isi{iterative alternation} that targets \textit{-n-ou} stems, let us work with the example involving the \ili{Czech} \textit{kop-n-ou-t} `give a kick' and \textit{kop-a-t} `kick repeatedly'.

\subsection{Structures that shrink \is{shrinking} in the middle}

The addition of the Asp head to the \isi{semelfactive} stem \textit{kop-n-ou} illustrated in \ref{solved} triggers \isi{spell-out}. If movement possibilities are exhausted, the derivation backtracks to the inside of the NP root \textit{kop} and spells out its subset structure, as shown in the following, where the structure of the NP root is represented as a sequence of N$_{n}$ heads that indicate contiguous the levels of embedding. 

\ex. \begin{forest}nice empty nodes, for tree={l sep=0.65em,l=0,calign angle=63} 
[N$_{3}$P [N$_{3}$]
[N$_{2}$P  [N$_{2}$]
[N$_{1}$P [N$_{1}$]]
]{\draw (.east) node[right]{$\Rightarrow$ \textit{kop} `kick'}; }
]
\end{forest}

Instead of spelling out N$_{3}$ by \textsc{stay}, N$_{3}$ is spelled out following the evacuation of the node spelled out at the previous cycle, as shown in \Next. If the lexical entry for \textit{-aj} has a foot in N$_{3}$, then the N$_{3}$P remnant can be now spelled out as the \textit{-aj} suffix on the \isi{root}.

\ex. \begin{forest}nice empty nodes, for tree={l sep=0.65em,l=0,calign angle=63} 
[~, s sep=35pt [N$_{2}$P,name=tgt  [N$_{2}$]
[N$_{1}$P [N$_{1}$]]
]{\draw (.east) node[right]{$\Rightarrow$ \textit{kop}}; }
[N$_{3}$P [N$_{3}$][...,name=t]
]{\draw (.east) node[right]{$\Rightarrow$ \textit{aj}}; }
]
\draw[dashed,->,>=stealth] (t) [in=-140,out=-130,looseness=2.25]  to (tgt);
\end{forest}

\vskip -0.75cm
Subsequent mergers of the \isi{feature}s ranging from the \is{light Get} up to the \isi{iterative} Asp are spelled out in the same way, by successive cyclic movement of N$_{2}$P \textit{kop}, as shown in the following. 

\ex.\label{bckt:aj}

\vskip -0.9cm
{\small \begin{forest}nice empty nodes, for tree={l sep=0.65em,l=0,calign angle=63}
[~,s sep=25pt 
[N$_{2}$P,name=F6 [N$_{2}$][N$_{1}$P [N$_{1}$]]
]{\draw (.east) node[right]{$\Rightarrow$ \textit{kop}}; }
[AspP, s sep=15pt [Asp] 
[,s sep=5pt [...,name=F5] 
[V$_{2}$P,s sep=12pt [V$_{2}$]
[,s sep=5pt [...,name=F4] 
[V$_{1}$P,s sep=10pt [V$_{1}$]
[,s sep=0pt [...,name=F3] 
[GiveP,s sep=8pt [Give]
[,s sep=5pt [..., name=F2][GetP [Get]
[ [...,name=N3]
[N$_{3}$P [N$_{3}$]]
]]]]]]]]]]{\draw (.east) node[right]{$\Rightarrow$ \textit{aj}}; }
]
 \draw[dashed,->,>=stealth] (F3) to[out=south west,in=south west,looseness=1.7] (F4);
  \draw[dashed,->,>=stealth] (F4) to[out=south west,in=south west,looseness=1.7] (F5);
   \draw[dashed,->,>=stealth] (F5) to[out=south west,in=south west,looseness=1.9] (F6);
    \draw[dashed,->,>=stealth] (F2) to[out=south west,in=south west,looseness=1.7] (F3);
     \draw[dashed,->,>=stealth] (N3) to[out=south west,in=south west,looseness=1.5] (F2);
 \end{forest}}  

\vskip 0.25cm
\noindent The insertion of \textit{-aj} in AspP in \ref{bckt:aj} is possible if its lexical entry is defined as in the following:\is{thematic suffix}

\ex.\label{alt:aj} Lexical entry for the \textit{-aj} theme (alternative to \ref{final:aj})\\[0.5ex]
[ Asp [ V$_{2}$ [ V$_{1}$ [ Give [ Get [ N$_{3}$ ]]]]]] $\Leftrightarrow$ \textit{-aj}

\noindent
However, while the entry defined as in \ref{alt:aj} will be inserted in the AspP in accusative iteratives based on  semelfactives like \textit{kop-a-t}, it
will not be inserted in the AspP in the other two kinds of \textit{-aj} stems that alternate with \textit{-n-ou} stems: those based on unergative \isi{semelfactive}s like \textit{gwizd-a-\'c} `whistle repeatedly' and those based on prefixed roots of  degree achievements like \textit{za-mrz-a-t} `freeze repeatedly'.  
When compared to the representation in \ref{bckt:aj}, the first include an extra V$_{3}$P layer (cf. \ref{unerg:gwizd}); the second lack two layers: GiveP and V$_{2}$P (cf. \ref{so:zamarzaj}). 
The insertion of \textit{-aj} into the AspP that dominates structures that shrink \is{shrinking} in the middle is possible in derivations involving subextract since it relies on \isi{pointer}s to earlier spell-outs as \textit{-n} and \textit{-ou}. The same solution is unavailable for the derivation involving \isi{backtracking}. This is so since for \textit{-aj} to be inserted in AspP in \ref{bckt:aj}, its lexical entry must not include a \isi{pointer} to \textit{-n} and \textit{-ou}, as these \isi{morpheme}s are not formed in the \isi{backtracking} derivation. Assuming the way the discussion of the  alternation between \textit{-n-ou} and \textit{-aj} stems has been set up, this constitutes an argument in favor of the analysis based on subextract over the analysis based on \isi{backtracking}. \is{subextraction}

\subsection{Shrinking \is{shrinking} at the root?}

An essential theoretical contrast between subextract and backtracking is that in the \isi{backtracking} derivation, the \isi{root} constituent shrinks. \is{subextraction} As illustrated in \sectref{sec:back} with an abstract sequence of \isi{feature}s, \textit{ROOT} in a backtracking derivation in \ref{so:back} spells out a subset structure spelled out as \textit{ROOT} in a derivation involving subextract in \ref{so:Z2nd}. This is also the case with the subset \isi{spell-out} of the root \textit{kop} `kick' in the \isi{backtracking} derivation discussed above. 
Thus, the question is whether this theoretical contrast is linked to an empirical difference. Specifically, what needs to be considered is the fact whether the form of the root stays the same in the \isi{semelfactive} and in the \isi{iterative}. If it always does, this fact may constitute an argument in favor of the \isi{subextraction}. If the \isi{root} alternates, this may be a potential argument in favor of the \isi{backtracking} analysis.
\par
Such an alternation indeed exists in a subset of \ili{Czech} roots. Namely, the vowel in the root of the iterative \textit{-aj} stem either shortens or lengthens, as shown in the following.

\ex.\label{short} Shortening (\ili{Czech})\is{shortening}
\a. \v{s}l\'ap-n-ou-t -- \v{s}lap-a-t (`step on once / repeatedly')
\b.  hr\'ab-n-ou-t -- hrab-a-t (`rake once / repeatedly')
\c. \v{r}\'iz-n-ou-t -- \v{r}ez-a-t (`cut once / repeatedly')
\d. \v{c}\'is-n-ou-t -- \v{c}es-a-t (`comb once / repeatedly')

\ex.\label{leng} Lengthening (\ili{Czech})\is{lengthening}
\a. \v{r}ek-n-ou-t -- \v{r}\'ik-a-t (`say once / repeatedly')
\b. st\v{r}ih-n-ou-t -- st\v{r}\'ih-a-t (`trim once / repeatedly')
\c. za-mk-n-ou-t -- za-myk-a-t (`lock once / repeatedly')
\d. po-slech-n-ou-t -- po-slouch-a-t (`listen once / repeatedly')

It has been suggested by a reviewer that since these vocalic changes in the roots exist alongside the majority of non-alternating roots, it is perhaps reasonable to treat them as cases of (mild) suppletion. If such an analysis is on the right track then the \isi{backtracking} analysis has an advantage over \isi{subextraction}, since only the first predicts that the roots in the  semelfactive-\isi{iterative alternation} lexicalize syntactic structures of different sizes. For example, under the \isi{backtracking} derivation, the \isi{root} \textit{\v{r}\'ik} `say' could realize the structure as in: \is{iterative}

\ex.
\begin{forest}nice empty nodes, for tree={l sep=0.65em,l=0,calign angle=63} 
[N$_{2}$P  [N$_{2}$]
[N$_{1}$P [N$_{1}$]]
]{\draw (.east) node[right]{$\Rightarrow$ \textit{\v{r}\'ik}}; }
]
\end{forest}

while \textit{\v{r}ek} could realize a bigger structure with a \isi{pointer} to \textit{\v{r}\'ik}, as in the following:

\ex. 
\begin{forest}nice empty nodes, for tree={l sep=0.65em,l=0,calign angle=63}
 [N$_{3}$P, s sep=16pt [N$_{3}$][\textit{\v{r}\'ik}, edge+={->, line width=0.15mm}] 
 ]{\draw (.east) node[right]{$\Rightarrow$ \textit{\v{r}ek}}; } 
\end{forest}

\noindent
However, there exists a possible alternative account of the changing roots in the \isi{iterative alternation} in \ili{Czech}. Since we find vocalic changes in both directions (both vowel \isi{shortening} and vowel \isi{lengthening} takes place), this alternation strongly appears to be an instance of a templatic effect, rather than a case of (mild) root suppletion. More specifically, it has been argued in \citeauthor{Scheer2003} (\citeyear{Scheer2003,Scheer2011}) that the \isi{spell-out} of the iterative stems is regulated by a prosodic template, which governs the distribution of vowel length. Assuming the structure of the \ili{Slavic} verb stem that comprises the \isi{root} and a separate \isi{thematic suffix}, Scheer argues there exists a template that constrains the shape of iterative stems in \ili{Czech}, which states the following:

\ex. \ili{Czech} \isi{iterative}s weigh exactly 3 morae (\citealt[112]{Scheer2003}).

In order to satisfy this restriction, the suffixation of a heavy root with the heavy thematic suffix such as the iterative \textit{-ova}, will require vowel \isi{shortening} to take place in the root. For example, the long vowel in \textit{\v{s}l\'ap-n-ou-t} `step on' becomes short in \textit{\v{s}lap-ov-a-t} `step on repeatedly'. The templatic \isi{shortening} is not restricted to roots that form \textit{-n-ou} stems, as seen in \textit{v\'y\v{s}-i-t} -- \textit{vy\v{s}-ov-a-t } `elevate'. In turn, the suffixation of a light \isi{root} with a light \isi{iterative} \isi{thematic suffix}, will require vowel \isi{lengthening} to take place in the root. For example, the short vowel in \textit{\v{r}ek-n-ou-t} `say once' becomes long in \textit{\v{r}\'ik-a-t} `say repeatedly' when it merges with the short iterative suffix \textit{-aj}. Iterative lengthening applies also to roots that do not form \textit{-n-ou} stems, as for instance  \textit{sko\v{c}-i-t} -- \textit{sk\'ak-a-t} `jump'.
\par
The change of the vowel length that is restricted by a prosodic template accounts for the examples involving \isi{lengthening} in the root in a non-arbitrary way. More generally speaking, such an account belongs to a body of work that reanalyses instances of (mild) \isi{allomorphy} that targets roots or affixes in predictable phonological terms (\citealt{Steriade2016} and \citealt{Kiparsky2018} being recent examples). 
\par
However, assuming that the \textit{-aj} theme always weighs one mora, then the list of \isi{root}s involving \isi{shortening} in \ref{short} all constitute counter-examples that must be controlled for. \citet[115]{Scheer2003} states that both the examples with shortening in \ref{short} as well as examples without the expected lengthening, e.g. \textit{pad-n-ou-t -- pad-a-t} `fall down once/repeatedly', 
 indicate that the attested cases of \isi{iterative} shortening and \isi{lengthening} are lexically recorded properties of templatic \isi{activity} that was once active in the history of \ili{Czech} but is no longer active synchronically. An argument in favor of the non-synchronic status of the iterative template is that it is no longer a productive process. The example provided in \cite{Scheer2003} involves the lack of  lengthening in \textit{klik-n-ou-t} -- \textit{klik-a-t} `click (computer)'. If the templatic restriction was active in present day \ili{Czech}, we would expect a bi-moraic stem in \textit{klik-a-t} to undergo lenghtening. With \textit{kl\'ik-a-t} rejected by native speakers of \ili{Czech}, this is unconfirmed.


\section{Remaining issues}

There are two remaining issues that must be pointed out in the discussion of the alternation between perfective \textit{-n-ou} stems and \isi{iterative} \textit{-aj} stems. The first concerns what can be called the \textit{-n-ou} drop: the fact that certain forms of \isi{semelfactive}s can occur without \textit{-n-ou} morphology but will still produce \textit{-aj} iteratives. The other concerns the observation that there are examples of stems where the \textit{-aj} theme seems to stack on top of the \textit{-n} suffix.

\subsection{\textit{-N-ou} drop}

The analysis of the alternation rests on the idea that the input to the formation of iterative 
\textit{-aj} stems includes not only bare roots of \isi{semelfactive}s and perfectivized degree achievements but their stems, i.e. the sequences \textit{ROOT-n-ou}. An argument in favor of such a setup has been the fact that the \textit{-aj} stems derived from these two categories preserve their \isi{argument structure}, which is associated with the \textit{-ou} suffix, not the bare \isi{root}. This fact serves as an argument in favor of either the \isi{subextraction} analysis or the \isi{backtracking} analysis of the  alternation, since both these alternatives rely on the presence of the syntactic representation of the argument structure projected on top of the root.
\par
However, as pointed out by a reviewer, \isi{semelfactive}s are known to occur also without \textit{-n-ou}, most productively with the past \textit{l}-participle, yielding double forms, such as shown for \ili{Czech} in the following:

\exg. Jan \{ kop-n-u-l  {/} {kop-l  \}} m\'i\v{c}.\\
Jan.\textsc{nom} {}  kick-\textsc{give-ou-part} {} kick-\textsc{part}  ball.\textsc{acc}\\ 
\strut `Jan kicked the ball.'

The possibility to drop \textit{-n-ou} holds also in  degree achievements, as shown for \ili{Czech} in the following:

\exg. Jan \{ bled-n-u-l  {/} {bled-l  \}}.\\
Jan.\textsc{nom} {}  pale-\textsc{get-ou-part} {} pale-\textsc{part}  \\ 
\strut `Jan was getting pale.'

This raises the question about the input to the \isi{iterative alternation}, namely whe-ther forms like \textit{kop-a-l} `kicked repeatedly' are derived from the \textit{-n-ou} stem or from the bare root. The second option would involve an unremarkable increase in the number of suffixes. Putting aside the argument from the conservation of the argument structure, the preservation of the idea that 
the alternation targets the \textit{-n-ou} stems rather than their bare roots depends on the analysis of the \textit{-n-ou} drop. The grammatical environment for the disappearing \textit{-n-ou} constitutes a reason to link it with the forms of the higher \textit{l}-participle rather than with the root, though.
\par
While there is variation among \ili{Czech} speakers, the \textit{-n-ou} sequence tends to appear only in the masculine singular form of the past \textit{l}-participle and it tends to drop throughout singular and plural forms of the participle. This can be illustrated with the following examples from \cite{LTN}:

\ex.
\a. kop-(n-u)-l \\
kick-\textsc{(give-ou)-part.3.msc.sg}
\bg. kop-($^{??}${n-u)-l}-{\{\,a / i / o\,\}}\\
kick-($^{??}$\textsc{give-ou)-part}-{other than \textsc{msc.sg}}\\
\strut `gave a kick'

\ex.
\a. bled-(n-u)-l \\
pale-\textsc{(give-ou)-part.msc.sg}
\bg. bled-($^{??}$n-u)-l-{\{\,a / i / o\,\}}\\ 
pale-\textsc{($^{??}$get-ou)-part}-other than \textsc{msc.sg}\\
\strut `got pale'

The drop is much harder to obtain in \ili{Polish} than it is in \ili{Czech}. By and large, it seems the easiest to obtain in 3rd person feminine and neuter singular rather than masculine, as shown in:

\ex. 
\ag. kop-*(n-\k{a})-\l{}\\
kick-*\textsc{(give-ou)-part.3.msc.sg}\\
\bg. kop-$^{??}$(n-\k{e})-\l{}-{\{\,a / o\,\}}\\
kick-$^{??}$\textsc{(give-ou)-part-{3.fem.sg / 3.neu.sg}}\\
\strut `gave a kick'

\subsection{\textit{-Aj} on top of \textit{-n}}

There are some some examples in \ili{Czech} where \textit{-aj} seems to attach on top of \textit{-n}, as in the following examples:

\ex.Czech\label{zapus}
\ag. za-p-n-ou-t -- za-p\'i-n-a-t\\
 \textsc{pref}-switch-\textsc{n-ou-inf} {} \textsc{pref}-switch-\textsc{n-aj-inf}\\
 \strut `switch on / repeatedly'
\bg. u-s-n-ou-t -- u-s\'i-n-a-t\\
\textsc{pref}-fall.asleep-\textsc{n-ou-inf} {} \textsc{pref}-fall.asleep-\textsc{n-aj-inf}\\
\strut `fall asleep / repeatedly'\label{cz:usinat}

The fact that we are able to form participles with the \textit{-n-ou} drop, \textit{za-p-l} `swiched on' and \textit{u-s-l} `he fell asleep', suggests that the roots are \textit{p-} and \textit{s-}, respectively. The existence of forms like in \ref{zapus} thus seems to suggests that if \textit{-aj} can attach on top of \textit{-n} then perhaps the majority of forms where it does not should be treated as derived from bare roots.
\par
For what it's worth, such a conclusion at the very least requires controlling for the status of the root-final \textit{n}.
\par
First, the status of \textit{p-} and \textit{s-} as roots in \textit{zapnout} and \textit{usnout} is challenged by the fact that, by and large, \ili{Czech} roots are phonological structures bigger than a single consonant (with the theme vowel \is{thematic suffix} often complementing a CVC root in a CVCV stem). This can suggest that the \textit{-n} belongs to the root in \textit{za-pn-ou-t} and \textit{u-sn-ou-t}, in which case the light \isi{verb} structure present in  semelfactives would be realized by the roots \textit{p}V\textit{n}- and \textit{s}V\textit{n}- and their prefixes, which jointly form semelfactive bases for the merger with the theme \textit{-ou}. If so, then \textit{-aj} does not stack on top of the light verb suffix \textit{-n} but simply replaces the theme vowel \textit{-ou} in \textit{za-p\'in-a-t} and \textit{u-s\'in-a-t}. While this calls for an explanation why \textit{-aj} replaces \textit{-ou} in these examples, \ref{zapus} are not genuine examples of \textit{-aj} stacking on top of the light \textit{-n} suffix. 
\par
Second, a related possibility to consider is a situation where \textit{p-} is a contextual allomorph \is{allomorphy} of \textit{p}V\textit{n-} before the participle as in \textit{za-p-l} and \textit{s-} is an allomorph of \textit{s}V\textit{n-} in \textit{u-s-l}. A circumstantial argument that can support -- or at least allow not to reject such a hypothesis right away -- is the fact that in \ili{Polish}, the equivalent of the \ili{Czech} iterative in \ref{cz:usinat} includes a suppletive root, as shown in the following:

\ex. Polish
\ag.[] \hspace{-22pt}za-s-n-\k{a}-\'c -- za-sypi-a-\'c\\
\hspace{-22pt}\textsc{pref}-sleep-\textsc{n-ou-inf} {} \textsc{pref}-fall.asleep-\textsc{aj-inf}\\
\hspace{-22pt}\strut  `fall asleep / repeatedly'

The root in \textit{za-sn-\k{a}-\'c} appears to be the same as in the noun \textit{sen} `a dream' or in the \isi{verb} \textit{\'sn-i-\'c} `to dream', where the shape of the  \textit{s}V\textit{n} root is clearer than in the \ili{Czech} example. The suppletive root in \textit{za-sypi-a-\'c} is shared with the verb \textit{sp-a-\'c} `sleep'.

\section{Concluding remarks}

There is no doubt that the list of `remaining issues' could continue in the domain of possible and impossible  alternations with the \textit{-aj} theme. \is{thematic suffix} Instead of trying to bring here all possible and impossible structures of roots and stems that can be inputs to the alternations, I have concentrated on an interesting instance of a predictable alternation that involves \textit{-n-ou} stems. On the proviso that the alternation is derivationally related, it results in the \isi{reduction} in the number of affixes on the root.
\par
Working with phrasal \isi{spell-out}, I have considered two alternative possibilities for deriving this \isi{reduction}, with \isi{subextraction} and with \isi{backtracking}, and have pointed out some of the strengths and possible challenges for both. Adopting subextraction means that the existing list of spell-out driven movements discussed in \cite{Starke2018} must be extended to the effect that it includes all
three kinds of attested phrasal movement: snowballing, spec-to-spec movement, and \isi{subextraction}.
\par
The data discussed in this chapter does not indicate how these  movements should be ordered with respect to one another. One possibility is to follow the logic of trying to move first as little as possible and order \isi{subextraction} before spec-to-spec movement and snowballing, an option suggested to me by \ia{Caha, Pavel} Pavel Caha (p.c.). An alternative possibility is to try to move first the node that is closest to the \isi{feature} targeted by \isi{spell-out} at a given cycle. In that case, the order of attempted movements will be reversed: spell-out will first try to target the complement node, then the specifier node, and then its internal node.
\par
 Both these ordering possibilities also raise the question if the so-called deep extractions (\isi{subextraction}s from an even more embedded node) are also attested as movements resulting in the \isi{spell-out} of a newly added \isi{feature}. I leave these questions open at this point.
The argumentation in the subsequent chapters will not rely on subextraction. Instead,
I will concentrate on how the problems with morphological \isi{containment} and syncretic alignment in the domain of declarative complemenizers and related categories can be resolved using phrasal spell-out and the \isi{spell-out} procedure in a more general sense. \is{syncretism} By that I understand the existence of a grammar where the merger of a \isi{feature} is followed by an attempt to spell it as part of the syntactic tree either ``as is'', following a movement operation, or following a subderivation.

   
 
 
 
